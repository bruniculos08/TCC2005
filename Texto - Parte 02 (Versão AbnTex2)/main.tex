% abnTeX2: Modelo de Trabalho Academico (tese de doutorado, dissertacao de
% mestrado e trabalhos monograficos em geral) em conformidade com 
% ABNT NBR 14724:2011: Informacao e documentacao - Trabalhos academicos - Apresentação
% Disponível em: https://github.com/abntex/abntex2/blob/master/doc/latex/abntex2/examples/abntex2-modelo-trabalho-academico.tex

% Configurações do documento:
\documentclass[12pt, openright, twoside, a4paper, english, french, spanish, brazil]{abntex2}

% Pacotes básicos:
\usepackage{lmodern}
\usepackage[T1]{fontenc}
\usepackage[utf8]{inputenc}
\usepackage{indentfirst}
\usepackage{color}
\usepackage{graphicx}
\usepackage{microtype}

% Pacotes adicionais, usados apenas no âmbito do Modelo Canônico do abnteX2:
\usepackage{lipsum}

% Pacotes de citações:
\usepackage[brazilian,hyperpageref]{backref}
\usepackage[alf]{abntex2cite}

% Configurações do pacote backref:
\renewcommand{\backrefpagesname}{Citado na(s) página(s):~}
\renewcommand{\backref}{}
\renewcommand*{\backrefalt}[4]{
	\ifcase #1 
		Nenhuma citação no texto.
	\or
		Citado na página #2.
	\else
		Citado #1 vezes nas páginas #2.
	\fi
}

% Capa e Folha de Rosto:
% Informações para CAPA e FOLHA DE ROSTO:
\titulo{Formalização do \textit{Símbolo de Legendre} em \textit{Coq}}
\autor{Bruno Rafael dos Santos}
\local{Brasil}
\data{\today}
\orientador{Karina Girardi Roggia}
\coorientador{Paulo Henrique Torrens}
\instituicao{
  Universidade do Estado de Santa Catarina -- UDESC
  % \par
  % Departamento de Ciência da Computação
  \par
  Bacharelado em Ciência da Computação
}

\tipotrabalho{Trabalho de Conclusão de Curso}
% O preambulo deve conter o tipo do trabalho, o objetivo, 
% o nome da instituição e a área de concentração 
\preambulo{Trabalho de Conclusão de Curso apresentado ao curso de Bacharelado em Ciência da Computação do Centro de Ciências Tecnológicas da Universidade do Estado de Santa Catarina, como requisito parcial para a obtenção do grau de Bacharel em Ciência da Computação.}

% \preambulo{Trabalho de Conclusão de Curso apresentado como requisito parcial para obtenção do título de Bacharelado em Ciência da Computação pelo Centro de Ciências Tecnológicas da Universidade do Estado de Santa Catarina, como requisito parcial para a obtenção do grau de Bacharel em Ciência da Computação.}
% Outras configurações:
% ---- Arquivo com as configurações do PDF

% Alteração para fonte de capítulos:
\renewcommand{\ABNTEXchapterfont}{\normalfont}
\renewcommand{\ABNTEXchapterfont}{\bfseries}

% Definindo a cor azul em RGB:
\definecolor{blue}{RGB}{41,5,195}

% Informações do PDF:
\makeatletter
\hypersetup{
		pdftitle={\@title}, 
		pdfauthor={\@author},
    	pdfsubject={\imprimirpreambulo},
	    pdfcreator={Bruno Rafael dos Santos},
		pdfkeywords={Algoritmo \textit{RESSOL}}{Algoritmo de Tonelli-Shanks}{Lei de Reciprocidade Quadrática}{abntex2}{trabalho acadêmico}, 
		colorlinks=true,
    	linkcolor=blue,
    	citecolor=blue,
    	filecolor=magenta,
		urlcolor=blue,
		bookmarksdepth=4
}
\makeatother

% Posiciona figuras e tabelas no topo da página quando adicionadas sozinhas em uma página em branco (ver https://github.com/abntex/abntex2/issues/170):
\makeatletter
\setlength{\@fptop}{5pt} 
\makeatother

% Possibilita criação de Quadros e Lista de quadros (ver https://github.com/abntex/abntex2/issues/176):
\newcommand{\quadroname}{Quadro}
\newcommand{\listofquadrosname}{Lista de quadros}
\newfloat[chapter]{quadro}{loq}{\quadroname}
\newlistof{listofquadros}{loq}{\listofquadrosname}
\newlistentry{quadro}{loq}{0}

% Configurações para atender às regras da ABNT
\setfloatadjustment{quadro}{\centering}
\counterwithout{quadro}{chapter}
\renewcommand{\cftquadroname}{\quadroname\space} 
\renewcommand*{\cftquadroaftersnum}{\hfill--\hfill}

\setfloatlocations{quadro}{hbtp} % Ver https://github.com/abntex/abntex2/issues/176

% O tamanho do parágrafo é dado por:
\setlength{\parindent}{1.3cm}

% Controle do espaçamento entre um parágrafo e outro:
\setlength{\parskip}{0.2cm}  % tente também \onelineskip

% Configurações adicionadas por Bruno:
	% Teoremas e etc:
		\newtheorem{definição}{Definição}
		\newtheorem{teorema}{Teorema}
		\newtheorem{lema}{Lema}
	% Operadores e etc:
		\usepackage{amsmath, amsfonts, amssymb}
		\usepackage{mathtools}
		\DeclareMathOperator{\mdc}{mdc}
		\DeclareMathOperator{\mmc}{mmc}
		\DeclarePairedDelimiter\abs{\lvert}{\rvert}
		\usepackage{proof}
		\usepackage{mathpartir}
	% Highlight de código em coq:
		\definecolor{violet}{RGB}{80,5,100}
		\definecolor{teal}{RGB}{0,128,128}
		\definecolor{orange}{RGB}{255,128,13}
		\definecolor{darkgreen}{RGB}{0,100,0}
		\usepackage{listings, Estilos/coq, Estilos/coq-error}

% Compilar índice:
\makeindex

\begin{document}
\selectlanguage{brazil}
\frenchspacing 
	\pretextual
	\imprimircapa
	\imprimirfolhaderosto*{}
	% % \begin{fichacatalografica}
% 	\sffamily
% 	\vspace*{\fill}					% Posição vertical
% 	\begin{center}					% Minipage Centralizado
% 	\fbox{
%         \begin{minipage}[c][7cm]{12.5cm}		% Largura
%         % \begin{minipage}[c][8cm]{13.5cm}		% Largura
%         \small
%         \imprimirautor
%         %Sobrenome, Nome do autor
        
%         \hspace{0.5cm} \imprimirtitulo  / \imprimirautor. --
%         \imprimirlocal, \imprimirdata-
        
%         \hspace{0.5cm} \thelastpage p. : il. (algumas color.) ; 30 cm.\\
        
%         % \hspace{0.5cm} \imprimirorientadorRotulo~\imprimirorientador\\
%         % Editado por Bruno (parte de orientador e coorientador):
%         \hspace{0.5cm} \imprimirorientadorRotulo~\imprimirorientador
%         \par \hspace{0.5cm} \imprimircoorientadorRotulo~\imprimircoorientador \\
        
%         \hspace{0.5cm}
%         \parbox[t]{\textwidth}{\imprimirtipotrabalho~--~\\\imprimirinstituicao,
%         \imprimirdata.}\\
        
%         \hspace{0.5cm}
%             1. símbolo de Legendre.
%             2. criptografia.
%             3. Teoria dos Números.
%             I. Karina Girardi Roggia.
%             II. Universidade do Estado de Santa Catarina.
%             III. Faculdade de Ciência da Computação.
%             IV. Formalização do \textit{Símbolo de Legendre} em \textit{Coq}. 			
%         \end{minipage}
%     }
% 	\end{center}
% \end{fichacatalografica}

% Código de ficha catalográfica do .tex da UDESC:

\begin{fichacatalografica}
	%\sffamily
	%\rmfamily
	\ttfamily \hbadness=10000
	\vspace*{\fill}					% Posição vertical
	\begin{center}					% Minipage Centralizado
	% Para gerar a ficha catalográfica de teses e \\ 
	% dissertações acessar o link:  \\
	% https://www.udesc.br/bu/manuais/ficha
	
	\vspace*{8pt}
	
%	\begin{minipage}[c]{8cm}
%	\centering \sffamily
%	 Ficha catalográfica elaborada pelo(a) autor(a), com auxílio do programa de geração automática da Biblioteca Setorial do CCT/UDESC
%	\end{minipage}
	\fbox{
        \begin{minipage}[c]{12.5cm}		% Largura
	        \flushright
            {
                \begin{minipage}[c]{10.5cm}		% Largura
                \vspace{1.25cm}
                %\footnotesize
                \setlength{\parindent}{1.5em}
                % \noindent \invertname{\imprimirautor} \par
                \noindent dos Santos, Bruno Rafael \par
                \imprimirtitulo{ }/{ }\imprimirautor. -- \imprimirlocal, \imprimirdata .\par
                \pageref{LastPage} p. : il. ; 30 cm.\par
                \vspace{1.5em}
                \imprimirorientadorRotulo~\imprimirorientador.\par
                \imprimircoorientadorRotulo~\imprimircoorientador.\par
                \imprimirtipotrabalho~--~\imprimirinstituicao, \imprimirlocal, \imprimirdata.\par
                \vspace{1.5em}
                    1. símbolo de Legendre.
                    2. criptografia.
                    3. Teoria dos Números.
                    % 4. Palavra-chave.
                    % 5. Palavra-chave.
                    % I. \invertname{\imprimirorientador}.
                    I. Roggia, Karina Girardi.
                    % II. \invertname{\imprimircoorientador}.
                    II. Torrens, Paulo Henrique.
                    III. \imprimirinstituicao.
                    % IV. Título. %
                    IV. Formalização do \textit{Símbolo de Legendre} em \textit{Coq}. %
                \vspace{1.25cm}	%		
                \end{minipage}%
            }% 
	    \end{minipage}
    }%
	\vspace*{0.5cm}
	\end{center}
\end{fichacatalografica}
	\include{Partes/Pre-texto}
	\textual
	\begin{frame}{Introdução}
    \begin{itemize}
        \item A Teoria dos Números é um ramo da matemática que lida, em sua maior parte, com  propriedades de números inteiros;
        \item É muito presente em temas relacionados a criptografia;
        \item Envolve definições de diversas relações em $\mathbb{Z}$, sendo duas dessas as relações de divisibilidade e congruência;
        \item Neste contexto que se apresenta o \textit{símbolo de Legendre}, o qual possui relação com o algoritmo \textit{RESSOL} e está presente na \textit{Lei de Reciprocidade Quadrática}.
        % algoritmo \textit{RESSOL}, também conhecido como algoritmo de Tonelli-Shanks, e a Lei de Reciprocidade Quadrática;
        \item A seguir se apresentam as definições de divisibilidade e congruência.
    \end{itemize}
\end{frame}

\begin{frame}{Introdução}
    \begin{definicao}[\textit{Divisibilidade}]
            $\forall d, a \in \mathbb{Z}$, \textbf{$d$ divide $a$} (ou em outras palavras: $a$ é um múltiplo de $d$) se e somente se a seguinte proposição é verdadeira:
            \begin{equation*}
                \exists q \in \mathbb{Z}, a = d \cdot q
            \end{equation*}
            assim, se tal proposição é verdadeira e portanto $d$ divide $a$, tem-se a seguinte notação que representa tal afirmação:
            \begin{equation*}
                d \mid a
            \end{equation*}
            caso contrário, a negação de tal afirmação ($d$ não divide $a$) é representada por:
            \begin{equation*}
                d \nmid a
            \end{equation*}
    \end{definicao}
\end{frame}

\begin{frame}{Introdução}
    \begin{definicao}[\textit{Congruência}]
        Para todo $a, b, n \in \mathbb{Z}$, $a$ é congruente a $b$ módulo $n$ se e somente se, pela divisão euclidiana $\frac{a}{n}$ e $\frac{b}{n}$ (onde $0 \leq r_{a} < |n|$ e $0 \leq r_b < |n|$) tem-se
        \begin{equation*}
            a = n \cdot q_a + r_a
        \end{equation*}
        e
        \begin{equation*}
            b = n \cdot q_b + r_b
        \end{equation*}
        com $r_a = r_b$, o que também equivale a dizer que:
        \begin{equation*}
            n \mid a - b
        \end{equation*}
        tal relação entre os inteiros $a$, $b$ e $n$ é representada por:
        \begin{equation*}
            a \equiv b \pmod{n}
        \end{equation*}
    \end{definicao}
\end{frame}

% \begin{frame}{Introdução}
%     \begin{itemize}
%         \item Sendo $p$ um número primo e $r, n \in \mathbb{Z}$, uma congruência quadrática é uma equação da seguinte forma:
%         \begin{equation*}
%             r^2 \equiv n \pmod{p}
%         \end{equation*}
%         \item O objetivo do algoritmo \textit{RESSOL} é, tendo os valores de $p$ e $n$, computar um valor de $r$ que satisfaça tal equação;
%         \item Inicialmente, um algoritmo para resolução deste problema foi publicado em \cite{Tonelli1891};
%         \item Mais tarde foi publicada uma nova versão em \apud{danielShanks}{Maheswari}, que é a versão a ser tratada neste trabalho;
%         \item Algumas de suas aplicações são: \textit{Rabin Cryptosystem} \cite{Huynh1581080} e sistemas de criptografia que envolvem curvas elípticas \cite{PalashSarkar2024AdvancesinMathematicsofCommunications}.
%         % \cite{kumar2021algorithm} e \cite{7133812}.
%     \end{itemize}
% \end{frame}

% \begin{frame}{Introdução}
%     \begin{itemize}
%         \item Outro tema também abordado neste trabalho (porém não nesta apresentação) é a Lei de Reciprocidade Quadrática.
%     \end{itemize}
% \end{frame}


	\bibliography{referencias}
\end{document}