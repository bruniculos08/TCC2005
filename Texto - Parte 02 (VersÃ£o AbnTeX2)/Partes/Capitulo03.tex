\chapter{Base Teórica}
\label{cap:base}

Para que se realizem as implementações serão necessários diversos teoremas, lemas e funções, dos quais, parte, já estão implementados na biblioteca Mathematical Components. Sendo assim, o presente capítulo busca trazer a descrição da maioria destes itens, colocando também suas respectivas implementações disponíveis na biblioteca (se houver). 

A maior parte do conteúdo deste capítulo se baseia no livro \cite{book:2399854} e em \cite{euler-crit-planet}, que foi amplamente estudado para realização deste trabalho. Sendo assim, as provas não apresentadas aqui se encontram nesse livro.

\section{Máximo Divisor Comum}
Tendo sido apresentado os conceitos de módulo e divisibilidade, outro pilar fundamental da Teoria dos Números é o conceito de $\mdc$ (máximo divisor comum). A princípio, a definição seria auto-explicativa, mas parte de diversos teoremas importantes a utiliza e portanto é interessante que se tenha uma definição equivalente específica para facilitar provas futuras. Para se obter tal definição alternativa, observe que, formalmente, a definição de $\mdc$ é:
    \begin{equation*}
        \forall a, b, n \in \mathbb{Z},  \mdc(a, b) = n \Leftrightarrow (n \mid a) \land (n \mid b) \land (\forall k \in \mathbb{Z}, (k \mid a) \land (k \mid b) \rightarrow k \leq n)
    \end{equation*}
Analisando a proposição:
    \begin{equation*}
        \forall k \in \mathbb{Z}, (k \mid a) \land (k \mid b) \rightarrow k \leq n
    \end{equation*}
Pode-se verificar os seguintes casos:
    \begin{enumerate}
        \item $|k| = |n|$, isto é, $k = n$ ou $k = -n$. Em ambos estes casos $k \mid n$.

        \item $|k| < |n|$, assim, como $n \mid a$ e $n \mid b$, então, $\exists q_a, q_b \in \mathbb{Z}, (a = q_a \cdot n \land b = q_b \cdot n) $, onde $\mdc(q_a, q_b) = 1$ (caso contrário $n$ não seria $\mdc(a,b)$, pois haveria o divisor $\mdc(q_a, q_b) \cdot n$ que é maior que $n$, se $n > 1$). Portanto, como $k \mid a$ e $k \mid b$, então $k \mid q_a \cdot n$ e $k \mid q_b \cdot n$, mas sabe-se que $k \nmid q_a$ e $k \nmid q_b$, logo, $k \mid n$.
    \end{enumerate}
Olhando o que acontece quando $k \mid n$, é fácil notar que $|k| \leq |n|$, logo, no contexto em que $k \mid a$ e $k \mid b$, as afirmações $k \leq n$ e $k \mid n$ são equivalentes. Sendo assim tem-se a seguinte definição alternativa:
\begin{equation*}
        \forall k \in \mathbb{Z}, (k \mid a) \land (k \mid b) \rightarrow k \mid n
\end{equation*}    

\section{Algoritmo de Euclides}
\label{sec:algoritmo-de-euclides}

O algoritmo de Euclides é um método de computar o $\mdc$ entre dois números inteiros $a$ e $b$, tendo a seguinte descrição (Algoritmo \ref{algo:euclides}):

        \begin{algorithm}[!htbp]
                \SetAlgoLined
                \vspace{3mm}
                \Entrada{$a, b \in \mathbb{Z}$}
                \Saida{inteiro $n$.}
                \Se{$a = 0$}
                {\Retorna{$b$}}
                \SeN
                {\Retorna{\hyperref[algo:euclides]{\textsc{Euclides}}($b$, $a\bmod b$)}}
                \caption{\textsc{Euclides}}\label{algo:euclides}
        \end{algorithm}

\noindent
Esse algoritmo se baseia no seguinte lema:
\begin{lema} \label{lema:euclides}
    $\forall a, b \in \mathbb{Z}$, seja $a = b \cdot q  + r$, onde $0 \leq r < |b|$, então:
    \begin{equation*}
        \mdc(a, b) = \mdc(b, r)
    \end{equation*}
\end{lema}
    % \begin{proof}
\noindent
\textit{Demonstração}: inicialmente deve-se notar que, provar tal lema equivale a demonstrar:
        \begin{equation} \label{eq : 1}
            \forall n \in \mathbb{Z}, (\mdc(a, b) = n \leftrightarrow \mdc(b, r) = n)
        \end{equation}
        Além disso, deve-se considerar o seguinte lema trivial que versa sobre combinações lineares:
        \begin{lema}[\textit{Divisibilidade e combinações lineares}]
            \begin{equation*}
                \forall a, b, n \in \mathbb{Z}, (n \mid a) \land (n \mid b) \rightarrow \forall c_1, c_2 \in \mathbb{Z}, n \mid (c_1 \cdot a + c_2 \cdot b)
            \end{equation*}
        \end{lema}
        Dados esses adendos, prova-se inicialmente a volta da bi-implicação \ref{eq : 1}. Para isso, observe que, se $x = \mdc(b, r)$ então $x \mid b$ e $x \mid r$, e como $r = a - b \cdot q$ então $x \mid (a - b \cdot q)$. Pode-se então fazer uma combinação linear escolhendo $c_1 = q$ e $c_2 = 1$, donde se chega em:
        \begin{equation*}
                x \mid q \cdot b + 1 \cdot (a - b \cdot q)
        \end{equation*}
        \begin{equation*}
                x \mid q \cdot b + a - b \cdot q
        \end{equation*}
        \begin{equation*}
                x \mid a
        \end{equation*}
        Resta então provar que:
        \begin{equation*}
            \forall y \in \mathbb{Z}, (y \mid a) \land (y \mid b) \rightarrow (y \mid x) 
        \end{equation*}
        Da hipótese tem-se que
        \begin{equation*}
            \forall y \in \mathbb{Z}, (y \mid b) \land (y \mid r) \rightarrow (y \mid x) 
        \end{equation*}
        Se $y \mid a$ e $y \mid b$, como $a = b \cdot q + r$ então
        $y \mid (b \cdot q + r)$. Utilizando o teorema sobre combinação linear novamente, com $c_1 = -q$ e $c_2 = 1$:
        \begin{equation*}
            y \mid -q \cdot b + 1 \cdot (b \cdot q + r) 
        \end{equation*}
        \begin{equation*}
            y \mid - b \cdot q + b \cdot q + r 
        \end{equation*}
        \begin{equation*}
            y \mid r 
        \end{equation*}
        Como $y \mid b$ e $y \mid r$, da hipótese temos que $y \mid x$ portanto se provou o que restava. Para a ida da bi-implicação a prova é semelhante. \qed
    % \end{proof}
% \end{lema}

O algoritmo \hyperref[algo:euclides]{\textsc{Euclides}} é implementado da seguinte forma \footnote{Pode-se observar nesta implementação e em outras apresentadas neste trabalho, o uso de notações como \codequotes\lstinline[language=coq-error]|_.+1|\codequotes, \codequotes\lstinline[language=coq]|_.-1|\codequotes, \codequotes\lstinline[language=coq]|_.+2|\codequotes, \codequotes\lstinline[language=coq]|_.-2|\codequotes, \codequotes\lstinline[language=coq]|_.*2|\codequotes\;e \codequotes\lstinline[language=coq]|_./2|\codequotes. Esse uso serve para indicar as respectivas operações, dadas por cada notação, com precedência maior sobre outras no restante da expressão.} na biblioteca Mathematical Components (em sua versão para números naturais) dada em \cite{mathcomp-div}:
    \begin{lstlisting}[language=coq,frame=single,tabsize=1]
Fixpoint gcdn m n :=
    let n' := n %% m in if n' is 0 then m else
    if m - n'.-1 is m'.+1 then gcdn (m' %% n') n' else n'.
    \end{lstlisting}
A versão para números inteiros é basicamente \lstinline[language = coq]{gcdn} aplicada sobre o valor absoluto dos números inteiros de entrada e está implmentada em \cite{mathcomp-intdiv} com nome \coqinline[]{gcdz}.
% \begin{lstlisting}[language = coq]
%     Definition gcdz m n := (gcdn `|m| `|n|)%:Z.
% \end{lstlisting}
Sobre a notação \lstinline[language = coq] essa representa a operação de resto da divisão, que, para números naturais é definida em \cite{mathcomp-div} por:
    \begin{lstlisting}[language=coq,frame=single,tabsize=1]
Definition modn_rec d := 
    fix loop m := if m - d is m'.+1 then loop m' else m.
Definition modn m d := if d > 0 then modn_rec d.-1 m else m.
Notation "m %% d" := (modn m d) : nat_scope.
    \end{lstlisting}
E utilizando a função 
\lstinline[language = coq]{divz}, definida em \cite{mathcomp-intdiv}, para divisão entre números inteiros, a operação de resto da divisão é implementada da seguinte forma (para inteiros):
    \begin{lstlisting}[language=coq,frame=single,tabsize=1]
Definition modz (m d : int) : int := m - divz m d * d.
Infix "%%" := modz : int_scope.
    \end{lstlisting}

\section{Teorema Bachet-Bézout}

Uma das consequências teóricas da definição de $\mdc$ é o Teorma Bachet-Bézout. Este por sua vez traz uma aplicação do conceito de $\mdc$ na resolução de equações. O seu enunciado é dado por:
\begin{teorema}[\textit{Bachet-Bézout}] \label{teorema : bachetbezout}
    $\forall a, b \in \mathbb{Z}, \exists x, y \in \mathbb{Z}$ tal que
    \begin{equation*}
        a \cdot x + b \cdot y = \mdc(a, b)
    \end{equation*}
\end{teorema}
\noindent
Como exemplo de consequências deste teorema têm-se:
\begin{enumerate}
    \item $\forall c \in \mathbb{Z}, c \mid a \land c \mid b \Rightarrow c \mid \mdc(a, b)$, ou seja, caso $c$ divida tanto $a$ quanto $b$ então $c$ divide $\mdc(a, b)$.
    \item $\forall c \in \mathbb{Z}, (\exists x, y \in \mathbb{Z}, a \cdot x + b \cdot y = c) \Longleftrightarrow \mdc(a, b) \mid c $, ou seja, a equação $a \cdot x + b \cdot y = c$ tem solução se e somente se $\mdc(a, b)$ divide $c$.
\end{enumerate}
A prova manual do Teorema \ref{teorema : bachetbezout} e as provas das consequências mencionadas se encontram em \cite[p.~20-21]{book:2399854}. 

    Em relação à biblioteca Mathematical Components e se tratando do Teorema \ref{teorema : bachetbezout}, essa possui a implementação de um algoritmo
    % do algoritmo conhecido como "algoritmo de Euclides extendido" (
    que encontra os coeficientes $x$ e $y$ e o valor de $\mdc(a,b)$. Esse algoritmo possui duas versões, sendo identificado na biblioteca como \lstinline[language = coq]{egcdn} para naturais e \lstinline[language = coq]{egcdz} para inteiros. Além disso, existem os lemas \lstinline[language = coq]{egcdnP} e \lstinline[language = coq]{egcdzP}, tratando da corretude desses algoritmos (respectivamente) e para o caso dos números inteiros, há o lema \lstinline[language = coq]{Bezoutz} que é equivalente ao Teorema \ref{teorema : bachetbezout}.

\section{Propriedades de Congruência}
As propriedade da relação de congruência serão usadas com muita frequência (e de maneira implícita) no decorrer desse documento. Por essa razão, nesta seção será apresentada uma lista de propriedades semelhante à apresentada em \cite[p.~34]{book:2399854}, para qual a demonstração de maioria dessas propriedades se encontra em \cite[p.~34-35]{book:2399854}. Recomenda-se então ao leitor consultar o conteúdo aqui apresentado em caso de dúvidas no desenvolvimento de equações modulares.
    
    Seguindo para as propriedades apresentadas em \cite{book:2399854}, para todo $a,b, c, d, n \in \mathbb{Z}$ vale que:
\begin{enumerate}
    \item \textit{Reflexividade}: 
    \\
    $a \equiv a \pmod{n}$
    % \begin{equation*} 
    %     a \equiv a \pmod{n}
    % \end{equation*}
    
    \item \textit{Simetria}: 
    \\ 
    $a \equiv b \pmod{n} \Longrightarrow b \equiv a \pmod{n}$
    % \begin{equation*}
    %     a \equiv b \pmod{n} \Longrightarrow b \equiv a \pmod{n}
    % \end{equation*}
    
    \item \textit{Transitividade}: 
    \\
    $a \equiv b \pmod{n} \land b \equiv c \pmod{n} \Longrightarrow a \equiv c \pmod{n}$
        % \begin{equation*}
        %     a \equiv b \pmod{n} \land b \equiv c \pmod{n} \Longrightarrow a \equiv c \pmod{n}            
        % \end{equation*}
    
    \item \textit{Compatibilidade com a soma}: 
    \\
    $a \equiv b \pmod{n} \land c \equiv d \pmod{n} \Longrightarrow a + c \equiv b + d \pmod{n}$
        % \begin{equation*}
        %     a \equiv b \pmod{n} \land c \equiv d \pmod{n} \Longrightarrow a + c \equiv b + d \pmod{n}
        % \end{equation*}

    \item \textit{Compatibilidade com a diferença}: 
    \\
    $a \equiv b \pmod{n} \land c \equiv d \pmod{n} \Longrightarrow a - c \equiv b - d \pmod{n}$
        % \begin{equation*}
        %     a \equiv b \pmod{n} \land c \equiv d \pmod{n} \Longrightarrow a - c \equiv b - d \pmod{n}
        % \end{equation*}

    \item \label{item:propcong6-produto} \textit{Compatibilidade com o produto}: 
    \\
    $a \equiv b \pmod{n} \land c \equiv d \pmod{n} \Longrightarrow a \cdot c \equiv b \cdot d \pmod{n}$ 
    \\
        % \begin{equation*}
        %     a \equiv b \pmod{n} \land c \equiv d \pmod{n} \Longrightarrow a \cdot c \equiv b \cdot d \pmod{n}
        % \end{equation*}
    A partir dessa propriedade, note que, para todo $k \in \mathbb{N}$:
    \\
    $a \equiv b \pmod{n} \Longrightarrow a^k \equiv b^k \pmod{n}$
        % \begin{equation*}
        %     a \equiv b \pmod{n} \Longrightarrow a^k \equiv b^k \pmod{n}
        % \end{equation*}

    \item \label{item:propcong7-cancelamento} \textit{Cancelamento}:
    \\
    $\mdc(c, n) = 1 \Longrightarrow (a \cdot c \equiv b \cdot c \pmod{n} \Longleftrightarrow a \equiv b \pmod{n})$
    \\
        % \begin{equation*}
        %     \mdc(c, n) = 1 \Longrightarrow (a \cdot c \equiv b \cdot c \pmod{n} \Longleftrightarrow a \equiv b \pmod{n})   
        % \end{equation*}
\end{enumerate}

Na biblioteca Mathematical Components, as relações de módulo são definidas pelas seguintes notações \cite{mathcomp-intdiv}:
    \begin{lstlisting}[language=coq,frame=single,tabsize=1]
Notation "m = n %[mod d ]" := (modz m d = modz n d) : int_scope.
Notation "m == n %[mod d ]" := (modz m d == modz n d) : int_scope.
Notation "m <> n %[mod d ]" := (modz m d <> modz n d) : int_scope.
Notation "m != n %[mod d ]" := (modz m d != modz n d) : int_scope.
    \end{lstlisting}
Deve-se observar que existem duas igualdades, sendo a primeira a de Leibniz e a segunda uma função booleana. O mesmo ocorre com as desigualdades (na mesma ordem).

Tais definições são equivalentes a definição apresentada em \cite[p.~34]{book:2399854}, conforme é mostrado pelo lema \lstinline[language = coq]{eqz_mod_dvd} (no caso do tipo \coqinline[]{int}) implementado na biblioteca \footnote{O operador \lstinline[language = coq]{\%Z} serve para indicar o escopo padrão como \lstinline[language=coq]|int_scope|.} em \cite{mathcomp-intdiv}:
    \begin{lstlisting}[language=coq,frame=single,tabsize=1]
Lemma eqz_mod_dvd d m n : (m == n %[mod d])%Z = (d %| m - n)%Z.
    \end{lstlisting}

Quanto a implementação das propriedades apresentadas nesta seção, a biblioteca não as implementa, apesar de fazer isso para lemas semelhantes a algumas destas propriedades. A exemplo, ainda em \cite{mathcomp-intdiv}, tem-se:
    \begin{lstlisting}[language=coq,frame=single,tabsize=1]
Lemma modzDm m n q : ((m %% q)%Z + (n %% q)%Z = m + n %[mod q])%Z.
    \end{lstlisting}
Este lema se assemelha a propriedade de \textit{compatibilidade com a soma}, no sentido de que, considerando a existência do seguinte lema também em \cite{mathcomp-intdiv}:
    \begin{lstlisting}[language=coq,frame=single,tabsize=1]
Lemma modz_mod m d : ((m %% d)%Z = m %[mod d])%Z.
    \end{lstlisting}
possuem a mesma utilidade. 

% Outro detalhe importante de se mencionar aqui é que também é possível trabalhar com esta relação (módulo) de outras maneiras usando implementações disponíveis na biblioteca. Uma destas é por meio de estruturas implementadas no módulo \textit{generic\_quotient} \footnote{\url{https://math-comp.github.io/htmldoc_2_2_0/mathcomp.ssreflect.generic_quotient.html}}.

\section{Anel de Inteiros Módulo $n$}

Uma estrutura que será utilizada no presente trabalho, e que pode ser implementada usando elementos disponíveis na biblioteca Mathcomp, são os anéis de inteiros módulo $n$. De acordo com \cite[p.~40]{book:2399854}, dada a relação $\sim$ sobre um conjunto $A$, se esta relação é uma relação de equivalência ela possui as seguintes propriedades:
% acima era \cite[p.~40-41]{book:2399854} mas mudei para \cite[p.~40]{book:2399854}
\begin{enumerate}
    \item \textbf{\textit{reflexividade}}: $\forall x \in A, x \sim x$
    \item \textbf{\textit{transitividade}}: $\forall x, y, z \in A, x \sim y \land y \sim z \rightarrow x \sim z$
    \item \textbf{\textit{simetria}}: $\forall x, y \in A, x \sim y \leftrightarrow y \sim x$
\end{enumerate}
Como exposto em \cite[p.~40-41]{book:2399854}, estabelecer uma relação de equivalência sobre um conjunto $A$ é o mesmo que definir um partição sobre o mesmo, isto é, dividir $A$ em subconjuntos, em que, sendo cada subconjunto identificado como $A_i$ onde $i$ pertence a um conjunto $I$, então
% acima era \cite[p.~40]{book:2399854} mas mudei para \cite[p.~40-41]{book:2399854}
\begin{equation*}
    A = \bigcup_{i \in I} A_{i}
\end{equation*}
Particionando $X$ por meio da relação $\sim$, tem-se que, dados $x, y \in X$, então $x, y \in X_\lambda$ se e somente se $x \sim y$. Além disso pode-se definir a \textit{classe de equivalência} $\overline{x}$ em que:
\begin{equation*}
    \overline{x} = \{y \in A \mid y \sim x \}
\end{equation*}
O conjunto de classes de equivalência $\{\overline{x} \mid x \in A\}$ é denominado quociente de $A$ por $\sim$ e é representado por $A/\sim$.

Particionando $\mathbb{Z}$ por meio da relação $\equiv \mod n$ para algum $n \in \mathbb{Z} - \{0\}$, tem-se um conjunto de \textit{classes de equivalência} denominado \textit{anel de inteiros módulo $n$}, que costuma ser representado por $\mathbb{Z}/(n)$, onde então:
\begin{equation*}
    \mathbb{Z}/(n) = \{\overline{0}, ..., \overline{n - 1}\}
\end{equation*}
Note que, no entanto, as classes de equivalência podem ser denotadas por diferentes números, desde que tenha o mesmo resto na divisão inteira por $n$. A exemplo disso observe que:
\begin{equation*}
    \overline{0} = \overline{n}
\end{equation*}
pois
\begin{equation}
    0 \equiv n \pmod{n}
\end{equation}
Portanto, para quaisquer $a, b \in \mathbb{Z}$, se $\overline{a}, \overline{b} \in \mathbb{Z}/(n)$ tem-se:
\begin{equation*}
    \overline{a} = \overline{b} \Longleftrightarrow a \equiv b \pmod{n}
\end{equation*}
Com isso, dada que as seguintes propriedades são válidas para as relações de congruência, com quaisquer $a, b, c, d \in \mathbb{Z}$:
\begin{itemize}
    \item $a \equiv b \mod n \land c \equiv d \pmod{n} \Rightarrow a + c \equiv b + d \pmod{n}$
    \item $a \equiv b \mod n \land c \equiv d \pmod{n} \Rightarrow a - c \equiv b - d \pmod{n}$
    \item $a \equiv b \mod n \land c \equiv d \pmod{n} \Rightarrow a \cdot c \equiv b \cdot d \pmod{n}$
\end{itemize}
São definidas então as operações de soma, subtração e multiplicação em $\mathbb{Z}/(n)$, das seguintes formas para todo $\overline{a}, \overline{b} \in \mathbb{Z}/(n)$
\begin{itemize}
    \item $\overline{a} + \overline{b} = \overline{a + b}$
    \item $\overline{a} - \overline{b} = \overline{a - b}$
    \item $\overline{a} \cdot \overline{b} = \overline{a \cdot b}$
\end{itemize}
Como explicado em \cite[p.~46-47]{book:2399854} e devido a estas operações, o nome \textit{anel de inteiros módulo $n$} é justificado pela definição de \textit{anel}: qualquer conjunto $A$ com duas operações binárias $+$ e $\cdot$, de modo que $A$ satisfaz as seguintes propriedades:
% é um \textit{grupo abeliano}, isto é, um grupo que além das condições para ser um grupo, é comutativo, ou seja, se $A$ é um grupo por meio da operação binária $\cdot$, então:
% \begin{equation*}
%     \forall a, b \in A, a \cdot b = b \otimes a
% \end{equation*}
% Além desta condição, devem ser também satisfeitas as seguintes \cite{book:2399854}:
\begin{itemize}
    \item $(A, +)$ é um \textit{grupo abeliano}, isto é, um grupo que possui a propriedade de \textit{comutatividade} por meio da operação binária $+$, ou seja:
    \begin{equation*}
        \forall a, b \in A, a + b = b + a
    \end{equation*}
    com elemento neutro que será denotado por $0$ (neste caso $0$ é um elemento quaisquer, e não necessariamente o número $0$).
    % pedaço de texto comentado acima pois era redundante (um grupo já tem um elemento neutro, isso não uma propriedade especial de grupos abelianos)
    % \item \textit{\textbf{Associatividade do produto}}: $\forall a, b, c \in A, (a \cdot b) \cdot c = a \cdot (b \cdot c)$
    % \item \textit{\textbf{Elemento neutro do produto:}} $\forall a \in A, \exists 1 \in A, (a \cdot 1 = 1 \cdot a)$ (semelhante a primeira condição, $1$ é um elemento quaisquer, e não necessariamente o número 1)
    % \item \textit{\textbf{Distributividade}}: $\forall a, b, c \in A, a \cdot (b + c) = a \cdot b + a \cdot c \land (b + c) \cdot a = b \cdot a + c \cdot a$
    \item $(A, \cdot)$ é um monoide com elemento neutro $1$.
\end{itemize}
Com esta definição de \textit{anel} chega-se em outras também importantes:
\begin{itemize}
    \item se $\forall a, b \in A, a \cdot b = b \cdot a$ então $A$ é um \textit{anel comutativo}.

    \item se $A$ é um \textit{anel comutativo} em que os elementos neutros deste são diferentes ($0 \neq 1$) e $\forall a, b \in A, a \cdot b = 0 \Rightarrow a = 0 \lor b = 0$ então $A$ é um \textit{domínio}.

    \item se $A$ é um \textit{anel comutativo} em que os elementos neutros deste são diferentes ($0 \neq 1$) e todo elemento diferente de $0$ em $A$ possui inverso na operação $\cdot$, isto é, $(A - \{0\}, \cdot)$ é um grupo então $A$ é um \textit{corpo}, o que em inglês é conhecido como \textit{field}, e em geral se representa como $\mathbb{F}_n$ ao invés de simplesmente $\mathbb{Z}/(n)$.
\end{itemize}

% coisa q to mudando:
Havendo conhecimento da definição de \textit{corpos}, tem-se o seguinte importante Lema apresentado em \cite[p.47]{book:2399854} (cuja a prova depende fortemente do Lema \ref{lema : mdcinv}):
    \begin{lema}
        $\forall n \in \mathbb{Z}, \mathbb{Z}/(n)$ é um corpo se e somente se $n$ é primo. 
    \end{lema}


% Relacionados a \textit{corpos}, tem-se os seguintes lemas importantes apresentados em \cite{book:2399854}:
% \begin{lema}
%     $\forall a, n \in \mathbb{Z}, n > 0 \Rightarrow \exists b \in \mathbb{Z}$ tal que $a \cdot b \equiv 1 \pmod n$ se, e somente se, $\mdc(a, n) = 1$.
% \end{lema}

% \begin{lema}
%     $\forall n \in \mathbb{Z}, \mathbb{Z}/(n)$ é um corpo se e somente se $n$ é primo. 
% \end{lema}

Observando o Lema \ref{lema : mdcinv}, em um \textit{anel de inteiros módulo $n$}, só há inverso multiplicativo para um determinada \textit{classe de equivalência} $\overline{a}$ se $\mdc(a, n) = 1$, pois só assim existirá outra \textit{classe de equivalência} $\overline{b}$ tal que $\overline{a} \cdot \overline{b} = \overline{a \cdot b} = \overline{1}$ (onde $\overline{1}$ é o elemento neutro da operação $\cdot$).

Outro conceito importante originado a partir da definição de \textit{anéis de inteiros módulo $n$} é a definição de \textit{grupo de unidades}, denotado por $(\mathbb{Z}/(n))^{\times}$. Esse é um subconjunto formado pelas \textit{classes de equivalência} invertíveis de $\mathbb{Z}/(n)$, ou seja:
\begin{equation} \label{def:totientset}
    (\mathbb{Z}/(n))^{\times} = \{\overline{a} \in \mathbb{Z}/(n) | \mdc(a, n) = 1\}
\end{equation}

Sobre a implementação de \textit{anéis de inteiros módulo $n$} na biblioteca Mathematical Components, segundo apresentado em \cite{assia_mahboubi_2022_7118596}, essa implementação envolve o tipo \lstinline[language = coq]{ordinal}. Este é, conforme \cite{assia_mahboubi_2022_7118596}, declarado da seguinte forma (junto de sua notação e \textit{coercion} para \lstinline[language=coq]|nat|):
% \cite[p.~145]{assia_mahboubi_2022_7118596}
% Este é idêntico ao \textit{record} \lstinline[language = coq]{smaller} (usado como exemplo na Subseção \ref{subsection:coercion}), porém é declarado da seguinte forma (junto de sua notação e \textit{coercion} relacionada):
    \begin{lstlisting}[language=coq,frame=single,tabsize=1]
Inductive ordinal n := Ordinal m of m < n.
Notation "'I_' n" := (ordinal n).
Coercion nat_of_ord n (i : 'I_n) := let: @Ordinal _ m _ := i in m.
    \end{lstlisting}
% Com isso, para simular os \textit{inteiros módulo $n$} (ainda não tendo provadas as propriedades que fazem destes conjuntos \textit{anéis}), se utiliza o seguinte código:
Com isso, para implementar operações sobre \textit{inteiros módulo $n$}, é, segundo \cite{assia_mahboubi_2022_7118596}, utilizado o seguinte código:
    \begin{lstlisting}[language=coq,frame=single,tabsize=1]
Variable p' : nat.
Local Notation p := p'.+1.
Implicit Types x y z : 'I_p.
Definition inZp i := @Ordinal p (i %% p) (ltn_pmod i (ltn0Sn p')).
    \end{lstlisting}
em que o comando \lstinline[language = coq]{Variable} declara uma variável no contexto de quaisquer declarações a partir daquela linha, o que é equivalente a utilizar nessas \lstinline[language = coq]{\forall (p' : nat)} (e é o que é considerado nas declarações fora da \lstinline[language = coq]{Section} em que foi declarada a variável) \cite{coqteam2022manual}. Sobre o comando \lstinline[language = coq]{Local Notation}, esse cria um notação válida apenas para o módulo em que ela é declarada (assim, importar o módulo não importará a notação), e quanto ao comando \lstinline[language = coq]{Implicit Types}, esse faz com que nas declarações a seguir, se forem utilizadas variáveis com tipos implícitos e de nome \lstinline[language = coq]{x}, \lstinline[language = coq]{y} ou \lstinline[language = coq]{z}, o \textit{Coq} infira como tipo dessas \lstinline[language = coq]{'I_p} \cite{coqteam2022manual}.

Em relação a definição \lstinline[language = coq]{InZp}, esta serve como maneira para converter números naturais em elementos do tipo \lstinline[language = coq]{'I_p}, e portanto, é útil para definir operações como soma e multiplicação sobre \lstinline[language = coq]{'I_p}. Tal definição utiliza $2$ lemas de \cite{mathcomp-div} e \cite{mathcomp-ssrnat}, cujos enunciados são, respectivamente:
    \begin{lstlisting}[language=coq,frame=single,tabsize=1]
Lemma ltn_pmod m d : 0 < d -> m %% d < d.
    \end{lstlisting}
    \begin{lstlisting}[language=coq,frame=single,tabsize=1]
Lemma ltn0Sn n : 0 < n.+1.
    \end{lstlisting}
Portanto note que a expressão \lstinline[language = coq]{ltn_pmod i (ltn0Sn p')} constrói uma prova de que \lstinline[language = coq]{i %% p < p}, assim construindo um objeto do tipo \lstinline[language = coq]{'I_p}. Por isso a definição local da variável \lstinline[language = coq]{p} força que esta seja maior ou igual a 1 (não é possível construir um objeto do tipo \lstinline[language = coq]{'I_0}).

São então definidas, em \cite{mathcomp-zmodp}, as classes de equivalência $\overline{0}$ e $\overline{1}$ e operações para instâncias de \lstinline[language = coq]{'I_p} da seguinte maneira:
%     \begin{lstlisting}[language=coq,frame=single,tabsize=1]
% Definition Zp0 : 'I_p := ord0.
% Definition Zp1 := inZp 1.
% Definition Zp_opp x := inZp (p - x).
% Definition Zp_add x y := inZp (x + y).
% Definition Zp_mul x y := inZp (x * y).
% Definition Zp_inv x := if coprime p x then inZp (egcdn x p).1 else x.
%     \end{lstlisting}
    \begin{lstlisting}[language=coq,frame=single,tabsize=1]
Definition Zp0 : 'I_p := ord0.
    \end{lstlisting}
    \begin{lstlisting}[language=coq,frame=single,tabsize=1]
Definition Zp1 := inZp 1.
    \end{lstlisting}
    \begin{lstlisting}[language=coq,frame=single,tabsize=1]
Definition Zp_opp x := inZp (p - x).
    \end{lstlisting}
    \begin{lstlisting}[language=coq,frame=single,tabsize=1]
Definition Zp_add x y := inZp (x + y).
    \end{lstlisting}
    \begin{lstlisting}[language=coq,frame=single,tabsize=1]
Definition Zp_mul x y := inZp (x * y).
    \end{lstlisting}
    \begin{lstlisting}[language=coq,frame=single,tabsize=1]
Definition Zp_inv x := if coprime p x then inZp (egcdn x p).1 else x.
    \end{lstlisting}
onde \lstinline[language = coq]{ord0} tem a seguinte definição em \cite{mathcomp-zmodp}:
    \begin{lstlisting}[language=coq,frame=single,tabsize=1]
Definition ord0 := Ordinal (ltn0Sn n').
    \end{lstlisting}
ou seja, \lstinline[language = coq]{ord0} é sempre o elemento de \lstinline[language = coq]{'I_p} construído com \lstinline[language = coq]{m} sendo 0 (em que \lstinline[language = coq]{p} é um valor inferido pelo \textit{Coq}).

Após isso são provados, em \cite{mathcomp-zmodp}, os lemas que tornam o conjunto \lstinline[language = coq]{'I_p} dotado das operações definidas com \lstinline[language = coq]{inZp} em um \textit{grupo abeliano} (conforme a definição em \cite[p.~41-46]{book:2399854}):
% aqui acho que é só a página \cite[p.~46] ou \cite[p.~41-42,46]
%     \begin{lstlisting}[language=coq,frame=single,tabsize=1]
% Lemma Zp_add0z : left_id Zp0 Zp_add.
% Lemma Zp_addNz : left_inverse Zp0 Zp_opp Zp_add.
% Lemma Zp_addA : associative Zp_add.
% Lemma Zp_addC : commutative Zp_add.
%     \end{lstlisting}
    \begin{lstlisting}[language=coq,frame=single,tabsize=1]
Lemma Zp_add0z : left_id Zp0 Zp_add.
    \end{lstlisting}
    \begin{lstlisting}[language=coq,frame=single,tabsize=1]
Lemma Zp_addNz : left_inverse Zp0 Zp_opp Zp_add.
    \end{lstlisting}
    \begin{lstlisting}[language=coq,frame=single,tabsize=1]
Lemma Zp_addA : associative Zp_add.
    \end{lstlisting}
e também em \cite{mathcomp-zmodp} são então provados os lemas que farão de \lstinline[language = coq]{'I_p} um \textit{anel comutativo} (de acordo com \cite[p.~46-47]{book:2399854}), considereando as operações binárias \lstinline[language = coq]{Zp_add} e \lstinline[language = coq]{Zp_mul}:
%     \begin{lstlisting}[language=coq,frame=single,tabsize=1]
% Lemma Zp_mulz1 : right_id Zp1 Zp_mul. (* Elemento neutro do produto a direita *)
% Lemma Zp_mulC : commutative Zp_mul. (* Comutatividade do produto *)
% Lemma Zp_mul1z : left_id Zp1 Zp_mul. (* Elemento neutro do produto a esquerda *)
% Lemma Zp_mulA : associative Zp_mul. (* Associatividade do produto *)
% Lemma Zp_mul_addr : right_distributive Zp_mul Zp_add. (* Distributividade a direita *)
% Lemma Zp_mul_addl : left_distributive Zp_mul Zp_add. (* Distributividade a esquerda *)
%     \end{lstlisting}
    \begin{lstlisting}[language=coq,frame=single,tabsize=1]
Lemma Zp_mulz1 : right_id Zp1 Zp_mul.
    \end{lstlisting}
    \begin{lstlisting}[language=coq,frame=single,tabsize=1]
Lemma Zp_mulC : commutative Zp_mul.
    \end{lstlisting}
    \begin{lstlisting}[language=coq,frame=single,tabsize=1]
Lemma Zp_mul1z : left_id Zp1 Zp_mul
    \end{lstlisting}
    \begin{lstlisting}[language=coq,frame=single,tabsize=1]
Lemma Zp_mulA : associative Zp_mul.
    \end{lstlisting}
    \begin{lstlisting}[language=coq,frame=single,tabsize=1]
Lemma Zp_mul_addr : right_distributive Zp_mul Zp_add.
    \end{lstlisting}
    \begin{lstlisting}[language=coq,frame=single,tabsize=1]
Lemma Zp_mul_addl : left_distributive Zp_mul Zp_add.
    \end{lstlisting}

Por fim, dada a condição para que um \textit{anel comutativo} seja um \textit{domínio} e para que seja \textit{corpo}, as seguintes notações em \cite{mathcomp-zmodp} abragem os conjuntos \lstinline[language=coq]|'I_p| que se encaixam nas respectivas categorias (desconsiderando a necessidade dos lemas sobre tais notações):
    \begin{lstlisting}[language=coq,frame=single,tabsize=1]
Definition Zp_trunc p := p.-2.

Notation "''Z_' p" := 'I_(Zp_trunc p).+2
  (at level 8, p at level 2, format "''Z_' p") : type_scope.
  
Notation "''F_' p" := 'Z_(pdiv p)
  (at level 8, p at level 2, format "''F_' p") : type_scope.
    \end{lstlisting}
onde a notação \lstinline[language=coq]|'Z_p| sempre retorna um tipo \lstinline[language=coq]|'I_p| tal que $2 \leq p$, portanto é sempre um \textit{domínio}, e a notação \lstinline[language=coq]|'F_p| retorna um tipo \lstinline[language=coq]|'I_p| tal que $2 \leq p$ e $p$ é primo (pois a função \lstinline[language=coq]|pdiv|, conforme \cite{mathcomp-prime}, retorna o primeiro primo divisor do número passado como argumento), portanto é sempre um \textit{corpo}.

    % Por fim, vale aqui ressaltar que o desenvolvimento deste tipo na biblioteca vai muito além do que foi apresentado aqui, envolvendo detalhes relacionados a interfaces de anéis e outros conceitos tratados na biblioteca.

\section{Função $\varphi$ de Euler}

Uma função muito presente em grande parte dos conteúdos de teoria dos números é a função $\varphi$ de Euler. Essa também é conhecida como função totiente de Euler e conforme \cite[p.~48]{book:2399854}, para quaisquer $n$ inteiro positivo, é definida como:
    \begin{equation} \label{def:phi}
        \varphi(n) = |(\mathbb{Z}/(n))^{\times}|
    \end{equation}
isto é equivalente a:
    \begin{equation}
        \varphi(n) = |\{x \in \mathbb{Z} | 0 \leq x \leq n-1 \land mdc(x,n) = 1 \}|
    \end{equation}
Essa função possui algumas propriedades importantes apresentadas em \cite[p.~48-49]{book:2399854}:
%(estas são apresentadas em \cite{book:2399854} \textcolor{red}{e maior parte das provas não apresentadas aqui se encontram neste livro}):
    \begin{enumerate}
    \item $\varphi(1) = \varphi(2) = 1$
    \item \label{item:prop-phi-2} $\forall n, n > 2 \Rightarrow 1 < \varphi(n) < n$
    \item \label{item:prop-phi-3} $\forall p,$ se $p$ é primo então $\forall k \in \mathbb{N} - \{0\}, \varphi(p^k) = p^k - p^{k-1}$, portanto, $\varphi(p) = p - 1$
    \item \label{item:prop-phi-4} $\forall n, m \in \mathbb{N} - \{0\}, \mdc(n, m) = 1 \Rightarrow \varphi(n \cdot m) = \varphi(n) \cdot \varphi(m) $
    \item \label{item:prop-phi-5} $\forall n \in \mathbb{N} - \{0\}$, se a fatoração de $n$ em potências de primos distintos é dada por $n = p_{1}^{\alpha_{1}} \cdot ... \cdot p_{k}^{\alpha_{k}}$, então:
        \begin{equation} \label{lema:phi-formula}
            \varphi(n) = \prod_{1 \leq i \leq k} \varphi(p_{i}^{\alpha_{i}}) = \prod_{1 \leq i \leq k} p_{i}^{\alpha_{i}} - p_{i}^{\alpha_{i} - 1} = n \cdot \prod_{1 \leq i \leq k} \left( 1 - \frac{1}{p_{i}} \right)
        \end{equation}
    \end{enumerate}

Além dessas propriedade existem dois teoremas apresentados em \cite[p.~49-50]{book:2399854} que devem ser notados, e são eles:

\begin{teorema}[\textit{Teorema de Euler-Fermat}]
\label{eq : euler-fermat}
$\forall a, m \in \mathbb{Z},$ se  $ m > 0$  e $\mdc(a,m) = 1$ então:
    \begin{equation*}
        a^{\varphi(m)} \equiv 1 \pmod{m}
    \end{equation*}
\end{teorema}

\noindent
\textit{Demonstração}: seja $R = \{r_1, r_2, ..., r_{\varphi(m)}\}$ o conjunto de valores no intervalo $[1, m-1]$ em que $\mdc(r_i, m) = 1$ para $i \in [1, \varphi(m)]$ (por isso $|R| = |\varphi(m)|$), observe que o conjunto $A = \{a \cdot r_1, a \cdot r_2, ..., a \cdot r_{\varphi(m)}\}$ é composto apenas de valores tais que para $i \in [1, \varphi(m)]$, $\mdc(a \cdot r_i, m) = 1$. Além disso, note que cada elemento do conjunto $A$, assim como cada um do conjunto $R$, é único, pois se $a \cdot r_i \equiv a \cdot r_j \pmod{m}$, então pelo Item \ref{item:propcong7-cancelamento}, $r_i \equiv r_j \pmod{m}$, logo como $r_i, r_j \in [1, m-1]$, $r_i = r_j$ (ou seja, $i = j$). Como $A$ possui a mesma quantidade de elementos que $R$, sendo todos distintos módulo $m$, então para cada elemento $a \cdot r_i \in A$ existe um elemento em $r_j \in R$ tal que $a \cdot r_i \equiv r_j \pmod{m}$. Essa última afirmação pode ser provada por absurdo, pois supondo que não exista tal $r_j$, então $a \cdot r_i \equiv r \pmod{m}$, tal que $r \not\in R \land r \in [1, m-1] \land \mdc(r, m) = 1$ (pelo Lema \ref{lema:euclides}), mas se esse for o caso então $R$ não é o conjunto de valores no intervalo $[1, m-1]$ em que $\mdc(r_i, m) = 1$ para $i \in [1, \varphi(m)]$, como definido inicialmente, logo, tem-se um absurdo.
Por meio dos pares congruentes módulo $m$, de forma $a \cdot r_i \equiv r_j \pmod{m}$, tem-se pelo Item \ref{item:propcong6-produto}:
\begin{align*}
    \prod_{i = 1}^{\varphi(m)} a \cdot r_i \equiv \prod_{i = 1}^{\varphi(m)} r_i \pmod{m}
\end{align*}
manipulando a equação:
\begin{align*}
    \prod_{i = 1}^{\varphi(m)} a \cdot r_i \equiv \prod_{i = 1}^{\varphi(m)} r_i \pmod{m}
    \\
    \Longleftrightarrow a^{\varphi(m)} \cdot \prod_{i = 1}^{\varphi(m)} r_i \equiv \prod_{i = 1}^{\varphi(m)} r_i \pmod{m}
\end{align*}
e como:
\begin{equation*}
    \mdc\left(\prod_{i = 1}^{\varphi(m)} r_i, m\right) = 1
\end{equation*}
pelo Item \ref{item:propcong7-cancelamento} tem-se:
\begin{align*}
    a^{\varphi(m)} \cdot \prod_{i = 1}^{\varphi(m)} r_i & \equiv \prod_{i = 1}^{\varphi(m)} r_i \pmod{m}
    \\
    \Longleftrightarrow a^{\varphi(m)} & \equiv 1 \pmod{m} 
\end{align*} \qed

\begin{teorema}[\textit{Pequeno Teorema de Fermat}]
\label{eq : pequeno-fermat}
$\forall a \in \mathbb{N} - \{0\}$, dado um número primo $p$, tem-se que:
    \begin{equation*}
        a^p \equiv a \pmod{p}
    \end{equation*}
\end{teorema}

\noindent
\textit{Demonstração}: no caso em que $p \mid a$ a congraciência é válida (pois ambos os lados são congruentes a $0$), enquanto no caso em que $p \nmid a$, pelo Teorema \ref{eq : euler-fermat}, tem-se que:
\begin{equation*}
    a^{\varphi(p)} \equiv 1 \pmod{p}
\end{equation*}
e pelo Item \ref{item:prop-phi-3}:
\begin{align*}
    a^{\varphi(p)} \equiv 1 \pmod{p}
    \\
    \Longleftrightarrow a^{p-1} \equiv 1 \pmod{p}
\end{align*}
assim utilizando-se da propriedade descrita no Item \ref{item:propcong6-produto} tem-se:
\begin{align*}
    a^{p-1} \equiv 1 \pmod{p}
    \\
    \Longleftrightarrow a^{p} \equiv a \pmod{p}
\end{align*} \qed

Na biblioteca Mathematical Components, em \cite{mathcomp-prime}, a função $\varphi$ de Euler é implementada da seguinte maneira:
\begin{lstlisting}[language=coq,frame=single,tabsize=1]
Definition totient n := 
    foldr add_totient_factor (n > 0) (prime_decomp n).
\end{lstlisting}
Em que por meio de uma \textit{coercion} de \lstinline[language = coq]{bool} para \lstinline[language = coq]{nat} o valor retornado por \lstinline[language = coq]{n > 0} é convertido para $0$ (se for \lstinline[language = coq]{false}) ou $1$ (se for \lstinline[language = coq]{true}) e a função \lstinline[language = coq]{add_totient_factor} é definida em \cite{mathcomp-prime} como:
\begin{lstlisting}[language=coq,frame=single,tabsize=1]
Definition add_totient_factor f m := 
    let: (p, e) := f in p.-1 * p ^ e.-1 * m.
\end{lstlisting}
e \lstinline[language = coq]{prime_decomp} é uma função que, segundo \cite{mathcomp-prime}, recebe um número $n$ qualquer e retorna uma lista de tuplas $k$ da forma $(p_i, e_i)$ em que:
    \begin{equation*}
        n = p_1^{e_1} \cdot p_2^{e_2} \cdot ... \cdot p_k^{e_k}
    \end{equation*}
ou seja, retorna a fatoração de $n$ em primos, o que é garantido pelo seguinte lema \lstinline[language = coq]{prime_decomp_correct} disponível na biblioteca (cuja proposição não será apresentada aqui devido a sua extensão).

Além disso, tem-se na biblioteca, em \cite{mathcomp-prime}, lemas sobre algumas das propriedades aqui expostas. Iniciando pela propriedade \ref{item:prop-phi-3}, tem-se:
\begin{lstlisting}[language=coq,frame=single,tabsize=1]
Lemma totient_pfactor p e :
  prime p -> e > 0 -> totient (p ^ e) = p.-1 * p ^ e.-1.
\end{lstlisting}
Também em \cite{mathcomp-prime} há um lema equivalente a propriedade \ref{item:prop-phi-4} (em que a condição de maior divisor comum igual a $1$ é dada por \lstinline[language = coq]{coprime}, que é por sua vez uma função booleana que recebe dois números e retorna \lstinline[language = coq]{true} se o $\mdc$ desses for $1$ e \lstinline[language = coq]{false} caso contrário):
\begin{lstlisting}[language=coq,frame=single,tabsize=1]
Lemma totient_coprime m n :
    coprime m n -> totient (m * n) = totient m * totient n.
\end{lstlisting}
Por último, em \cite{mathcomp-prime}, há também um lema que estabelece a equivalência entre a definição da função $\varphi$ de Euler exposta em \cite[p.~48]{book:2399854} (Definição \ref{def:phi}) e a definição da biblioteca baseada na Equação \ref{lema:phi-formula}\footnote{Vale aqui observar que novamente há uma \textit{coercion} de \lstinline[language = coq]{bool} para \lstinline[language = coq]{nat} sobre o retorna da função \lstinline[language = coq]{coprime} (a \textit{coercion} converte \lstinline[language = coq]{false} para $0$ e \lstinline[language = coq]{true} para $1$), dado que o somatório requer valores numéricos.}%colocar aqui q retorna 0 se false e 1 se true?
\begin{lstlisting}[language=coq,frame=single,tabsize=1]
Lemma totient_count_coprime n : 
    totient n = \sum_(0 <= d < n) coprime n d.
\end{lstlisting} 

\section{Congruência de Grau 2 e Símbolo de Legendre}
Sendo um técnica muito eficiente para verificar se um número é um resíduo quadrático em relação a um outro número primo, os símbolos de Legendre, além de serem um objetivo de implementação deste trabalho, são diretamente utilizados no algoritmo RESSOL. Entretanto para se explicar o que são esses, é necessário uma breve introdução sobre congruências de grau 2 (ou quadráticas).
Como motivação para se tratar deste assunto, note que, sendo $p > 2$ um número primo e $a, b, c \in \mathbb{Z}$, em que $a$ não é divisível por $p$, suponha que se deseje resolver a seguinte equação:
\begin{equation} \label{eq : ax2bxc}
    a \cdot x^2 + b \cdot x + c \equiv 0 \pmod p
\end{equation}
Manipulando essa equação com objetivo de obter um resultado semelhante ao da fórmula de Bhãskara, tem-se (multiplicando ambos os lados por $4$):
\begin{equation*}
    4 \cdot a^2 \cdot x^2 + 4 \cdot a \cdot b \cdot x + 4 \cdot a \cdot c \equiv 0 \pmod p
\end{equation*}
e como 
\begin{equation*}
    b^2 - 4 \cdot a \cdot c \equiv b^2 - 4 \cdot a \cdot c \pmod p
\end{equation*}
pode-se adicionar esses valor em ambos os lados:
\begin{equation*}
    4 \cdot a^2 \cdot x^2 + 4 \cdot a \cdot b \cdot x + b^2 \equiv b^2 - 4 \cdot a \cdot c \pmod p
\end{equation*}
assim, finalmente se chega ao resultado desejado:
\begin{equation} \label{eq : bhaskara}
    (2 \cdot a \cdot x + b)^2 \equiv b^2 - 4 \cdot a \cdot c \pmod p
\end{equation}
Pode se verificar que resolver a Equação \ref{eq : ax2bxc}  é equivalente a resolver \ref{eq : bhaskara}. Rescrevendo com $X = 2 \cdot a \cdot x + b$ e $d = b^2 - 4 \cdot a \cdot c$, obtêm-se:
\begin{equation} \label{eq : quadcong}
    X^2 \equiv d \pmod p
\end{equation}
Com isso, note que um problema mais complexo (Equação \ref{eq : ax2bxc}) foi transformado em um problema mais simples (Equação \ref{eq : quadcong}). Sobre esse último, se possui solução, isto é, $d$ é um quadrado perfeito em $\mathbb{Z}/(p)$, então, se diz que $d$ é um \textit{resíduo quadrático} módulo $p$. Além disso, conforme \cite[p.~86]{book:2399854}, existem precisamente $\frac{p+1}{2}$ resíduos quadráticos módulo $p$ (valores de $d$ menores que $p$ para os quais \ref{eq : quadcong} tem solução), que são neste caso:
\begin{equation} \label{eq : listquadres}
    0^2 \bmod{p}, 1^2 \bmod{p}, 2^2 \bmod{p}, 3^2 \bmod{p}, ..., \left(\frac{p -1}{2} \right)^2 \bmod{p} 
\end{equation}
O motivo desse fato é que para todo $x \in \mathbb{Z}$ existe algum $i$ no intervalo $\left[0, \frac{p-1}{2}\right]$ tal que $x \equiv i \pmod p $ ou $ x \equiv -i \pmod p$.
Tal afirmação pode ser inferida facilmente, visto que tem-se todos os restos até $\frac{p-1}{2}$ 
com $i$, e para qualquer resto $r > \frac{p-1}{2}$ basta escolher $i = p - r$ (o que está obviamente dentro do intervalo de $i$), pois:
\begin{align*}
    (p - r) \equiv i \pmod p
    &\begin{aligned}
        \;\; \Longrightarrow \; -(p - r) \equiv -i \pmod p
    \end{aligned} \\
    &\begin{aligned}
        \;\; \Longrightarrow \; r - p \equiv -i \pmod p
    \end{aligned} \\
    &\begin{aligned}
        \;\; \Longrightarrow \; r \equiv -i \pmod p
    \end{aligned}
\end{align*}
Logo, $x^2$ é congruente à um dos números da lista \ref{eq : listquadres}, pois dado $y = \pm i$, ou seja, $y \in \left[- \frac{p-1}{2}, \frac{p-1}{2}\right]$:
\begin{align*}
    x \equiv y \pmod p
    &\begin{aligned}
        \;\; \Longrightarrow \; x^2 \equiv y^2 \pmod p
    \end{aligned} \\
    &\begin{aligned}
        \;\; \Longrightarrow \; x^2 \equiv i^2 \pmod p
    \end{aligned}
\end{align*}
e $i^2$ está na Lista \ref{eq : listquadres}.

Outro fato interessante em relação a lista \ref{eq : listquadres} é que todos este números são distintos em módulo $p$, haja vista, para quaisquer $i, j \in \left[0, \frac{p-1}{2}\right]$:
\begin{align}
    i^2 \equiv j^2 \pmod p
    &\begin{aligned}
        \;\; \Longleftrightarrow p \mid (i^2 - j^2)
    \end{aligned} \\
    &\begin{aligned}
        \;\; \Longleftrightarrow p \mid (i - j)\cdot(i + j)
    \end{aligned} \\
    &\begin{aligned} \label{eq : ordivp}
        \;\; \Longleftrightarrow p \mid (i - j) \lor p \mid (i + j)
    \end{aligned} 
    % \\
    % &\begin{aligned}
    %     \;\; \Longleftrightarrow \; i \equiv j \pmod p \;\; \lor \;\; i \equiv -j \pmod p
    % \end{aligned} \\
\end{align}
Dado que $i, j \in \left[0, \frac{p-1}{2}\right]$ então $0 \leq i + j \leq p - 1$, logo, existem as seguintes possibilidades:
\begin{enumerate}
    \item $i = j = 0$ e portanto $i \equiv j \pmod p$.
    \item $0 < i + j \leq p-1$ (visto que $0 < i, j \leq \frac{p-1}{2}$) e portanto $p$ não divide $i + j$ (pois essa soma resulta em um valor menor que $p$ e maior que $0$), e então pela disjunção em \ref{eq : ordivp} resta apenas a possibilidade de $p \mid (i - j)$, o que equivale a $i \equiv j \pmod p$, ou seja, $i$ é igual $j$ módulo $p$ se e somente se seus quadrados também são.
\end{enumerate}
\noindent
A partir destas conclusões expostas aqui é importante estabelecer o seguinte lema a ser utilizado futuramente:
\begin{lema} \label{lema:existnonquadratic}
    Seja $p > 2$ um número primo, existem exatamente $\frac{p+1}{2}$ resíduos quadráticos módulo $p$ e $\frac{p-1}{2}$ resíduos não quadráticos módulo $p$.
\end{lema}
\noindent
\textit{Demonstração}: note que a seguinte lista contém todos os resíduos quadráticos módulo $p$
\begin{equation*}
    0^2 \bmod{p}, 1^2 \bmod{p}, 2^2 \bmod{p}, 3^2 \bmod{p}, ..., (p - 1)^2 \bmod{p}
\end{equation*}
No entanto, essa lista contém valores repetidos, pois
\begin{align}
    (p-x)^2 \equiv (p-x)^2 \pmod{p} & \Longleftrightarrow
    (p-x)^2 \equiv p^2 - 2 \cdot p \cdot x + x^2 \pmod{p} 
    \\
    & \Longleftrightarrow (p-x)^2 \equiv x^2 \pmod{p} \label{eq:paresquad}
\end{align}
Assim, retirando os valores repetidos da lista (isto é, remover valores de modo que não hajam pares como em \ref{eq:paresquad}), tem-se:
\begin{equation*}
    0^2 \bmod{p}, 1^2 \bmod{p}, 2^2 \bmod{p}, 3^2 \bmod{p}, ..., \left(\frac{p -1}{2} \right)^2 \bmod{p} 
\end{equation*}
Cada um desses valores é um número em $[1, p-1]$ (são restos), porém existem apenas $\frac{p+1}{2}$ desses valores, logo existem números no mesmo intervalo que não são resíduos quadráticos, e quantidade desses é $p - \frac{p+1}{2} = \frac{2 \cdot p - p - 1}{2} = \frac{p - 1}{2}$. \qed

% \begin{lema}
%     Seja $p > 2$ um número primo e $a \in \mathbb{Z}$, a equação $x^2 \equiv a \pmod{p}$ possui exatamente $\frac{p-1}{2}$ soluções no intervalo $[1, p-1]$ e os números 
% \end{lema}

Apresentados estes conceitos sobre congruências quadráticas, dado um número primo $p > 2$ e $a \in \mathbb{Z}$, se define o \textit{símbolo de Legendre} por:
\begin{equation*}
    \left( \frac{a}{p} \right) = \begin{cases}
        1 \text{, se $p \hspace{-4pt}\not|\hspace{2pt} a$ e $a$ é um resíduo quadrático módulo $p$}
        \\
        0 \text{, se $p \mid a$}
        \\
        -1 \text{, caso contrário ($a$ não é um resíduo quadrático)}
        \end{cases}
\end{equation*}
Essa definição, por si só, não traz qualquer utilidade, no entanto há o então chamado \textit{Critério de Euler}, que apresenta uma maneira eficiente para computar o valor de um símbolo de Legendre. Esse critério afirma o seguinte:
\begin{teorema}[\textit{Critério de Euler}] $\forall a \in \mathbb{Z}$, seja $p > 2$ um número primo, então: \label{teorema:criteriodeeuler}
    \begin{equation*}
        \left( \frac{a}{p} \right) \equiv a^{\frac{p-1}{2}} \pmod p
    \end{equation*}
\end{teorema}
\noindent
Logo, para se computar um símbolo de Legendre basta verificar se o resto da divisão inteira de $a^{\frac{p-1}{2}}$ por $p$ é igual a $1$, $0$ ou $-1 \bmod p$.

Para se realizar a demonstração do \textit{Critério de Euler}, antes é necessário apresentar o conceito de inverso multiplicativo módulo $n$ e alguns lemas e teoremas envolvidos:

\begin{definição}[\textit{Inverso multiplicativo módulo $n$}]
Dados $a, m, n \in \mathbb{Z}$, se $a \cdot m \equiv 1 \pmod n$, se diz que $m$ é um \textit{inverso de $a$ módulo $n$}, e pode ser denotado por $a^{-1}$.
\end{definição}

\begin{lema} \label{lema : mdcinv}
    Para todo $a, n \in \mathbb{Z}$, se $n > 0$, então, existe $b \in \mathbb{Z}$ tal que $a \cdot b \equiv 1 \pmod{n}$ se, e somente se, $\mdc(a, n) = 1$.
\end{lema}
\noindent
\textit{Demonstração}: note que
    \begin{align*}
        a \cdot b \equiv 1 \pmod{n} & \Longleftrightarrow n \mid a \cdot b - 1
        \\
        & 
        \Longleftrightarrow \exists q \in \mathbb{Z}, n \cdot q = a \cdot b - 1
        \\
        & 
        \Longleftrightarrow \exists q \in \mathbb{Z},  a \cdot b - n \cdot 1  = 1
    \end{align*}
Por consequência do Teorema \ref{teorema : bachetbezout}, só existe tal $q$ se $\mdc(a, n) = 1$, e portanto $b$ existe se e somente se isto ocorre.

\begin{lema}[\textit{Unicidade de inverso multiplicativo módulo p}] Dado um número primo $p$, seja 
$a \in [1, p-1]$, existe $k \in [1, p-1]$ tal que $ a \cdot k \equiv 1 \pmod p$ e $k$ é portanto o único inverso multiplicativo de  módulo $p$ de $a$ no intervalo $[1, p-1]$. \label{lema : invmod}
\end{lema}
\noindent
\textit{Demonstração}: primeiramente, sabe-se que $k$ existe pelo Lema \ref{lema : mdcinv} (pois $p$ é primo, logo $\mdc(a, p) = 1$). Assim, dado que $a \cdot k \equiv 1 \pmod p$, suponha que existe $k' \in [1, p-1]$ tal que $a \cdot k' \equiv 1 \pmod p$, então:
\begin{align*}
    a \cdot k \equiv a \cdot k' \pmod p \;
    &\begin{aligned}
        \Longleftrightarrow p \mid a \cdot k - a \cdot k'
    \end{aligned} \\
    &\begin{aligned}
        \Longleftrightarrow p \mid a \cdot (k - k')
    \end{aligned} \\
    &\begin{aligned}
        \Longleftrightarrow p \mid a \cdot (k - k')
    \end{aligned}
    \\
    &\begin{aligned}
        \Longleftrightarrow p \mid a \; \lor \; p \mid (k - k')
    \end{aligned}
\end{align*}
Como $a \in [1, p-1]$, para que a disjunção seja válida deve ser o caso que $p | (k - k')$, e como $|k - k'| < p-1$, a única maneira disto ocorrer é se $k - k' = 0$, ou seja, $k = k'$. \qed
\begin{lema} Seja $a \in [1, p-1]$ em que $p$ é um número primo maior que $2$, se $x^2 \equiv a \pmod p$ não tem solução, então para todo $h \in [1, p-1]$ existe $k \in [1, p-1]$, tal que: \label{lema : hkequivamodp}
    \begin{equation*} 
        h \neq k \land h \cdot k \equiv a \pmod p 
    \end{equation*}
\end{lema}
\noindent \textit{Demonstração}: pelo Lema \ref{lema : invmod}, sabe-se que existe $h^{-1} \in [1, p-1]$ tal que:
\begin{equation*}
    h^{-1} \cdot h \equiv 1 \pmod p
\end{equation*}
e têm-se:
\begin{equation*}
    a \equiv a \pmod p \Rightarrow h^{-1} \cdot a \equiv h^{-1} \cdot a \pmod p
\end{equation*}
Neste momento é importante notar que independentemente de ser o caso de $h^{-1} \cdot a > p - 1$ ou não, existe algum $r \in [1, p-1]$ tal que:
\begin{equation*}
    r \equiv a \cdot h^{-1} \pmod p
\end{equation*}
Seguindo então, pode-se obter o seguinte:
\begin{equation*}
    a \cdot (h^{-1} \cdot h) \equiv a \pmod p
\end{equation*}
pois $h^{-1} \cdot h \equiv 1 \pmod p$. Manipulando essa equação, se chega em:
\begin{equation*}
    (a \cdot h^{-1}) \cdot h \equiv a \pmod p \Longleftrightarrow r \cdot h \equiv a \pmod p
\end{equation*}
Como $r \in [1, p-1]$, resta apenas provar que $r \neq h$, o que é válido pela hipótese de que $x^2 \equiv a \pmod p$ não tem solução (se $r = h$ haveria solução e portanto se teria uma contradição). \qed

\begin{lema} \label{lema : kk'modp}
    Seja $a, h, k, k' \in [1, p-1]$, se $k \cdot h \equiv a \pmod{p}$ e $k' \cdot h \equiv a \pmod{p}$ então $k = k'$ ($k$ é único).
\end{lema}
\noindent
\textit{Demonstração}: observe que, seguindo da hipótese:
\begin{align*}
    k \cdot h \equiv k' \cdot h \pmod{p} 
    &
        \Longleftrightarrow p \mid k \cdot h - k' \cdot h
    \\
    &
        \Longleftrightarrow p \mid h \cdot (k - k')
    \\
    &
        \Longleftrightarrow p \mid h \lor p \mid k - k'
\end{align*}
Como $p \nmid h$ (pois $|h| < p$) só pode ser o caso de que $p \mid k - k'$, porém $|k - k'| < p$, o que implica que $k - k' = 0$ e portanto $k = k'$. \qed

\begin{lema} \label{lema : modp-1fat}
    Seja $p > 2$ um número primo, para todo $a \in \mathbb{Z}$, se $\mdc(a, p) = 1$ e $x^2 \equiv a \pmod p$ não tem solução então:
    \begin{equation*}
        (p - 1)! \equiv a^{\frac{p-1}{2}} \pmod{p}
    \end{equation*}
    
\end{lema}
\noindent
\textit{Demonstração}: pelos lemas \ref{lema : invmod}, \ref{lema : hkequivamodp} e \ref{lema : kk'modp}, pode-se escolher $\frac{p-1}{2}$ pares, utilizando todos os números no intervalo $[1, p-1]$, sem que qualquer número esteja em mais de um par (ou seja, se repita) e de modo que para cada par $(x_i, y_i)$, $x_i \cdot y_i \equiv a \pmod p$, logo:
    \begin{equation*}
        (x_1 \cdot y_1) \cdot (x_2 \cdot y_2) \cdot ... \cdot \left(x_{\frac{p-1}{2}} \cdot y_{\frac{p-1}{2}}\right) \equiv a^{\frac{p-1}{2}} \pmod{p}
    \end{equation*}
Note que o lado esquerdo da equação é uma multiplicação entre todos os valores no intervalo $[1, p-1]$ (sem repetição), o que é igual a $(p - 1)!$, portanto:
    \begin{equation*}
        (p - 1)! \equiv a^{\frac{p-1}{2}} \pmod{p}
    \end{equation*}
    \qed

Como exemplo, escolhendo $n = 3$, esse não é resíduo não quadrático em módulo $7$ (verificação deixada a cargo do leitor) e $7 \nmid 3$, assim, dado que $3^{\frac{7-1}{2}} = 27$ e $(7 - 1)! = 720$, como esperado, $27 \bmod{p} = 720 \bmod{p} = 6$. O mesmo não ocorre tendo $n = 4$, pois apesar de $7 \nmid 4$ para $x = 2$ vale $x ^ 2 \equiv 4 \pmod{7}$, e realizando os cálculos $(4-1)! \bmod{7} = 6 \bmod{7} = 6$ e $4^{\frac{7 - 1}{2}} \bmod{7} = 64 \bmod{7} = 1$.

\begin{lema} Seja $p$ um número primo, então para quaisquer soluções de $x^2 \equiv 1 \pmod{p}$ têm-se que $x \equiv 1 \pmod{p}$ ou $x \equiv -1 \pmod{p}$. Portanto para qualquer outro valor $y$ que não é uma solução, $y \not\equiv y^{-1} \pmod{p}$.
\label{lema : eq1modp}
\end{lema}
\noindent
\textit{Demonstração}: se $x$ é uma solução então
    \begin{align*}
        x^2 \equiv 1 \pmod{p} &
        \Longleftrightarrow p \mid (x^2 - 1)
        \\
        &
        \Longleftrightarrow p \mid (x - 1) \cdot (x + 1)
        \\
        &
        \Longleftrightarrow p \mid (x - 1) \lor p \mid (x + 1)
        \\
        &
        \Longleftrightarrow x \equiv 1 \pmod{p} \lor x \equiv -1 \pmod{p}
    \end{align*}
Portanto, para qualquer valor $y$ tal que $y^2 \not\equiv 1 \pmod{p}$ tem-se que $y \not\equiv y^{-1} \pmod{p}$ (caso contrário $y$ seria uma solução). \qed

\begin{teorema}[\textit{Teorema de Wilson}] \label{teorema : wilson}
    Seja número composto um número que pode ser escrito como a multiplicação de dois outros números menores então, dado $n > 1$:
    \begin{equation*}
        (n - 1)! \equiv \begin{cases}
                        -1 \pmod{n} \; \textit{se $n$ é primo} \\
                        0 \pmod{n} \; \textit{se $n$ é composto e $n \neq 4$}
                        \end{cases}
    \end{equation*}
\end{teorema}
\noindent
\textit{Demonstração}: têm-se os seguintes casos:
\begin{enumerate}
    \item Se $n$ é composto mas não é quadrado de um número primo, pode-se escrever $n = a \cdot b$ em que $1 < a < b < n$, então $a$ e $b$ são fatores de $(n-1)!$, portanto $n \mid (n-1)!$, ou seja, $(n-1)! \equiv 0 \pmod{n}$.

    \item Se $n = p^2$ onde $p$ é um número primo maior que $2$ então $p$ e $2 \cdot p$ são fatores de $(n-1)!$, portanto, novamente $n \mid (n-1)!$, ou seja, $(n-1)! \equiv 0 \pmod{n}$.

    \item \label{item:caso3wilson} Se $n$ é primo, como $n - 1 \equiv -1 \pmod{n}$ partindo de $(n-2)! \equiv (n-2)! \pmod{p}$ se obtém que $(n-1)! \equiv -(n-2)! \pmod{n}$, e agora, observe que do lado direito da equação, pelos lemas \ref{lema : mdcinv}, \ref{lema : hkequivamodp} e \ref{lema : eq1modp}, pode-se manipular a expressão de modo a organizá-la em $\frac{n-3}{2}$ pares $(x \cdot y)$ onde $x, y \in [2, n-2]$ e                      $x \cdot y \equiv 1 \pmod{n}$, portanto $(n-2)! \equiv 1 \pmod{p}$ e então $-(n-2)! \equiv -1 \pmod{p}$, logo $(n-1)! \equiv -1 \pmod{n}$.
\qed
\end{enumerate}

A exemplo, para $n = 19$ (que é um número primo) e portanto $(n - 1)! = 6402373705728000$, o resultado de $6402373705728000 \bmod{19}$ é $18$. Escolhendo um número composto $n = 20$, então $(n - 1)! = 121645100408832000$ e $121645100408832000 \bmod{20}$ é $19$.

Esse teorema possui implementação na biblioteca Mathematical Components, e esta será implementação será apresentada em \ref{sec:implementacoes}. 
Agora será então apresentada a demonstração do \textit{Critério de Euler}.
% Demonstração do Critério de Euler
\\
\noindent
\textit{Demonstração}: tem-se os seguintes casos
\begin{enumerate}
    \item se $a \equiv 0 \pmod p$, ou seja, $p \mid a$, pelas propriedades de módulo pode-se elevar ambos os lados por $\frac{p-1}{2}$, e
    então se chega em $a^{\frac{p-1}{2}} \equiv 0 \pmod p$

    \item se $p \nmid a$, então pelo Teorema \ref{eq : euler-fermat} tem-se 
\begin{align*}
        a^{\varphi(p)} \equiv 1  \pmod p \; \Longleftrightarrow
        &\begin{aligned}
            \;\; a^{p - 1} \equiv 1  \pmod p 
        \end{aligned} \\
        &\begin{aligned}
            \;\; a^{\frac{p-1}{2}} \cdot a^{\frac{p-1}{2}} \equiv 1  \pmod p 
        \end{aligned}
\end{align*}
subtraindo 1 de ambos os lados:
\begin{align*}
        a^{\varphi(p)} \equiv 1  \pmod p \; \Longleftrightarrow
        &\begin{aligned}
            \;\; a^{\frac{p-1}{2}} \cdot a^{\frac{p-1}{2}} -1 \equiv 0  \pmod p 
        \end{aligned} \\
        \Longleftrightarrow
        &\begin{aligned}
            \;\; (a^{\frac{p-1}{2}} + 1) \cdot (a^{\frac{p-1}{2}} - 1) \equiv 0  \pmod p 
        \end{aligned} \\
        \Longleftrightarrow
        &\begin{aligned}
            \;\; p \mid (a^{\frac{p-1}{2}} + 1) (a^{\frac{p-1}{2}} - 1)
        \end{aligned} \\
        \Longleftrightarrow
        &\begin{aligned}
            \;\; p \mid (a^{\frac{p-1}{2}} + 1) \;\; \lor \;\; p \mid (a^{\frac{p-1}{2}} - 1)
        \end{aligned} \\
        \Longleftrightarrow
        &\begin{aligned}
            \;\; a^{\frac{p-1}{2}} \equiv -1 \pmod p \;\; \lor \;\; a^{\frac{p-1}{2}} \equiv 1 \pmod p
        \end{aligned}
\end{align*}
Agora deve-se mostrar que o lado direito da disjunção é válido se e somente (bi-implicação) $a$ é um resíduo quadrático módulo $p$. Para isso, provando a volta da bi-implicação, suponha que $a$ é um resíduo quadrático, e portanto existe algum $i$ tal que
$a \equiv i^2 \pmod p$. Podemos elevar ambos os lados por $\frac{p-1}{2}$, donde se obtêm:
\begin{equation*}
    a^{\frac{p-1}{2}} \equiv i^{p-1} \pmod p
\end{equation*}
pelo Teorema \ref{eq : euler-fermat} (e pela transitividade da relação de módulo), ocorre o seguinte:
\begin{align*}
    i^{p-1} \equiv 1 \pmod p \;\; \Longrightarrow
    &\begin{aligned}
        \;\; a^{\frac{p-1}{2}} \equiv 1 \pmod p
    \end{aligned}
\end{align*}
Assim está provada a volta. Agora, para provar a ida:
\begin{equation*}
    a^{\frac{p-1}{2}} \equiv 1 \pmod p \Longrightarrow \exists i, a \equiv i^2 \pmod p 
\end{equation*}
por contraposição, é equivalente provar que:
\begin{equation} \label{eq : contrapos}
    \forall i, a \not\equiv i^2  \pmod p \Longrightarrow a^{\frac{p-1}{2}} \not\equiv 1 \pmod p
\end{equation}
Pela hipótese e pelo Lema \ref{lema : modp-1fat} tem-se que $a^{\frac{p-1}{2}} \equiv (p-1)! \pmod{p}$. Usando o Teorema \ref{teorema : wilson}, por transitividade, $a^{\frac{p-1}{2}} \equiv -1 \pmod{p}$, então de fato $a^{\frac{p-1}{2}} \not\equiv 1 \pmod{p}$. \qed

\end{enumerate}

Quanto aos conceitos apresentados nessa sessão, a grande parte (se não todos) não estão implementados na biblioteca Mathematical Components. Sendo assim, apresentam um desafio considerável para que se complete o objetivo proposto neste trabalho, a ser apresentado com maiores detalhes no Capítulo \ref{cap:implementacao}.