\chapter{Introdução} 

\label{chap:intro}

Durante os cursos de Ciência da Computação, são vistas estruturas matemáticas muito diferentes daquelas as quais alunos de ensino médio estão habituados. No geral, grande parte destas estruturas são abstratas por não parecerem uma representação de um objeto real ou por, apesar de parecer, a razão de sua formulação não ser bem motivada de início. A exemplo de tais estruturas temos vetores, matrizes, filas e grafos, utilizados na modelagem de diversos problemas. Apesar destas ferramentas serem extremamente úteis, há um tipo de objeto matemático sempre presente na maioria dos problemas e que muitas vezes são considerados limitados e apenas objetos auxiliares demasiadamente utilizados: estes são os números inteiros. O conjunto dos números inteiros, apesar de ser formado por objetos (números) vistos como simples, possui diversas endorrelações que levam a muitas conclusões e invenções de grande importância, principalmente para o campo da criptografia. Dentre estas relações, duas delas são pilares fundamentais para tais conclusões e invenções mencionadas: a relação de divisibilidade e de congruência. A primeira é definida da seguinte forma 
\cite{book:2399854}:

