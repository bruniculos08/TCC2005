% ---- Arquivo com as configurações do PDF

% Alteração para fonte de capítulos:
\renewcommand{\ABNTEXchapterfont}{\normalfont}
\renewcommand{\ABNTEXchapterfont}{\bfseries}

% Definindo a cor azul em RGB:
\definecolor{blue}{RGB}{41,5,195}

% Informações do PDF:
\makeatletter
\hypersetup{
		pdftitle={\@title}, 
		pdfauthor={\@author},
    	pdfsubject={\imprimirpreambulo},
	    pdfcreator={Bruno Rafael dos Santos},
		pdfkeywords={Algoritmo \textit{RESSOL}}{Algoritmo de Tonelli-Shanks}{Lei de Reciprocidade Quadrática}{abntex2}{trabalho acadêmico}, 
		colorlinks=true,
    	linkcolor=blue,
    	citecolor=blue,
    	filecolor=magenta,
		urlcolor=blue,
		bookmarksdepth=4
}
\makeatother

% Posiciona figuras e tabelas no topo da página quando adicionadas sozinhas em uma página em branco (ver https://github.com/abntex/abntex2/issues/170):
\makeatletter
\setlength{\@fptop}{5pt} 
\makeatother

% Possibilita criação de Quadros e Lista de quadros (ver https://github.com/abntex/abntex2/issues/176):
\newcommand{\quadroname}{Quadro}
\newcommand{\listofquadrosname}{Lista de quadros}
\newfloat[chapter]{quadro}{loq}{\quadroname}
\newlistof{listofquadros}{loq}{\listofquadrosname}
\newlistentry{quadro}{loq}{0}

% Configurações para atender às regras da ABNT
\setfloatadjustment{quadro}{\centering}
\counterwithout{quadro}{chapter}
\renewcommand{\cftquadroname}{\quadroname\space} 
\renewcommand*{\cftquadroaftersnum}{\hfill--\hfill}

\setfloatlocations{quadro}{hbtp} % Ver https://github.com/abntex/abntex2/issues/176

% O tamanho do parágrafo é dado por:
\setlength{\parindent}{1.3cm}

% Controle do espaçamento entre um parágrafo e outro:
\setlength{\parskip}{0.2cm}  % tente também \onelineskip

% Configurações adicionadas por Bruno:
	% Teoremas e etc:
	\newtheorem{definição}{Definição}
	\newtheorem{teorema}{Teorema}
	\newtheorem{lema}{Lema}
	% Operadores:
	\usepackage{amsmath, amsfonts, amssymb}
	\usepackage{mathtools}
	\DeclareMathOperator{\mdc}{mdc}
	\DeclareMathOperator{\mmc}{mmc}
	\DeclarePairedDelimiter\abs{\lvert}{\rvert}
	% \DeclareMathOperator{\ndiv}{$\hspace{-4pt}\not|\hspace{2pt}$}
	\usepackage{proof}
	\usepackage{mathpartir}