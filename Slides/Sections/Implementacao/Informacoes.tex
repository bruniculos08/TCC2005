% \subsection[]{Informações Úteis}
\begin{frame}[fragile]{Implementações Externas}
    A prova do Lema \coqinline[]{fact_sqr_exp} é bastante complexa, por isso existem alguns pontos a serem mencionados aqui:
    \begin{itemize}
        \item Na prova é feito o uso extenso de estruturas como anéis e corpos (\textit{fields}), relacionadas pelo seguinte lema:
            \begin{lema} \label{lema : anel:corpo}
                $\forall n \in \mathbb{Z}, \mathbb{Z}/(n)$ é um corpo se e somente se $n$ é primo. 
            \end{lema}
        Quanto a implementação destas estruturas na biblioteca tem-se:
            \begin{itemize}
                \item[$\blacktriangleright$] O tipo \coqinline[]{ordinal}:
                
                    \begin{lstlisting}[language=coq,frame=single,tabsize=1]
Inductive ordinal n := Ordinal m of m < n.
Notation "'I_' n" := (ordinal n).
Coercion nat_of_ord n (i : 'I_n) := 
    let: @Ordinal _ m _ := i in m.
                    \end{lstlisting}
            \end{itemize}
    \end{itemize}
\end{frame}

\begin{frame}[fragile]{Implementações Externas}
    \begin{itemize}
        \item[] 
            \begin{itemize}
                \item[$\blacktriangleright$] A função \coqinline[]{inZp}:
                    \begin{lstlisting}[language=coq,frame=single,tabsize=1]
Definition inZp i := 
    @Ordinal p (i %% p) (ltn_pmod i (ltn0Sn p')).
                    \end{lstlisting}
                \item[$\blacktriangleright$] A implementação de anéis (com elementos neutros diferentes para cada operação) e corpos na biblioteca:
                    \begin{lstlisting}[language=coq,frame=single,tabsize=1]
Definition Zp_trunc p := p.-2.

Notation "''Z_' p" := 'I_(Zp_trunc p).+2
  (at level 8, p at level 2, format "''Z_' p") : type_scope.

Notation "''F_' p" := 'Z_(pdiv p)
  (at level 8, p at level 2, format "''F_' p") : type_scope.
                    \end{lstlisting}
            \end{itemize}
    \end{itemize}
\end{frame}

\begin{frame}[fragile]{Implementações Externas}
    \begin{itemize}
        % \item A prova faz um uso extenso do operador \coqinline[]{bigop} que é utilizado na representação de somatórios e produtórios na biblioteca Mathematical Components:
        \item A prova feita por Laurent, semelhante a prova manual, se baseia na reorganização de um produtório de tamanho arbitrário, logo há o uso extenso do operador \coqinline[]{bigop} sobre o qual tem-se:
        \begin{itemize}
            \item[$\blacktriangleright$] Definição do operador \coqinline[]{bigop}:
                \begin{lstlisting}[language=coq,frame=single,tabsize=1]
Definition bigop R I idx op r (P : pred I) (F : I -> R) : 
    R := foldr (fun i x => if P i then op (F i) x else x) 
        idx r.
                \end{lstlisting}
                                
            \item[$\blacktriangleright$] Notação para o operador \coqinline[]{bigop}:
                \begin{lstlisting}[language=coq,frame=single,tabsize=1]
 Notation "\big [ op / idx ]_ ( i <- r | P ) F" :=
    (bigop idx op r (fun i => P%B) (fun i => F)) : big_scope.
                \end{lstlisting}
        \end{itemize}
        % \item A prova feita por Laurent é semelhante a prova manual no sentido de que se baseia na reorganização de um produtório ($(p-1)!$) de tamanho arbitrário 
    \end{itemize}
\end{frame}

\begin{frame}[fragile]{Implementações Externas}
    \begin{itemize}
        \item[]
        \begin{itemize}
            \item[$\blacktriangleright$] Lema \coqinline[]{partition_big} que permite ``quebrar'' uma aplicação de \coqinline[]{bigop} em duas aplicações aninhadas:
                \begin{lstlisting}[language=coq,frame=single,tabsize=1]
Lemma partition_big {R : Type} {idx : R} 
{op : Monoid.com_law idx} {I : Type} {s : seq I} 
{J : finType} {P : pred I} (p : I -> J) (Q : pred J) 
{F : I -> R} :
(\forall i : I, P i -> Q (p i)) ->
    \big[op/idx]_(i <- s | P i) F i =
        \big[op/idx]_(j | Q j) \big[op/idx]_(i <- s | P i && (p i == j)) F i
                \end{lstlisting}

%             \item[$\blacktriangleright$] Lema \coqinline[]{eq_bigr} útil para provar a igualdade entre duas aplicações do operador \coqinline[]{bigop}:
%                 \begin{lstlisting}[language=coq,frame=single,tabsize=1]
% Lemma eq_bigr r (P : pred I) F1 F2 : (forall i, P i -> F1 i = F2 i) ->
%         \big[op/idx]_(i <- r | P i) F1 i = \big[op/idx]_(i <- r | P i) F2 i.
%                 \end{lstlisting}
        \end{itemize}
    \end{itemize}
\end{frame}

\begin{frame}[fragile]{Implementações Externas}
    \begin{itemize}
        \item[]
        \begin{itemize}
%             \item[$\blacktriangleright$] Lema \coqinline[]{partition_big} que permite ``quebrar'' uma aplicação de \coqinline[]{bigop} em duas aplicações aninhadas:
%                 \begin{lstlisting}[language=coq,frame=single,tabsize=1]
% Lemma partition_big {R : Type} {idx : R} 
% {op : Monoid.com_law idx} {I : Type} {s : seq I} 
% {J : finType} {P : pred I} (p : I -> J) (Q : pred J) 
% {F : I -> R} :
% (\forall i : I, P i -> Q (p i)) ->
%     \big[op/idx]_(i <- s | P i) F i =
%         \big[op/idx]_(j | Q j) \big[op/idx]_(i <- s | P i && (p i == j)) F i
%                 \end{lstlisting}
                
            \item[$\blacktriangleright$] Lema \coqinline[]{eq_bigr} útil para provar a igualdade entre duas aplicações do operador \coqinline[]{bigop}:
                \begin{lstlisting}[language=coq,frame=single,tabsize=1]
Lemma eq_bigr r (P : pred I) F1 F2 : 
    (forall i, P i -> F1 i = F2 i) ->
        \big[op/idx]_(i <- r | P i) F1 i = 
            \big[op/idx]_(i <- r | P i) F2 i.
                \end{lstlisting}

            \item [$\blacktriangleright$] Lema \coqinline[]{big1} útil para provar que um resultado de uma aplicação de \coqinline[]{bigop} é igual ao elemento \coqinline[]{idx}:
                \begin{lstlisting}[language=coq,frame=single,tabsize=1]
Lemma big1 {R : Type} {idx : R} {op : Monoid.law idx} I r 
    (P : pred I) F :
    (forall i : I, P i -> F i = idx) -> 
        \big[op/idx]_(i <- r | P i) F i = idx.
                \end{lstlisting}
            \textbf{Obs.:} no uso deste lema com \coqinline[]{apply} pode-se ao invés de provar que o termo geral é sempre igual a \coqinline[]{idx}, provar que nenhum tempo do tipo \coqinline[]{I} satisfaz o predicado \coqinline[]{P}.
                
        \end{itemize}
    \end{itemize}
\end{frame}

\begin{frame}[fragile]{Implementações Externas}
    \begin{itemize}
        \item[]
        \begin{itemize}
            \item[$\blacktriangleright$] Existem diversas notações relacionadas ao operador \coqinline[]{bigop} para casos específicos da aplicação deste, como para produtórios, e:
            
            \begin{itemize}
                \item[$\blackdiamond$] Para cada caso específico, em geral existe mais de uma notação;
                
                \item[$\blackdiamond$] Algumas destas se aproveitam da existência de tipos finitos (para os quais há uma lista que contém todos os elementos), como nas seguintes expressões que aparecem durante a prova:
            
                \begin{lstlisting}[language=coq,frame=single,tabsize=1]
\prod_(i in 'F_p | i != 0%R) i
                \end{lstlisting}

                \begin{lstlisting}[language=coq,frame=single,tabsize=1]
\prod_(j in 'F_p | j < f j) (j * f j)
                \end{lstlisting}

            \end{itemize}
        \end{itemize}
    \end{itemize}
\end{frame}

\begin{frame}[fragile]{Implementações Externas}
    \begin{itemize}
        \item Durante a prova aparecem diversos conjuntos que são definidos por meio de predicados sobre tipos finitos como em:
            \begin{lstlisting}[language=coq,frame=single,tabsize=1]
set A := [predD1 'F_p & 0%R].
            \end{lstlisting}

            \begin{lstlisting}[language=coq,frame=single,tabsize=1]
pose B := [pred i |  (i : 'F_p) < f i].
            \end{lstlisting}

        assim \coqinline[]{A} é contém os elementos de \coqinline[]{'F_p} diferentes de \coqinline[]{0} e \coqinline[]{B} é contém todos os elementos de \coqinline[]{'F_p} tal que \coqinline[]{i < f i}. 
    \end{itemize}
\end{frame}