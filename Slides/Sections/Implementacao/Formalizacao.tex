\begin{frame}[fragile]{Formalização do Símbolo de Legendre}
    O \textit{símbolo de Legendre} foi implementado por meio da seguinte função:
        
        \begin{lstlisting}[language=coq,frame=single,tabsize=1]
Definition legendre_symb {p : int} (pL2 : (2 < p)%R) 
    (pP : primez.primez p) (a : int) :=
        if (p %| a)%Z then 0%Z else if (resz_quad p a)
        then 1%Z else (-1)%Z.
        \end{lstlisting}
        
        onde \coqinline[]{resz_quad} tem a seguinte definição (que por sua vez é baseada na definição de \coqinline[]{res_quad}):
        
        \begin{lstlisting}[language=coq,frame=single,tabsize=1]
Definition resz_quad p a := 
    has (fun i => ((i * i)%:Z == a %[mod p])%Z) (iota 0 `|p|).
        \end{lstlisting}
    
\end{frame}

\begin{frame}[fragile]{Formalização do Símbolo de Legendre}
    Um exemplo de uso da função \coqinline[]{legendre_symb} é:
            \begin{lstlisting}[language=coq,frame=single,tabsize=1]
Compute (legendre_symb (_ : 2 < 7)%R 
    (_ : primez.primez 7) 2).
            \end{lstlisting}

    A prova de corretude foi implementada usando o tipo indutivo \coqinline[]{reflect} e divido em duas partes por fins práticos (uso em táticas da \textit{ssreflect}).

\end{frame}

\begin{frame}[fragile]{Formalização do Símbolo de Legendre}
    Quanto a prova de corretude, esta foi implementada usando o tipo indutivo \coqinline[]{reflect} e foi divida em duas partes por fins práticos (uso em táticas da \textit{ssreflect}):
    \newline
        \begin{lstlisting}[language=coq,frame=single,tabsize=1, escapechar=@]
Theorem legendre_symbP {p : int} (pL2 : (2 < p)%R) 
    (pP : primez.primez p) (a : int):
    reflect (exists x, x^2 = a %[mod p]) 
    (if (p %| a)%Z then (((legendre_symb pL2 pP a) == 0)) 
    else ((legendre_symb pL2 pP a) == 1)).
        \end{lstlisting}
            
        \begin{lstlisting}[language=coq,frame=single,tabsize=1, escapechar=@]
Theorem legendre_symbnP {p : int} (pL2 : (2 < p)%R) (pP : primez.primez p) (a : int):
    reflect (~ exists x, x^2 = a %[mod p]) 
    ((legendre_symb pL2 pP a) == -1).
        \end{lstlisting}
\end{frame}

\begin{frame}[fragile]{Formalização do Símbolo de Legendre}
    Foram provadas todas as propriedades sobre o \textit{símbolo de Legendre} e uma versão do Critério de Euler idêntica apresentadas em \cite{book:2399854}, além de algumas propriedades auxiliares, mas por fim de breviedade serão mostradas aqui apenas os 2 primeiros grupos:
    \vspace{4mm}
    \\
    O enunciado do Critério de Euler em \textit{Coq} é dado por:
        \begin{lstlisting}[language=coq,frame=single,tabsize=1]
Lemma eulerz_criterion {p : int} (pL2 : (2 < p)%R) 
(pP : primez.primez p) (a : int):
    (a ^ ((p - 1) %/ 2)%Z = 
        (legendre_symb pL2 pP a) %[mod p])%Z.
        \end{lstlisting}

\end{frame}

\begin{frame}[fragile]{Formalização do Símbolo de Legendre}
    
    e considerando um número primo $p > 2$ e $a, b \in \mathbb{Z}$ tem-se as seguintes propriedades:
        
        \begin{itemize}
            \item $a \equiv b \pmod{p} \rightarrow \left(\frac{a}{p}\right) = \left(\frac{b}{p}\right)$, cuja declaração dada em \textit{Coq} é:
            \newline
                \begin{lstlisting}[language=coq,frame=single,tabsize=1]
Lemma legendre_symbE (p a b : int) (pL2 : (2 < p)%R) 
(pP : primez.primez p):
   (a == b %[mod p])%Z -> 
   ((legendre_symb pL2 pP a) == (legendre_symb pL2 pP b)).
                \end{lstlisting}

        \end{itemize}
\end{frame}


\begin{frame}[fragile]{Formalização do Símbolo de Legendre}

    \begin{itemize}
        \item $p \nmid a \rightarrow \left(\frac{a^2}{p}\right) = 1$, cuja declaração dada em \textit{Coq} é:
        \newline
            \begin{lstlisting}[language=coq,frame=single,tabsize=1]
Lemma legendre_symb_Ndvd (p a b : int) (pL2 : (2 < p)%R) 
(pP : primez.primez p):
~~(p %| a)%Z -> (legendre_symb pL2 pP (a^2)) == 1.
            \end{lstlisting}

        \item $\left(\frac{-1}{p}\right) = (-1)^{\frac{p - 1}{2}} = 1 \leftrightarrow p \equiv 1 \pmod{4}$, cuja declaração dada em \textit{Coq} é:
        \newline
            \begin{lstlisting}[language=coq,frame=single,tabsize=1]
Lemma legendre_symb_Neg1 (p : int) (pL2 : (2 < p)%R) 
(pP : primez.primez p):
((legendre_symb pL2 pP (-1)) == 1) = (p == 1 %[mod 4])%Z.
            \end{lstlisting}

    \end{itemize}

\end{frame}

\begin{frame}[fragile]{Formalização do Símbolo de Legendre}

    \begin{itemize}
        \item $\left(\frac{a \cdot b}{p}\right) = \left(\frac{a}{p}\right) \cdot \left(\frac{b}{p}\right)$, cuja declaração dada em \textit{Coq} é:
        \newline
            \begin{lstlisting}[language=coq,frame=single,tabsize=1]
Lemma legendre_symb_mul (p a b : int) (pL2 : (2 < p)%R) (pP : primez.primez p):
    (legendre_symb pL2 pP (a * b)%R) = 
    ((legendre_symb pL2 pP a) * 
        (legendre_symb pL2 pP b))%R.
            \end{lstlisting}
    \end{itemize}

\end{frame}