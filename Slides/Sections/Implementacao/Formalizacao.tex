\begin{frame}[fragile]{Formalização do Símbolo de Legendre}
    O \textit{símbolo de Legendre} foi implementado por meio da seguinte função:
        
        \begin{lstlisting}[language=coq,frame=single,tabsize=1]
Definition legendre_symb {p : int} (pL2 : (2 < p)%R) 
    (pP : primez p) (a : int) :=
        if (p %| a)%Z then 0%Z else if (resz_quad p a)
        then 1%Z else (-1)%Z.
        \end{lstlisting}
        
        onde \coqinline[]{resz_quad} tem a seguinte definição (que por sua vez é baseada na definição de \coqinline[]{res_quad}):
        
        \begin{lstlisting}[language=coq,frame=single,tabsize=1]
Definition resz_quad p a := 
    has (fun i => ((i * i)%:Z == a %[mod p])%Z) (iota 0 `|p|).
        \end{lstlisting}
    
\end{frame}

\begin{frame}[fragile]{Formalização do Símbolo de Legendre}
    Um exemplo de uso da função \coqinline[]{legendre_symb} é:
            \begin{lstlisting}[language=coq,frame=single,tabsize=1]
Compute (legendre_symb (_ : 2 < 7)%R (_ : primez 7) 2).
            \end{lstlisting}

\end{frame}

\begin{frame}[fragile]{Formalização do Símbolo de Legendre}
    Quanto a prova de corretude, esta foi implementada usando o tipo indutivo \coqinline[]{reflect} e foi divida em duas partes por fins práticos (uso em táticas da \textit{ssreflect}):
    \newline
        \begin{lstlisting}[language=coq,frame=single,tabsize=1, escapechar=@]
Theorem legendre_symbP {p : int} (pL2 : (2 < p)%R) 
    (pP : primez p) (a : int):
    reflect (exists x, x^2 = a %[mod p]) 
    (if (p %| a)%Z then (((legendre_symb pL2 pP a) == 0)) 
    else ((legendre_symb pL2 pP a) == 1)).
        \end{lstlisting}
            
        \begin{lstlisting}[language=coq,frame=single,tabsize=1, escapechar=@]
Theorem legendre_symbnP {p : int} (pL2 : (2 < p)%R) (pP : primez p) (a : int):
    reflect (~ exists x, x^2 = a %[mod p]) 
    ((legendre_symb pL2 pP a) == -1).
        \end{lstlisting}
\end{frame}

\begin{frame}[fragile]{Formalização do Símbolo de Legendre}
    Foram provadas todas as propriedades sobre o \textit{símbolo de Legendre} e uma versão do Critério de Euler idêntica apresentadas em \cite{book:2399854}, além de algumas propriedades auxiliares, mas por fim de breviedade serão mostradas aqui apenas os 2 primeiros grupos:
    \vspace{4mm}
    \\
    O enunciado do Critério de Euler em \textit{Coq} é dado por:
        \begin{lstlisting}[language=coq,frame=single,tabsize=1]
Lemma eulerz_criterion {p : int} (pL2 : (2 < p)%R) 
(pP : primez p) (a : int):
    (a ^ ((p - 1) %/ 2)%Z = 
        (legendre_symb pL2 pP a) %[mod p])%Z.
        \end{lstlisting}

\end{frame}

\begin{frame}[fragile]{Formalização do Símbolo de Legendre}
    
    e considerando um número primo $p > 2$ e $a, b \in \mathbb{Z}$ tem-se as seguintes propriedades:
        
        \begin{itemize}
            \item $a \equiv b \pmod{p} \rightarrow \left(\frac{a}{p}\right) = \left(\frac{b}{p}\right)$, cuja declaração dada em \textit{Coq} é:
            \newline
                \begin{lstlisting}[language=coq,frame=single,tabsize=1]
Lemma legendre_symbE (p a b : int) (pL2 : (2 < p)%R) 
(pP : primez p):
   (a == b %[mod p])%Z -> 
   ((legendre_symb pL2 pP a) == (legendre_symb pL2 pP b)).
                \end{lstlisting}

        \end{itemize}
\end{frame}


\begin{frame}[fragile]{Formalização do Símbolo de Legendre}

    \begin{itemize}
        \item $p \nmid a \rightarrow \left(\frac{a^2}{p}\right) = 1$, cuja declaração dada em \textit{Coq} é:
        \newline
            \begin{lstlisting}[language=coq,frame=single,tabsize=1]
Lemma legendre_symb_Ndvd (p a b : int) (pL2 : (2 < p)%R) 
(pP : primez p):
~~(p %| a)%Z -> (legendre_symb pL2 pP (a^2)) == 1.
            \end{lstlisting}

        \item $\left(\frac{-1}{p}\right) = (-1)^{\frac{p - 1}{2}} = 1 \leftrightarrow p \equiv 1 \pmod{4}$, cuja declaração dada em \textit{Coq} é:
        \newline
            \begin{lstlisting}[language=coq,frame=single,tabsize=1]
Lemma legendre_symb_Neg1 (p : int) (pL2 : (2 < p)%R) 
(pP : primez p):
((legendre_symb pL2 pP (-1)) == 1) = (p == 1 %[mod 4])%Z.
            \end{lstlisting}

    \end{itemize}

\end{frame}

\begin{frame}[fragile]{Formalização do Símbolo de Legendre}

    \begin{itemize}
        \item $\left(\frac{a \cdot b}{p}\right) = \left(\frac{a}{p}\right) \cdot \left(\frac{b}{p}\right)$, cuja declaração dada em \textit{Coq} é:
        \newline
            \begin{lstlisting}[language=coq,frame=single,tabsize=1]
Lemma legendre_symb_mul (p a b : int) (pL2 : (2 < p)%R) (pP : primez p):
    (legendre_symb pL2 pP (a * b)%R) = 
    ((legendre_symb pL2 pP a) * 
        (legendre_symb pL2 pP b))%R.
            \end{lstlisting}
    \end{itemize}

\end{frame}

\begin{frame}[fragile]{Formalização do Símbolo de Legendre}

    \begin{itemize}
        \item $\left(\frac{a \cdot b}{p}\right) = \left(\frac{a}{p}\right) \cdot \left(\frac{b}{p}\right)$, cuja declaração dada em \textit{Coq} é:
        \newline
            \begin{lstlisting}[language=coq,frame=single,tabsize=1]
Lemma legendre_symb_mul (p a b : int) (pL2 : (2 < p)%R) (pP : primez p):
    (legendre_symb pL2 pP (a * b)%R) = 
    ((legendre_symb pL2 pP a) * 
        (legendre_symb pL2 pP b))%R.
            \end{lstlisting}
    \end{itemize}

\end{frame}

\begin{frame}[fragile]{Formalização do Símbolo de Legendre}
    Algumas informações usadas durante as provas destes enunciados são muito úteis para futuros usuários da biblioteca; se destacam os seguintes itens:
    \begin{itemize}
        \item Os números inteiros são definidos da seguinte forma na biblioteca:
            \begin{lstlisting}[language=coq,frame=single,tabsize=1]
Variant int : Set := Posz of nat | Negz of nat.
            \end{lstlisting}
        e \coqinline[]{Posz} é uma definida como coerção de \coqinline[]{nat} para \coqinline[]{int}, o que permite utilizar naturais em expressões com números inteiros.

%         \item  A aplicação do construtor \coqinline[]{Posz} não é impressa, o que gera ambiguidades como entre as expressões \coqinline[]{Posz (a - b)} e \coqinline[]{Posz a - Posz b} que são impressas como \coqinline[]{a - b}; 
%             \begin{itemize}
%                 \item[$\blacktriangleright$] caso o usuário deseje imprimir as aplicações de \coqinline[]{Posz} ele pode usar o comando:
%                     \begin{lstlisting}[language=coq,frame=single,tabsize=1]
% Set Printing Coercions.
%                     \end{lstlisting}
%             \end{itemize}
        
    \end{itemize}
\end{frame}

\begin{frame}[fragile]{Formalização do Símbolo de Legendre}

    \begin{itemize}

        \item  A aplicação do construtor \coqinline[]{Posz} não é impressa, o que gera ambiguidades como entre as expressões \coqinline[]{Posz (a - b)} e \coqinline[]{Posz a - Posz b} que são impressas como \coqinline[]{a - b} \cite{mathcomp-ssrint}; 
            \begin{itemize}
                \item[$\blacktriangleright$] caso o usuário deseje imprimir as aplicações de \coqinline[]{Posz} ele pode usar o comando:
                    \begin{lstlisting}[language=coq,frame=single,tabsize=1]
Set Printing Coercions.
                    \end{lstlisting}
            \end{itemize}

        \item A aplicação da função \coqinline[]{Posz} possui certas propriedades, por exemplo:
            \begin{itemize}
                \item[$\blacktriangleright$] É sempre o caso de que \coqinline[]{Posz m + n = Posz m + Posz n}, assim tal reescrita pode ser feita utilizando o lema:
                    \begin{lstlisting}[language=coq,frame=single,tabsize=1]
Lemma PoszD : {morph Posz : m n / (m + n)%N >-> m + n}. 
                    \end{lstlisting}
                e para multiplicação, de forma análoga, pode ser utilizado o lema \coqinline[]{PoszM}.

%                 \item[$\blacktriangleright$] Quando \coqinline[]{n <= n} tem-se \coqinline[]{Posz m - n = Posz m - Posz n}, assim tal reescrita pode ser feita utilizando o lema:
%                     \begin{lstlisting}[language=coq,frame=single,tabsize=1]
%  Lemma subzn (m n : nat) : 
%     (n <= m)%N -> m%:Z - n%:Z = (m - n)%N. 
%                     \end{lstlisting}

                % \item[$\blacktriangleright$] há também lemas para fazer manipulações semelhantes com outras operações (mas as vezes é necessário reescrever o operador como sua versão para aneis).
            \end{itemize}
    \end{itemize}
\end{frame}

\begin{frame}[fragile]{Formalização do Símbolo de Legendre}
    \begin{itemize}
        \item[]
        \begin{itemize}
            \item[$\blacktriangleright$] Quando \coqinline[]{n <= n} tem-se \coqinline[]{Posz m - n = Posz m - Posz n}, assim tal reescrita pode ser feita utilizando o lema:
                \begin{lstlisting}[language=coq,frame=single,tabsize=1]
 Lemma subzn (m n : nat) : 
    (n <= m)%N -> m%:Z - n%:Z = (m - n)%N. 
                \end{lstlisting}

            \item[$\blacktriangleright$] Para operação de resto da divisão tem-se o lema:
                \begin{lstlisting}[language=coq,frame=single,tabsize=1]
Lemma modz_nat (m d : nat) : (m %% d)%Z = (m %% d)%N.
                \end{lstlisting}
            \item[$\blacktriangleright$] Há também lemas para fazer manipulações semelhantes com outras operações (mas as vezes é necessário reescrever o operador em sua versão para aneis):
                \begin{lstlisting}[language=coq,frame=single,tabsize=1]
Lemma rmorphXn n : {morph f : x / x ^+ n}.
                \end{lstlisting}
        \end{itemize}
    \end{itemize}
\end{frame}

\begin{frame}[fragile]{Formalização do Símbolo de Legendre}
    \begin{itemize}
        \item Para manipulações de expressões com \coqinline[]{int} não se utilizam lemas específicos para este tipo (por motivo explicado na Subsessão $2.3.4$);
            \begin{itemize}
                \item[$\blacktriangleright$] Para comutatividade da adição, por exemplo, se usa \coqinline[]{addrC}, para o qual o comando \coqinline[]{Print} traz a mensagem:
                    \begin{lstlisting}[language=coq-error,frame=single,tabsize=1]
GRing.addrC =
fun s : nmodType =>
GRing.isNmodule.addrC
    (GRing.Nmodule.GRing_isNmodule_mixin 
        (GRing.Nmodule.class s))
            : forall s : nmodType, commutative +%R

Arguments GRing.addrC {s} x y
                    \end{lstlisting}
            \end{itemize}
    \end{itemize}
\end{frame}

\begin{frame}[fragile]{Formalização do Símbolo de Legendre}
    \begin{itemize}
        \item Para lidar com o tipo \coqinline[]{ordinal} e coerções:
            \begin{itemize}
                \item[$\blacktriangleright$] Para uma hipótese da forma:
                    \begin{lstlisting}[language=coq,frame=single,tabsize=1]
H : forall i : 'I_p, P i
                    \end{lstlisting}
                tendo no contexto as variáveis \coqinline[]{p : int} e \coqinline[]{x : 'Z_p}, não é possível instânciar tal hipótese com \coqinline[]{x} pois é necessário que a 
                função aplicada sobre \coqinline[]{p} na notação resulte em \coqinline[]{p}, ou seja, é necessário escrever \coqinline[]{p} como alguma variável somada a $2$ o que pode ser resolvido pelo uso sucessivo da tática \coqinline[]{case: p => p}. 
                % Anotação: pois se p = 0 ou p = 1 então tem-se 'I_0 ou 'I_1 mas x, de tipo 'Z_1 ou 'Z_0 (frase talvez imprecisa), terá tipo 'I_2.
                \item[]
                \item[{$\blacktriangleright$}] A notação \coqinline[]{:> X} após uma equação (ou expressões como \coqinline[]{<>}) indica o tipo de ambos os termos como \coqinline[]{X} (essa notação é implementada na livraria padrão de \textit{Coq}), o que pode ser usado para forçar uma coerção; como exemplo tem-se o lema:
                    \begin{lstlisting}[language=coq,frame=single,tabsize=1]
Lemma natz (n : nat) : n%:R = n%:Z :> int.
                    \end{lstlisting}
                

            \end{itemize}
            
    \end{itemize}
\end{frame}