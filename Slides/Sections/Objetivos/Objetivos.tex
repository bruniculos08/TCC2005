% \begin{frame}{Objetivo Geral (TCC2)}
\begin{frame}{Objetivo Geral}
    Implementar o \textit{símbolo de Legendre} e realizar a formalização de suas propriedades (apresentadas em \cite{book:2399854}) e da corretude (da função que o implementa). 
\end{frame}

\begin{frame}{Objetivos Específicos}
    \begin{enumerate}
            \item Obter conhecimentos avançados sobre o assistente de provas \textit{Coq}.
            \item Realizar o estudo sobre as principais documentações da biblioteca Mathematical Components.
            \item Desenvolver a capacidade de realizar provas em \textit{Coq} utilizando as táticas da linguagem de provas \textit{SSReflect}.
            \item Estudar conteúdos de Teoria dos Números relacionados ao símbolo de Legendre.
            \item Implementar uma função que compute o valor do \textit{símbolo de Legendre} e provar a corretude da mesma se utilizando da biblioteca Mathematical Components.
            \item Provar teoremas úteis para manipulação de expressões envolvendo o símbolo de Legendre utilizando-se da biblioteca Mathematical Components.
            % \item Realizar uma introdução para conceitos básicos de Teoria dos Números;
            % \item Explicar conteúdos sobre \textit{Coq} relacionados a biblioteca Mathematical Components e considerados relevantes para apresentação de determinadas implementações disponíveis nessa;
            % \item Explorar o conteúdo teórico de Teoria dos Números necessário para a formalização do algoritmo \textit{RESSOL} e da Lei de Reciprocidade Quadrática;
            % \item Apresentar a definicao, prova manual de terminação e de corretude do algoritmo \textit{RESSOL};
            % \item Apresentar a definicao e prova manual da Lei de Reciprocidade Quadrática.
    \end{enumerate}
\end{frame}