\begin{frame}[fragile]{Congruência de Grau 2 e Símbolos de Legendre}
    \begin{itemize}
        \item Sendo $p$ um número primo maior que $2$ e $a,b,c \in \mathbb{Z}$ números não divisíveis por $p$, como motivação suponha que se deseje resolver a seguinte equação:
        \begin{equation} \label{eq : ax2bxc}
            a \cdot x^2 + b \cdot x + c \equiv 0 \pmod p
        \end{equation}
        Manipulando essa equação por meio das propriedades de congruência se obtém:
        \begin{equation} \label{eq : bhaskara}
            (2 \cdot a \cdot x + b)^2 \equiv b^2 - 4 \cdot a \cdot c \pmod p
        \end{equation}
    \end{itemize}
\end{frame}

\begin{frame}[fragile]{Congruência de Grau 2 e Símbolos de Legendre}
    \begin{itemize}
        \item Realizando a substituição $X = 2 \cdot a \cdot x + b$ e $d = b^2 - 4 \cdot a \cdot c$ na Equação \ref{eq : bhaskara}, tem-se:
        \begin{equation} \label{eq : quadcong}
            X^2 \equiv d \pmod p
        \end{equation}
        Portanto, resolver a Equação \ref{eq : ax2bxc} é equivalente a resolver a Equação \ref{eq : quadcong};

    \item Sobre a Equação \ref{eq : quadcong}, se diz que $d$ é um quadrado perfeito em $\mathbb{Z}/(p)$ e também que $d$ é um \textit{resíduo quadrático módulo $p$}.
    
    \end{itemize}
\end{frame}

\begin{frame}[fragile]{Congruência de Grau 2 e Símbolos de Legendre}
    Conforme \cite{book:2399854}, existem $\frac{p+1}{2}$ resíduos quadráticos módulo $p$, que são:
        \begin{equation} \label{eq : listquadres}
            0^2 \bmod{p}, 1^2 \bmod{p}, 2^2 \bmod{p}, ..., \left(\frac{p -1}{2} \right)^2 \bmod{p} 
        \end{equation}
    pois note que, para todo $x \in \mathbb{Z}$ existe algum $i \in [0, \frac{p-1}{2}]$ tal que $x \equiv i \pmod{p}$ ou $x \equiv -i \pmod{p}$, logo $x^2 \equiv i^2 \pmod{p}$ (usando a propriedade do Item \ref{item:propcong6-produto}) e $i^2$ está na Lista \ref{eq : listquadres}.
\end{frame}

\begin{frame}[fragile]{Congruência de Grau 2 e Símbolos de Legendre}
    Além disso, todos os os valores na Lista \ref{eq : listquadres} são distintos em módulo $p$, pois para todo $i, j \in \left[0, \frac{p-1}{2}\right]$:
    \begin{align}
        i^2 \equiv j^2 \pmod p
        &\begin{aligned}
            \;\; \Longleftrightarrow p \mid (i^2- j^2)
        \end{aligned} \\
        &\begin{aligned}
            \;\; \Longleftrightarrow p \mid (i -j)\cdot(i + j)
        \end{aligned} \\
        &\begin{aligned} \label{eq : ordivp}
            \;\; \Longleftrightarrow p \mid (i -j) \lor p \mid (i + j)
        \end{aligned}
    \end{align}
    Com isso, dado o intervalo de $i$ e $j$, então $0 \leq i + j \leq p - 1$, assim existem as seguintes possibilidades:
        \begin{enumerate}
            \item $i = j = 0$ e portanto $i \equiv j \pmod p$;
            \item $0 < i + j \leq p-1$, portanto $p \nmid i + j$, e então pela disjunção em \ref{eq : ordivp} resta que $p \mid (i - j)$, o que equivale a $i \equiv j \pmod p$, ou seja, $i$ é igual $j$ módulo $p$ se e somente se seus quadrados também são.
        \end{enumerate}
\end{frame}

\begin{frame}[fragile]{Congruência de Grau 2 e Símbolos de Legendre}
    Com essas conclusões (de que a Lista \ref{eq : listquadres} contém todos os resíduos quadráticos módulo $p$ e que todos os valores dela são distintos em módulo $p$) pode ser provado o seguinte lema:
        \begin{lema} \label{lema:existnonquadratic}
            Seja $p > 2$ um número primo, existem exatamente $\frac{p+1}{2}$ resíduos quadráticos módulo $p$ e $\frac{p-1}{2}$ resíduos não quadráticos módulo $p$.
        \end{lema}
\end{frame}

\begin{frame}[fragile]{Congruência de Grau 2 e Símbolos de Legendre}
    \begin{definicao}[Símbolo de Legendre]
        Seja $p > 2$ um número primo e $a \in \mathbb{Z}$, se define o símbolo de Legendre por:
        \begin{equation*}
            \left( \frac{a}{p} \right) = \begin{cases}
                1 \text{, se $p \nmid a$ e $a$ é um resíduo quadrático módulo $p$}
                \\
                0 \text{, se $p \mid a$}
                \\
                -1 \text{, caso contrário ($a$ não é um resíduo quadrático)}
                \end{cases}
        \end{equation*}
    \end{definicao}
\end{frame}

\begin{frame}[fragile]{Congruência de Grau 2 e Símbolos de Legendre}
    Uma maneira de se computar o valor de um \textit{símbolo de Legendre} é por meio do \textit{Critério de Euler}, descrito a seguir:
    \begin{teorema}[\textit{Critério de Euler}] Para todo $a \in \mathbb{Z}$, seja $p > 2$ um número primo, então: \label{teorema:criteriodeeuler}
        \begin{equation*}
            \left( \frac{a}{p} \right) \equiv a^{\frac{p-1}{2}} \pmod p
        \end{equation*}
    \end{teorema}
\end{frame}

\begin{frame}[fragile]{Congruência de Grau 2 e Símbolos de Legendre}
    Para a se realizar a prova do \textit{Critério de Euler} são necessários as definições, lemas e teoremas dados a seguir:

        \begin{definicao}[\textit{Inverso multiplicativo módulo $n$}]
            Dados $a, m, n \in \mathbb{Z}$, se $a \cdot m \equiv 1 \pmod n$, se diz que $m$ é um \textit{inverso de $a$ módulo $n$}, e pode ser denotado por $a^{-1}$.
        \end{definicao}

        \begin{lema} \label{lema : mdcinv}
            Para todo $a, n \in \mathbb{Z}$, se $n > 0$, então, existe $b \in \mathbb{Z}$ tal que $a \cdot b \equiv 1 \pmod{n}$ se, e somente se, $\mdc(a, n) = 1$.
        \end{lema}
    
\end{frame}

\begin{frame}[fragile]{Congruência de Grau 2 e Símbolos de Legendre}

    \begin{lema}[\textit{Unicidade de inverso multiplicativo módulo p}] Dado um número primo $p$, seja 
        $a \in [1, p-1]$, existe $k \in [1, p-1]$ tal que $ a \cdot k \equiv 1 \pmod p$ e $k$ é portanto o único inverso multiplicativo de  módulo $p$ de $a$ no intervalo $[1, p-1]$. \label{lema : invmod}
    \end{lema}

    \begin{lema} 
    Seja $a \in [1, p-1]$ em que $p$ é um número primo maior que $2$, se $x^2 \equiv a \pmod p$ não tem solução, então para todo $h \in [1, p-1]$ existe $k \in [1, p-1]$, tal que: \label{lema : hkequivamodp}
        \begin{equation*} 
            h \neq k \land h \cdot k \equiv a \pmod p 
        \end{equation*}
    \end{lema}

\end{frame}

\begin{frame}[fragile]{Congruência de Grau 2 e Símbolos de Legendre}

    \begin{lema} \label{lema : kk'modp}
    Seja $a, h, k, k' \in [1, p-1]$, se $k \cdot h \equiv a \pmod{p}$ e $k' \cdot h \equiv a \pmod{p}$ então $k = k'$ ($k$ é único).
    \end{lema}

    \begin{lema} \label{lema : modp-1fat}
    Seja $p > 2$ um número primo, para todo $a \in \mathbb{Z}$, se $\mdc(a, p) = 1$ e $x^2 \equiv a \pmod p$ não tem solução então:
        \begin{equation*}
            (p - 1)! \equiv a^{\frac{p-1}{2}} \pmod{p}
        \end{equation*}
    \end{lema}
        
\end{frame}

\begin{frame}[fragile]{Congruência de Grau 2 e Símbolos de Legendre}

        \begin{lema} Seja $p$ um número primo, então para quaisquer soluções de $x^2 \equiv 1 \pmod{p}$ têm-se que $x \equiv 1 \pmod{p}$ ou $x \equiv -1 \pmod{p}$. Portanto para qualquer outro valor $y$ que não é uma solução, $y \not\equiv y^{-1} \pmod{p}$.
        \label{lema : eq1modp}
        \end{lema}

        \begin{teorema}[\textit{Teorema de Wilson}] \label{teorema : wilson}
            Seja número composto um número que pode ser escrito como a multiplicação de dois outros números menores então, dado $n > 1$:
            \begin{equation*}
                (n - 1)! \equiv \begin{cases}
                                -1 \pmod{n} \; \textit{se $n$ é primo} \\
                                0 \pmod{n} \; \textit{se $n$ é composto e $n \neq 4$}
                                \end{cases}
            \end{equation*}
        \end{teorema}
        
\end{frame}

% \begin{frame}[fragile]{Congruência de Grau 2 e Símbolos de Legendre}
%     \begin{itemize}
%         \item Com esses itens pode ser realizada a prova do \textit{Critério de Euler};

%         \item Todos esses itens junto ao \textit{Critério de Euler} não estão implementados na biblioteca Mathematical Components, portanto constituem uma etapa intermediária para que se alcance o objetivo deste trabalho.
%     \end{itemize}
% \end{frame}

\begin{frame}{Congruência de Grau 2 e Símbolos de Legendre}
        Com esses itens apresentados pode ser realizada a prova do \textit{Critério de Euler}, qual é por sua vez o teorema mais importante para a formalização do algoritmo \textit{RESSOL}.
        \smallskip
        \smallskip

        \textbf{Obs.:} todos esses itens junto ao \textit{Critério de Euler} não estão implementados na biblioteca Mathematical Components, portanto constituem uma etapa intermediária para que se alcance o objetivo deste trabalho.
\end{frame}