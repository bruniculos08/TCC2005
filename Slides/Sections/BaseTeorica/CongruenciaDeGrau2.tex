\begin{frame}[fragile]{Congruência de Grau 2 e Símbolos de Legendre}
    Motivação sobre a resolução de congruências de grau 2:
    \vspace{4mm}
    \begin{itemize}
        \item Sendo $p$ um número primo maior que $2$ e $a,b,c \in \mathbb{Z}$ números não divisíveis por $p$, como motivação suponha que se deseje resolver a seguinte equação:
        \begin{equation} \label{eq : ax2bxc}
            a \cdot x^2 + b \cdot x + c \equiv 0 \pmod p
        \end{equation}
        Manipulando essa equação por meio das propriedades de congruência se obtém:
        \begin{equation} \label{eq : bhaskara}
            (2 \cdot a \cdot x + b)^2 \equiv b^2 - 4 \cdot a \cdot c \pmod p
        \end{equation}
    \end{itemize}
\end{frame}

\begin{frame}[fragile]{Congruência de Grau 2 e Símbolos de Legendre}
    Motivação sobre a resolução de congruências de grau 2: 
    (continuação)
    \vspace{4mm}
    \begin{itemize}
        \item Realizando a substituição $X = 2 \cdot a \cdot x + b$ e $d = b^2 - 4 \cdot a \cdot c$ na Equação \ref{eq : bhaskara}, tem-se:
        \begin{equation} \label{eq : quadcong}
            X^2 \equiv d \pmod p
        \end{equation}
        Portanto, resolver a Equação \ref{eq : ax2bxc} é equivalente a resolver a Equação \ref{eq : quadcong}.
    \end{itemize}
    \vspace{4mm}
    Sobre a Equação \ref{eq : quadcong}, se diz que $d$ é um quadrado perfeito em $\mathbb{Z}/(p)$ e também que $d$ é um \textit{resíduo quadrático módulo $p$}.

\end{frame}

% \begin{frame}[fragile]{Congruência de Grau 2 e Símbolos de Legendre}
%     Conforme \cite{book:2399854}, existem $\frac{p+1}{2}$ resíduos quadráticos módulo $p$, que são:
%         \begin{equation} \label{eq : listquadres}
%             0^2 \bmod{p}, 1^2 \bmod{p}, 2^2 \bmod{p}, ..., \left(\frac{p -1}{2} \right)^2 \bmod{p} 
%         \end{equation}
%     pois note que, para todo $x \in \mathbb{Z}$ existe algum $i \in [0, \frac{p-1}{2}]$ tal que $x \equiv i \pmod{p}$ ou $x \equiv -i \pmod{p}$, logo $x^2 \equiv i^2 \pmod{p}$ (usando a propriedade do Item \ref{item:propcong6-produto}) e $i^2$ está na Lista \ref{eq : listquadres}.
% \end{frame}

% \begin{frame}[fragile]{Congruência de Grau 2 e Símbolos de Legendre}
%     Além disso, todos os os valores na Lista \ref{eq : listquadres} são distintos em módulo $p$, pois para todo $i, j \in \left[0, \frac{p-1}{2}\right]$:
%     \begin{align}
%         i^2 \equiv j^2 \pmod p
%         &\begin{aligned}
%             \;\; \Longleftrightarrow p \mid (i^2- j^2)
%         \end{aligned} \\
%         &\begin{aligned}
%             \;\; \Longleftrightarrow p \mid (i -j)\cdot(i + j)
%         \end{aligned} \\
%         &\begin{aligned} \label{eq : ordivp}
%             \;\; \Longleftrightarrow p \mid (i -j) \lor p \mid (i + j)
%         \end{aligned}
%     \end{align}
%     Com isso, dado o intervalo de $i$ e $j$, então $0 \leq i + j \leq p - 1$, assim existem as seguintes possibilidades:
%         \begin{enumerate}
%             \item $i = j = 0$ e portanto $i \equiv j \pmod p$;
%             \item $0 < i + j \leq p-1$, portanto $p \nmid i + j$, e então pela disjunção em \ref{eq : ordivp} resta que $p \mid (i - j)$, o que equivale a $i \equiv j \pmod p$, ou seja, $i$ é igual $j$ módulo $p$ se e somente se seus quadrados também são.
%         \end{enumerate}
% \end{frame}

% \begin{frame}[fragile]{Congruência de Grau 2 e Símbolos de Legendre}
%     Com essas conclusões (de que a Lista \ref{eq : listquadres} contém todos os resíduos quadráticos módulo $p$ e que todos os valores dela são distintos em módulo $p$) pode ser provado o seguinte lema:
%         \begin{lema} \label{lema:existnonquadratic}
%             Seja $p > 2$ um número primo, existem exatamente $\frac{p+1}{2}$ resíduos quadráticos módulo $p$ e $\frac{p-1}{2}$ resíduos não quadráticos módulo $p$.
%         \end{lema}
% \end{frame}

\begin{frame}[fragile]{Congruência de Grau 2 e Símbolos de Legendre}
    \begin{definicao}[Símbolo de Legendre]
        Seja $p > 2$ um número primo e $a \in \mathbb{Z}$, se define o símbolo de Legendre por:
        \begin{equation*}
            \left( \frac{a}{p} \right) = \begin{cases}
                1 \text{, se $p \nmid a$ e $a$ é um resíduo quadrático módulo $p$}
                \\
                0 \text{, se $p \mid a$}
                \\
                -1 \text{, caso contrário ($a$ não é um resíduo quadrático)}
                \end{cases}
        \end{equation*}
    \end{definicao}
\end{frame}

\begin{frame}[fragile]{Congruência de Grau 2 e Símbolos de Legendre}
    \begin{teorema}[\textit{Critério de Euler}] Para todo $a \in \mathbb{Z}$, seja $p > 2$ um número primo, então: \label{teorema:criteriodeeuler}
        \begin{equation*}
            \left( \frac{a}{p} \right) \equiv a^{\frac{p-1}{2}} \pmod p
        \end{equation*}
    \end{teorema}
\end{frame}

\begin{frame}[fragile]{Congruência de Grau 2 e Símbolos de Legendre}

    Para a prova do Critério de Euler, tanto na versão feita por Laurent Théry (que auxilou na realização deste trabalho) quanto na prova manual apresentada neste trabalho foram necessários os seguintes enunciados: 

    \begin{lema} \label{lema : modp-1fat}
    Seja $p > 2$ um número primo, para todo $a \in \mathbb{Z}$, se $\mdc(a, p) = 1$ e $x^2 \equiv a \pmod p$ não tem solução então:
        \begin{equation*}
            (p - 1)! \equiv a^{\frac{p-1}{2}} \pmod{p}
        \end{equation*}
    \end{lema}

    % \begin{teorema}[\textit{Teorema de Wilson}] \label{teorema : wilson}
    %     Seja número composto um número que pode ser escrito como a multiplicação de dois outros números menores então, dado $n > 1$:
    %     \begin{equation*}
    %         (n - 1)! \equiv \begin{cases}
    %                         -1 \pmod{n} \; \textit{se $n$ é primo} \\
    %                         0 \pmod{n} \; \textit{se $n$ é composto e $n \neq 4$}
    %                         \end{cases}
    %     \end{equation*}
    % \end{teorema}
        
    % Em que, o Lema \ref{{lema : modp-1fat}} junto ao Critério de Euler foram provados por Laurent Théry durante o período de realização deste trabalho, enquanto o Teorema de Wilson já se encontrava na biblioteca.

\end{frame}

\begin{frame}[fragile]{Congruência de Grau 2 e Símbolos de Legendre}

    % Para a prova do Critério de Euler, tanto na biblioteca Mathematical Components quanto na prova manual apresentada neste trabalho foram necessários os seguintes teoremas: 

    % \begin{lema} \label{lema : modp-1fat}
    % Seja $p > 2$ um número primo, para todo $a \in \mathbb{Z}$, se $\mdc(a, p) = 1$ e $x^2 \equiv a \pmod p$ não tem solução então:
    %     \begin{equation*}
    %         (p - 1)! \equiv a^{\frac{p-1}{2}} \pmod{p}
    %     \end{equation*}
    % \end{lema}

    \begin{teorema}[\textit{Teorema de Wilson}] \label{teorema : wilson}
        Seja número composto um número que pode ser escrito como a multiplicação de dois outros números menores então, dado $n > 1$:
        \begin{equation*}
            (n - 1)! \equiv \begin{cases}
                            -1 \pmod{n} \; \textit{se $n$ é primo} \\
                            0 \pmod{n} \; \textit{se $n$ é composto e $n \neq 4$}
                            \end{cases}
        \end{equation*}
    \end{teorema}
        
    em que, o Lema \ref{lema : modp-1fat} junto a uma versão do Critério de Euler para números naturais foram provados por Laurent Théry durante o período de realização deste trabalho, enquanto o Teorema de Wilson já se encontrava na biblioteca.

\end{frame}

