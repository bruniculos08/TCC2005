\begin{frame}[fragile]{Propriedades de Congruência}
    \begin{enumerate}
        \item (\textit{Reflexividade}) $a \equiv a \pmod{n}$
        \item (\textit{Simetria}) $a \equiv b \pmod{n} \Longrightarrow b \equiv a \pmod{n}$
        \item (\textit{Transitividade})
            \begin{equation*}
                a \equiv b \pmod{n} \; \land \; b \equiv c \pmod{n} \Rightarrow a \equiv c \pmod{n}
            \end{equation*}
        \item (\textit{Compatibilidade com a soma})
            \begin{equation*}
                a \equiv b \pmod{n} \; \land \; c \equiv d \pmod{n} \Rightarrow a + c \equiv b + d \pmod{n}
            \end{equation*}
        \item (\textit{Compatibilidade com a diferença})
            \begin{equation*}
                a \equiv b \pmod{n} \; \land \; c \equiv d \pmod{n} \Rightarrow a - c \equiv b - d \pmod{n}
            \end{equation*}
        % \item \label{item:propcong6-produto} (\textit{Compatibilidade com o produto})
        % \begin{equation*}
        %     a \equiv b \pmod{n} \land c \equiv d \pmod{n} \Longrightarrow a \cdot c \equiv b \cdot d \pmod{n}
        % \end{equation*}
        % A partir dessa propriedade, note que, para todo $k \in \mathbb{N}$:
        % \begin{equation*}
        %     a \equiv b \pmod{n} \Longrightarrow a^k \equiv b^k \pmod{n}
        % \end{equation*}
        % \item \label{item:propcong7-cancelamento} (\textit{Cancelamento}) $\mdc(c, n) = 1 \Longrightarrow (a \cdot c \equiv b \cdot c \pmod{n} \Longleftrightarrow a \equiv b \pmod{n})$   
    \end{enumerate}
\end{frame}

\begin{frame}[fragile]{Propriedades de Congruência}
    \begin{enumerate}
          \setcounter{enumi}{5}
        \item \label{item:propcong6-produto} (\textit{Compatibilidade com o produto})
        \begin{equation*}
            a \equiv b \pmod{n} \land c \equiv d \pmod{n} \Longrightarrow a \cdot c \equiv b \cdot d \pmod{n}
        \end{equation*}
        A partir dessa propriedade, note que, para todo $k \in \mathbb{N}$:
        \begin{equation*}
            a \equiv b \pmod{n} \Longrightarrow a^k \equiv b^k \pmod{n}
        \end{equation*}
        \item \label{item:propcong7-cancelamento} (\textit{Cancelamento}) 
        \begin{equation*}
            \mdc(c, n) = 1 \Longrightarrow (a \cdot c \equiv b \cdot c \pmod{n} \Longleftrightarrow a \equiv b \pmod{n})
        \end{equation*}
    \end{enumerate}
\end{frame}