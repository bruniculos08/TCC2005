\begin{frame}{Conclusão}
    Neste trabalho foi feito(a):
    \begin{itemize}
        \item Implementação do \textit{símbolo de Legendre};
        \item Prova de corretude da função que representa \textit{símbolo de Legendre};
        \item Prova das propriedades sobre o \textit{símbolo de Legendre} apresentadas em \cite{book:2399854};
        \item Alguns lemas simples relacionados a números primos inteiros (a biblioteca só trata de números primos naturais);
        
        \item[] \textbf{Obs.:} também foram implementados conteúdos em relação ao \textit{inverso multiplicativo módulo $n$}, porém tais conteúdos não foram utilizados na implementação do \textit{símbolo de Legendre} e podem ser inúteis, visto que a intenção em tê-los poderia ser satisfeita com o uso de \textit{corpos} (já implementados na biblioteca).
    \end{itemize}
\end{frame}

\begin{frame}{Conclusão}
    Quanto a trabalhos futuros:
    \begin{itemize}
        \item Com as implementações mostradas aqui e principalmente acom as implementação realizadas por Laurent Théry (considerando a possibilidade de tratar apenas de números naturais), é facilitada a implementação do algoritmo \textit{RESSOL} na biblioteca;
        \item As implementações (tanto feitas aqui quanto as de Laurent) junto ao conhecimento obtido pela análise da prova do enunciado \coqinline[]{fact_sqr_exp} (e a própria prova) sobre manipulação do operador \coqinline[]{bigop} e de conjuntos abre caminho para a prova da \textit{Lei de Reciprocidade Quadrática} (dada sua prova manual).
    \end{itemize}
\end{frame}
