\documentclass[xcolor=table]{beamer}
\usepackage[utf8]{inputenc}
\usepackage[T1]{fontenc}
\usepackage[alf]{abntex2cite}
\usepackage{Estilos/udesc}
\usepackage{amsfonts,amsmath,amssymb,mathtools}
\usepackage{verbatim}
% \usepackage{listings, lstCoq} % Highlighting de código Coq
\usepackage[ddmmyyyy]{datetime}
\usepackage{hyperref, url}
\usepackage{graphicx}
\usepackage{bussproofs}
\usepackage{multirow}
\usepackage{changepage}
\usepackage{xspace}
\usepackage[normalem]{ulem}

\definecolor{darkgreen}{rgb}{0,0.6,0}
\definecolor{darkorange}{rgb}{0.8,0.4,0.2}
\definecolor{darkred}{rgb}{0.8,0,0}

\usepackage{svg}
\setsvg{inkscapeexe=inkscape}
\setsvg{inkscapeopt=-z -D}

\makeatletter
\patchcmd{\beamer@sectionintoc}
  {\ifnum\beamer@tempcount>0}
  {\ifnum\beamer@tempcount>-1}
  {}
  {}
% \beamer@tocsectionnumber=-1
\beamer@tocsectionnumber=0
\makeatother
% \setcounter{section}{-1}
\setcounter{section}{0} % O João tinha iniciado no 0 por que era colóquio, mas creio que nas regras da ABNT começa em 1.

\newcommand{\overbar}[1]{\mkern 1.5mu\overline{\mkern-1.5mu#1\mkern-1.5mu}\mkern 1.5mu}

\newcommand{\uglyphi}{\phi} % mantendo o \phi velho
\renewcommand \phi{\varphi}
\let \emptyset \varnothing

\newcommand{\Ltac}{$\mathcal{L}$\unskip tac}

\graphicspath{{Figuras/}}
\setbeamertemplate{frametitle continuation}{}

% suprimindo warnings do hyperref
\pdfstringdefDisableCommands{%
  \def\\{}%
  \def\texttt#1{<#1>}%
  \def\smallskip{}%
  \def\medskip{}%
}

% Adições de Bruno:
% (0) Alguma coisa tipo \begin{equation} mas não sei qual a diferença:
\usepackage{mathpartir}

% (1) Meus estilos do Coq:
\usepackage{listings, Estilos/coq, Estilos/coq-error}

% (2) Para teoremas e etc.:
\newtheorem{definicao}{Definição}
\newtheorem{teorema}{Teorema}
\newtheorem{lema}{Lema}
\DeclareMathOperator{\mdc}{mdc}
\DeclareMathOperator{\mmc}{mmc}
\DeclareMathOperator{\ndiv}{$\hspace{-4pt}\not|\hspace{2pt}$}
\setbeamertemplate{theorems}[numbered] % para númerar automaticamente
\renewcommand\qedsymbol{$\blacksquare$}

% (3) Para poder colar lstinline em math mode:
\usepackage{etoolbox}
\expandafter\patchcmd\csname \string\lstinline\endcsname{%
        \leavevmode
        \bgroup
    }{%
        \leavevmode
        \ifmmode\hbox\fi
        \bgroup
    } {}
    {%
        \typeout{Patching of \string\lstinline\space failed!}%
}

% (4) Para os algoritmos:
\usepackage[portuguese,linesnumbered,boxruled,noend]{algorithm2e}

% (5) Fonte em algoritmos: 
\SetArgSty{textnormal}
\makeatletter
\let\original@algocf@latexcaption\algocf@latexcaption
\long\def\algocf@latexcaption#1[#2]{%
  \@ifundefined{NR@gettitle}{%
    \def\@currentlabelname{#2}%
    \def\@currentlabelname{\textsc{#2}}
  }{%
    \NR@gettitle{#2}%
  }%
  \original@algocf@latexcaption{#1}[{#2}]%
}
\makeatother

% (6) Alguma coisa de margem nos algoritmos:
\SetNlSty{textbf}{}{:}
\setlength{\algomargin}{2em}

% (7) Para separar o algoritmo RESSOL em dois frames:
\usepackage{trimclip}
\usepackage[bahasa]{babel}
\newsavebox\mydef
\tolerance=1
\emergencystretch=\maxdimen
\hyphenpenalty=10000
\hbadness=10000

% (8) Para arrumar o bug de referências:
% \usepackage{bookmark}

% (9) Para adicionar alguns símbolos:
\usepackage{fdsymbol}

% (10) Comandos para suprimir e reativar numeração em lstlisting:
\let\origthelstnumber\thelstnumber
\makeatletter
\newcommand*\Suppressnumber{%
\lst@AddToHook{OnNewLine}{%
    \let\thelstnumber\relax%
    \advance\c@lstnumber-\@ne\relax%
    }%
}

\newcommand*\Reactivatenumber{%
\lst@AddToHook{OnNewLine}{%
\let\thelstnumber\origthelstnumber%
\advance\c@lstnumber\@ne\relax}%
}
\makeatother

% (11) Comando para \vdots customizado:
\makeatletter
\DeclareRobustCommand{\rvdots}{%
\vbox{
	\baselineskip4\p@\lineskiplimit\z@
	\kern-\p@
	\hbox{.}\hbox{.}\hbox{.}
}}
\makeatother

% (12) Comandos para códigos em Coq:
\newcommand{\codequotes}{\lstinline[language=coq]|"|}
\newcommand{\coqinline}[1][]{\lstinline[language=coq]}

% Fim das adições de Bruno.

\renewcommand{\figurename}{Figura}
\renewcommand{\tablename}{Tabela}
\sloppy
\title[]{Formalização do Algoritmo RESSOL utilizando Coq}

\author[Bruno Rafael dos Santos]{
    Bruno Rafael dos Santos\\\smallskip
    {\scriptsize Universidade do Estado de Santa Catarina \\\smallskip
    \texttt{bruniculos2014@gmail.com
    }\\\medskip
    {Orientadora: Dra Karina Girardi Roggia}\\
    {Coorientador: Me Paulo Henrique Torrens}\\
    }
}

% \date{\today}
% \date{26/06/2024}
\date{29/11/2024}

\titlegraphic{\includegraphics[scale=.15]{Logo-Função.png}}
% Abaixo é o que o João tinha colocado:
    % \titlegraphic{Apoio:\\\includegraphics[scale=.15,keepaspectratio]{Logo-CNPq.png}}
    % \logo{\includegraphics[scale=.05,keepaspectratio]{Logo-Função.png}}

\begin{document}
    \begin{frame}
        \titlepage
    \end{frame}

    \begin{frame}{Sumário}
        % \tableofcontents
        \tableofcontents[sections=0-3]
        % \framebreak
        % \tableofcontents[sections=4-10]
    \end{frame}
    
    \begin{frame}{Sumário}
        \tableofcontents[sections=4-10]
    \end{frame}

    \section[]{Introdução}
    
        \begin{frame}{Introdução}
    \begin{itemize}
        \item A Teoria dos Números é um ramo da matemática que lida, em sua maior parte, com  propriedades de números inteiros;
        \item É muito presente em temas relacionados a criptografia;
        \item Envolve definições de diversas relações em $\mathbb{Z}$, sendo duas dessas as relações de divisibilidade e congruência;
        \item Neste contexto que se apresenta o \textit{símbolo de Legendre}, o qual possui relação com o algoritmo \textit{RESSOL} e está presente na \textit{Lei de Reciprocidade Quadrática}.
        % algoritmo \textit{RESSOL}, também conhecido como algoritmo de Tonelli-Shanks, e a Lei de Reciprocidade Quadrática;
        \item A seguir se apresentam as definições de divisibilidade e congruência.
    \end{itemize}
\end{frame}

\begin{frame}{Introdução}
    \begin{definicao}[\textit{Divisibilidade}]
            $\forall d, a \in \mathbb{Z}$, \textbf{$d$ divide $a$} (ou em outras palavras: $a$ é um múltiplo de $d$) se e somente se a seguinte proposição é verdadeira:
            \begin{equation*}
                \exists q \in \mathbb{Z}, a = d \cdot q
            \end{equation*}
            assim, se tal proposição é verdadeira e portanto $d$ divide $a$, tem-se a seguinte notação que representa tal afirmação:
            \begin{equation*}
                d \mid a
            \end{equation*}
            caso contrário, a negação de tal afirmação ($d$ não divide $a$) é representada por:
            \begin{equation*}
                d \nmid a
            \end{equation*}
    \end{definicao}
\end{frame}

\begin{frame}{Introdução}
    \begin{definicao}[\textit{Congruência}]
        Para todo $a, b, n \in \mathbb{Z}$, $a$ é congruente a $b$ módulo $n$ se e somente se, pela divisão euclidiana $\frac{a}{n}$ e $\frac{b}{n}$ (onde $0 \leq r_{a} < |n|$ e $0 \leq r_b < |n|$) tem-se
        \begin{equation*}
            a = n \cdot q_a + r_a
        \end{equation*}
        e
        \begin{equation*}
            b = n \cdot q_b + r_b
        \end{equation*}
        com $r_a = r_b$, o que também equivale a dizer que:
        \begin{equation*}
            n \mid a - b
        \end{equation*}
        tal relação entre os inteiros $a$, $b$ e $n$ é representada por:
        \begin{equation*}
            a \equiv b \pmod{n}
        \end{equation*}
    \end{definicao}
\end{frame}

% \begin{frame}{Introdução}
%     \begin{itemize}
%         \item Sendo $p$ um número primo e $r, n \in \mathbb{Z}$, uma congruência quadrática é uma equação da seguinte forma:
%         \begin{equation*}
%             r^2 \equiv n \pmod{p}
%         \end{equation*}
%         \item O objetivo do algoritmo \textit{RESSOL} é, tendo os valores de $p$ e $n$, computar um valor de $r$ que satisfaça tal equação;
%         \item Inicialmente, um algoritmo para resolução deste problema foi publicado em \cite{Tonelli1891};
%         \item Mais tarde foi publicada uma nova versão em \apud{danielShanks}{Maheswari}, que é a versão a ser tratada neste trabalho;
%         \item Algumas de suas aplicações são: \textit{Rabin Cryptosystem} \cite{Huynh1581080} e sistemas de criptografia que envolvem curvas elípticas \cite{PalashSarkar2024AdvancesinMathematicsofCommunications}.
%         % \cite{kumar2021algorithm} e \cite{7133812}.
%     \end{itemize}
% \end{frame}

% \begin{frame}{Introdução}
%     \begin{itemize}
%         \item Outro tema também abordado neste trabalho (porém não nesta apresentação) é a Lei de Reciprocidade Quadrática.
%     \end{itemize}
% \end{frame}
    
    \section[]{Objetivos}

        % \begin{frame}{Objetivo Geral (TCC2)}
\begin{frame}{Objetivo Geral}
    Implementar o \textit{símbolo de Legendre} e realizar a formalização de suas propriedades (apresentadas em \cite{book:2399854}) e da corretude (da função que o implementa). 
\end{frame}

\begin{frame}{Objetivos Específicos}
    \begin{enumerate}
            \item Obter conhecimentos avançados sobre o assistente de provas \textit{Coq}.
            \item Realizar o estudo sobre as principais documentações da biblioteca Mathematical Components.
            \item Desenvolver a capacidade de realizar provas em \textit{Coq} utilizando as táticas da linguagem de provas \textit{SSReflect}.
            \item Estudar conteúdos de Teoria dos Números relacionados ao \textit{símbolo de Legendre}.
            \item Implementar uma função que compute o valor do \textit{símbolo de Legendre} e provar a corretude da mesma se utilizando da biblioteca Mathematical Components.
            \item Provar teoremas úteis para manipulação de expressões envolvendo o \textit{símbolo de Legendre} utilizando-se da biblioteca Mathematical Components.
            % \item Realizar uma introdução para conceitos básicos de Teoria dos Números;
            % \item Explicar conteúdos sobre \textit{Coq} relacionados a biblioteca Mathematical Components e considerados relevantes para apresentação de determinadas implementações disponíveis nessa;
            % \item Explorar o conteúdo teórico de Teoria dos Números necessário para a formalização do algoritmo \textit{RESSOL} e da Lei de Reciprocidade Quadrática;
            % \item Apresentar a definicao, prova manual de terminação e de corretude do algoritmo \textit{RESSOL};
            % \item Apresentar a definicao e prova manual da Lei de Reciprocidade Quadrática.
    \end{enumerate}
\end{frame}

    % \section[]{Biblioteca Mathematical Components}

        % \include{Sections/Biblioteca/Biblioteca}
        % \subsection[]{Structures e Records} \label{sub:structures-records}
        % \include{Sections/Biblioteca/StructuresRecords}
        % \subsection[]{Comando Canonical} \label{sub:comando-canonical}
        % \include{Sections/Biblioteca/Canonical}
        % \subsection[]{Comando Coercion} \label{sub:comando-coercion}
        % \include{Sections/Biblioteca/Coercion}

    \section[]{Base Teórica} 

        \begin{frame}{Base Teórica}
    A seguir serão apresentados os principais teoremas, lemas e definições considerados úteis para a realização do objetivo estabelecido. Esse conteúdo se baseia no livro \cite{book:2399854} que foi amplamente estudado para realização deste trabalho.
\end{frame}
        \subsection[]{Propriedades de Congruência} \label{sub:prop-cong}
        \begin{frame}[fragile]{Propriedades de Congruência}
    \begin{enumerate}
        \item (\textit{Reflexividade}) $a \equiv a \pmod{n}$
        \item (\textit{Simetria}) $a \equiv b \pmod{n} \Longrightarrow b \equiv a \pmod{n}$
        \item (\textit{Transitividade})
            \begin{equation*}
                a \equiv b \pmod{n} \; \land \; b \equiv c \pmod{n} \Rightarrow a \equiv c \pmod{n}
            \end{equation*}
        \item (\textit{Compatibilidade com a soma})
            \begin{equation*}
                a \equiv b \pmod{n} \; \land \; c \equiv d \pmod{n} \Rightarrow a + c \equiv b + d \pmod{n}
            \end{equation*}
        \item (\textit{Compatibilidade com a diferença})
            \begin{equation*}
                a \equiv b \pmod{n} \; \land \; c \equiv d \pmod{n} \Rightarrow a - c \equiv b - d \pmod{n}
            \end{equation*}
        % \item \label{item:propcong6-produto} (\textit{Compatibilidade com o produto})
        % \begin{equation*}
        %     a \equiv b \pmod{n} \land c \equiv d \pmod{n} \Longrightarrow a \cdot c \equiv b \cdot d \pmod{n}
        % \end{equation*}
        % A partir dessa propriedade, note que, para todo $k \in \mathbb{N}$:
        % \begin{equation*}
        %     a \equiv b \pmod{n} \Longrightarrow a^k \equiv b^k \pmod{n}
        % \end{equation*}
        % \item \label{item:propcong7-cancelamento} (\textit{Cancelamento}) $\mdc(c, n) = 1 \Longrightarrow (a \cdot c \equiv b \cdot c \pmod{n} \Longleftrightarrow a \equiv b \pmod{n})$   
    \end{enumerate}
\end{frame}

\begin{frame}[fragile]{Propriedades de Congruência}
    \begin{enumerate}
          \setcounter{enumi}{5}
        \item \label{item:propcong6-produto} (\textit{Compatibilidade com o produto})
        \begin{equation*}
            a \equiv b \pmod{n} \land c \equiv d \pmod{n} \Longrightarrow a \cdot c \equiv b \cdot d \pmod{n}
        \end{equation*}
        A partir dessa propriedade, note que, para todo $k \in \mathbb{N}$:
        \begin{equation*}
            a \equiv b \pmod{n} \Longrightarrow a^k \equiv b^k \pmod{n}
        \end{equation*}
        \item \label{item:propcong7-cancelamento} (\textit{Cancelamento}) 
        \begin{equation*}
            \mdc(c, n) = 1 \Longrightarrow (a \cdot c \equiv b \cdot c \pmod{n} \Longleftrightarrow a \equiv b \pmod{n})
        \end{equation*}
    \end{enumerate}
\end{frame}
        \subsection[]{Função $\varphi$ de Euler} \label{sub:funcao-phi}
        \begin{frame}[fragile]{Função $\varphi$ de Euler}
    \begin{definicao}[Função $\varphi$ de Euler] Para quaisquer $n$ inteiro positivo, a função $\varphi(n)$ é definida como: 
        \begin{equation} \label{def:phi}
            \varphi(n) = |(\mathbb{Z}/(n))^{\times}|
        \end{equation}
    \end{definicao}
\end{frame}

\begin{frame}[fragile]{Função $\varphi$ de Euler}
    Algumas propriedades da função $\varphi$ de Euler são:
    \begin{enumerate}
        \item $\varphi(1) = \varphi(2) = 1$
        \item \label{item:prop-phi-2} $\forall n, n > 2 \Rightarrow 1 < \varphi(n) < n$
        \item \label{item:prop-phi-3} $\forall p,$ se $p$ é primo então $\forall k \in \mathbb{N} - \{0\}, \varphi(p^k) = p^k - p^{k-1}$, portanto, $\varphi(p) = p - 1$
        % \item \label{item:prop-phi-4} $\forall n, m \in \mathbb{N} - \{0\}, \mdc(n, m) = 1 \Rightarrow \varphi(n \cdot m) = \varphi(n) \cdot \varphi(m) $
        % \item \label{item:prop-phi-5} $\forall n \in \mathbb{N} - \{0\}$, se a fatoração de $n$ em potências de primos distintos é dada por $n = p_{1}^{\alpha_{1}} \cdot ... \cdot p_{k}^{\alpha_{k}}$, então:
        %     \begin{equation} \label{lema:phi-formula}
        %         \varphi(n) = \prod_{1 \leq i \leq k} \varphi(p_{i}^{\alpha_{i}}) = \prod_{1 \leq i \leq k} p_{i}^{\alpha_{i}} - p_{i}^{\alpha_{i} - 1} = n \cdot \prod_{1 \leq i \leq k} \left( 1 - \frac{1}{p_{i}} \right)
        %     \end{equation}
    \end{enumerate}
\end{frame}

\begin{frame}[fragile]{Função $\varphi$ de Euler}
    % Algumas propriedades da função $\varphi$ de Euler são:
    \begin{enumerate}
        \setcounter{enumi}{3}
        % \item $\varphi(1) = \varphi(2) = 1$
        % \item \label{item:prop-phi-2} $\forall n, n > 2 \Rightarrow 1 < \varphi(n) < n$
        % \item \label{item:prop-phi-3} $\forall p,$ se $p$ é primo então $\forall k \in \mathbb{N} - \{0\}, \varphi(p^k) = p^k - p^{k-1}$, portanto, $\varphi(p) = p - 1$
        \item \label{item:prop-phi-4} $\forall n, m \in \mathbb{N} - \{0\}, \mdc(n, m) = 1 \Rightarrow \varphi(n \cdot m) = \varphi(n) \cdot \varphi(m) $
        \item \label{item:prop-phi-5} $\forall n \in \mathbb{N} - \{0\}$, se a fatoração de $n$ em potências de primos distintos é dada por $n = p_{1}^{\alpha_{1}} \cdot ... \cdot p_{k}^{\alpha_{k}}$, então:
            % \begin{equation} \label{lema:phi-formula}
            %     \varphi(n) = \prod_{1 \leq i \leq k} \varphi(p_{i}^{\alpha_{i}}) = \prod_{1 \leq i \leq k} p_{i}^{\alpha_{i}} - p_{i}^{\alpha_{i} - 1} = n \cdot \prod_{1 \leq i \leq k} \left( 1 - \frac{1}{p_{i}} \right)
            % \end{equation}
            \begin{equation} \label{lema:phi-formula}
                \begin{split}
                    \varphi(n) \; & = \; \prod_{1 \leq i \leq k} \varphi(p_{i}^{\alpha_{i}}) \\ 
                    & = \; \prod_{1 \leq i \leq k} p_{i}^{\alpha_{i}} - p_{i}^{\alpha_{i} - 1} \\ 
                    & = \; n \cdot \prod_{1 \leq i \leq k} \left( 1 - \frac{1}{p_{i}} \right)
                \end{split}
            \end{equation}
    \end{enumerate}
\end{frame}

\begin{frame}[fragile]{Função $\varphi$ de Euler}
    \begin{teorema}[\textit{Teorema de Euler-Fermat}]
        \label{eq : euler-fermat}
        Para todo $a, m \in \mathbb{Z},$ se  $ m > 0$  e $\mdc(a,m) = 1$ então:
        \begin{equation*}
            a^{\varphi(m)} \equiv 1 \pmod{m}
        \end{equation*}
    \end{teorema}

    \begin{teorema}[\textit{Pequeno Teorema de Fermat}]
        \label{eq : pequeno-fermat}
        Para todo $a \in \mathbb{N} - \{0\}$, dado um número primo $p$, tem-se que:
        \begin{equation*}
            a^p \equiv a \pmod{p}
        \end{equation*}
    \end{teorema}
\end{frame}
        \subsection[]{Congruência de Grau 2 e Símbolos de Legendre} \label{sub:cong-grau2}
        \begin{frame}[fragile]{Congruência de Grau 2 e Símbolos de Legendre}
    Motivação sobre a resolução de congruências de grau 2:
    \vspace{4mm}
    \begin{itemize}
        \item Sendo $p$ um número primo maior que $2$ e $a,b,c \in \mathbb{Z}$ números não divisíveis por $p$, como motivação suponha que se deseje resolver a seguinte equação:
        \begin{equation} \label{eq : ax2bxc}
            a \cdot x^2 + b \cdot x + c \equiv 0 \pmod p
        \end{equation}
        Manipulando essa equação por meio das propriedades de congruência se obtém:
        \begin{equation} \label{eq : bhaskara}
            (2 \cdot a \cdot x + b)^2 \equiv b^2 - 4 \cdot a \cdot c \pmod p
        \end{equation}
    \end{itemize}
\end{frame}

\begin{frame}[fragile]{Congruência de Grau 2 e Símbolos de Legendre}
    Motivação sobre a resolução de congruências de grau 2: 
    (continuação)
    \vspace{4mm}
    \begin{itemize}
        \item Realizando a substituição $X = 2 \cdot a \cdot x + b$ e $d = b^2 - 4 \cdot a \cdot c$ na Equação \ref{eq : bhaskara}, tem-se:
        \begin{equation} \label{eq : quadcong}
            X^2 \equiv d \pmod p
        \end{equation}
        Portanto, resolver a Equação \ref{eq : ax2bxc} é equivalente a resolver a Equação \ref{eq : quadcong}.
    \end{itemize}
    \vspace{4mm}
    Sobre a Equação \ref{eq : quadcong}, se diz que $d$ é um quadrado perfeito em $\mathbb{Z}/(p)$ e também que $d$ é um \textit{resíduo quadrático módulo $p$}.

\end{frame}

\begin{frame}[fragile]{Congruência de Grau 2 e Símbolos de Legendre}
    Conforme \cite{book:2399854}, existem $\frac{p+1}{2}$ resíduos quadráticos módulo $p$, que são:
        \begin{equation} \label{eq : listquadres}
            0^2 \bmod{p}, 1^2 \bmod{p}, 2^2 \bmod{p}, ..., \left(\frac{p -1}{2} \right)^2 \bmod{p} 
        \end{equation}
    pois note que, para todo $x \in \mathbb{Z}$ existe algum $i \in [0, \frac{p-1}{2}]$ tal que $x \equiv i \pmod{p}$ ou $x \equiv -i \pmod{p}$, logo $x^2 \equiv i^2 \pmod{p}$ (usando a propriedade do Item \ref{item:propcong6-produto}) e $i^2$ está na Lista \ref{eq : listquadres}.
\end{frame}

\begin{frame}[fragile]{Congruência de Grau 2 e Símbolos de Legendre}
    Além disso, todos os os valores na Lista \ref{eq : listquadres} são distintos em módulo $p$, pois para todo $i, j \in \left[0, \frac{p-1}{2}\right]$:
    \begin{align}
        i^2 \equiv j^2 \pmod p
        &\begin{aligned}
            \;\; \Longleftrightarrow p \mid (i^2- j^2)
        \end{aligned} \\
        &\begin{aligned}
            \;\; \Longleftrightarrow p \mid (i -j)\cdot(i + j)
        \end{aligned} \\
        &\begin{aligned} \label{eq : ordivp}
            \;\; \Longleftrightarrow p \mid (i -j) \lor p \mid (i + j)
        \end{aligned}
    \end{align}
    Com isso, dado o intervalo de $i$ e $j$, então $0 \leq i + j \leq p - 1$, assim existem as seguintes possibilidades:
        \begin{enumerate}
            \item $i = j = 0$ e portanto $i \equiv j \pmod p$;
            \item $0 < i + j \leq p-1$, portanto $p \nmid i + j$, e então pela disjunção em \ref{eq : ordivp} resta que $p \mid (i - j)$, o que equivale a $i \equiv j \pmod p$, ou seja, $i$ é igual $j$ módulo $p$ se e somente se seus quadrados também são.
        \end{enumerate}
\end{frame}

\begin{frame}[fragile]{Congruência de Grau 2 e Símbolos de Legendre}
    Com essas conclusões (de que a Lista \ref{eq : listquadres} contém todos os resíduos quadráticos módulo $p$ e que todos os valores dela são distintos em módulo $p$) pode ser provado o seguinte lema:
        \begin{lema} \label{lema:existnonquadratic}
            Seja $p > 2$ um número primo, existem exatamente $\frac{p+1}{2}$ resíduos quadráticos módulo $p$ e $\frac{p-1}{2}$ resíduos não quadráticos módulo $p$.
        \end{lema}
\end{frame}

\begin{frame}[fragile]{Congruência de Grau 2 e Símbolos de Legendre}
    \begin{definicao}[Símbolo de Legendre]
        Seja $p > 2$ um número primo e $a \in \mathbb{Z}$, se define o símbolo de Legendre por:
        \begin{equation*}
            \left( \frac{a}{p} \right) = \begin{cases}
                1 \text{, se $p \nmid a$ e $a$ é um resíduo quadrático módulo $p$}
                \\
                0 \text{, se $p \mid a$}
                \\
                -1 \text{, caso contrário ($a$ não é um resíduo quadrático)}
                \end{cases}
        \end{equation*}
    \end{definicao}
\end{frame}

\begin{frame}[fragile]{Congruência de Grau 2 e Símbolos de Legendre}
    Pode-se dizer que a relação entre o \textit{símbolo de Legendre} e função $\varphi$ de Euler, que para um número primo $p$ é $\varphi(p) = p - 1$, se dá pelo seguinte teorema:
    \begin{teorema}[\textit{Critério de Euler}] Para todo $a \in \mathbb{Z}$, seja $p > 2$ um número primo, então: \label{teorema:criteriodeeuler}
        \begin{equation*}
            \left( \frac{a}{p} \right) \equiv a^{\frac{p-1}{2}} \pmod p
        \end{equation*}
    \end{teorema}
\end{frame}

\begin{frame}[fragile]{Congruência de Grau 2 e Símbolos de Legendre}

    Para a prova do Critério de Euler, tanto na versão feita por Laurent Théry (que auxilou na realização deste trabalho) quanto na prova manual apresentada neste trabalho foram necessários os seguintes teoremas: 

    \begin{lema} \label{lema : modp-1fat}
    Seja $p > 2$ um número primo, para todo $a \in \mathbb{Z}$, se $\mdc(a, p) = 1$ e $x^2 \equiv a \pmod p$ não tem solução então:
        \begin{equation*}
            (p - 1)! \equiv a^{\frac{p-1}{2}} \pmod{p}
        \end{equation*}
    \end{lema}

    % \begin{teorema}[\textit{Teorema de Wilson}] \label{teorema : wilson}
    %     Seja número composto um número que pode ser escrito como a multiplicação de dois outros números menores então, dado $n > 1$:
    %     \begin{equation*}
    %         (n - 1)! \equiv \begin{cases}
    %                         -1 \pmod{n} \; \textit{se $n$ é primo} \\
    %                         0 \pmod{n} \; \textit{se $n$ é composto e $n \neq 4$}
    %                         \end{cases}
    %     \end{equation*}
    % \end{teorema}
        
    % Em que, o Lema \ref{{lema : modp-1fat}} junto ao Critério de Euler foram provados por Laurent Théry durante o período de realização deste trabalho, enquanto o Teorema de Wilson já se encontrava na biblioteca.

\end{frame}

\begin{frame}[fragile]{Congruência de Grau 2 e Símbolos de Legendre}

    % Para a prova do Critério de Euler, tanto na biblioteca Mathematical Components quanto na prova manual apresentada neste trabalho foram necessários os seguintes teoremas: 

    % \begin{lema} \label{lema : modp-1fat}
    % Seja $p > 2$ um número primo, para todo $a \in \mathbb{Z}$, se $\mdc(a, p) = 1$ e $x^2 \equiv a \pmod p$ não tem solução então:
    %     \begin{equation*}
    %         (p - 1)! \equiv a^{\frac{p-1}{2}} \pmod{p}
    %     \end{equation*}
    % \end{lema}

    \begin{teorema}[\textit{Teorema de Wilson}] \label{teorema : wilson}
        Seja número composto um número que pode ser escrito como a multiplicação de dois outros números menores então, dado $n > 1$:
        \begin{equation*}
            (n - 1)! \equiv \begin{cases}
                            -1 \pmod{n} \; \textit{se $n$ é primo} \\
                            0 \pmod{n} \; \textit{se $n$ é composto e $n \neq 4$}
                            \end{cases}
        \end{equation*}
    \end{teorema}
        
    Em que, o Lema \ref{lema : modp-1fat} junto a uma versão do Critério de Euler para números naturais foram provados por Laurent Théry durante o período de realização deste trabalho, enquanto o Teorema de Wilson já se encontrava na biblioteca.

\end{frame}

      

    % \section[]{Algoritmo de Tonelli-Shanks (ou RESSOL)}
    
    % \include{Sections/AlgoritmoRessol/AlgoritmoRessol}
    % \subsection[]{Descrição do Algoritmo} \label{sub:descricao-alg}
    % \include{Sections/AlgoritmoRessol/Descricao}
    % \subsection[]{Prova Manual} \label{sub:prova-manual}
    % \include{Sections/AlgoritmoRessol/ProvaManual}
    
    \section[]{Implementação}

        \begin{frame}[fragile]{Formalização do Símbolo de Legendre}
    O \textit{símbolo de Legendre} foi implementado por meio da seguinte função:
        
        \begin{lstlisting}[language=coq,frame=single,tabsize=1]
Definition legendre_symb {p : int} (pL2 : (2 < p)%R) 
    (pP : primez.primez p) (a : int) :=
        if (p %| a)%Z then 0%Z else if (resz_quad p a)
        then 1%Z else (-1)%Z.
        \end{lstlisting}
        
        onde \coqinline[]{resz_quad} tem a seguinte definição (que por sua vez é baseada na definição de \coqinline[]{res_quad}):
        
        \begin{lstlisting}[language=coq,frame=single,tabsize=1]
Definition resz_quad p a := 
    has (fun i => ((i * i)%:Z == a %[mod p])%Z) (iota 0 `|p|).
        \end{lstlisting}
    
\end{frame}

\begin{frame}[fragile]{Formalização do Símbolo de Legendre}
    Um exemplo de uso da função \coqinline[]{legendre_symb} é:
            \begin{lstlisting}[language=coq,frame=single,tabsize=1]
Compute (legendre_symb (_ : 2 < 7)%R 
    (_ : primez.primez 7) 2).
            \end{lstlisting}

    A prova de corretude foi implementada usando o tipo indutivo \coqinline[]{reflect} e divido em duas partes por fins práticos (uso em táticas da \textit{ssreflect}).

\end{frame}

\begin{frame}[fragile]{Formalização do Símbolo de Legendre}
    Quanto a prova de corretude, esta foi implementada usando o tipo indutivo \coqinline[]{reflect} e foi divida em duas partes por fins práticos (uso em táticas da \textit{ssreflect}):
    \newline
        \begin{lstlisting}[language=coq,frame=single,tabsize=1, escapechar=@]
Theorem legendre_symbP {p : int} (pL2 : (2 < p)%R) 
    (pP : primez.primez p) (a : int):
    reflect (exists x, x^2 = a %[mod p]) 
    (if (p %| a)%Z then (((legendre_symb pL2 pP a) == 0)) 
    else ((legendre_symb pL2 pP a) == 1)).
        \end{lstlisting}
            
        \begin{lstlisting}[language=coq,frame=single,tabsize=1, escapechar=@]
Theorem legendre_symbnP {p : int} (pL2 : (2 < p)%R) (pP : primez.primez p) (a : int):
    reflect (~ exists x, x^2 = a %[mod p]) 
    ((legendre_symb pL2 pP a) == -1).
        \end{lstlisting}
\end{frame}

\begin{frame}[fragile]{Formalização do Símbolo de Legendre}
    Foram provadas todas as propriedades sobre o \textit{símbolo de Legendre} e uma versão do Critério de Euler idêntica apresentadas em \cite{book:2399854}, além de algumas propriedades auxiliares, mas por fim de breviedade serão mostradas aqui apenas os 2 primeiros grupos:
    \vspace{4mm}
    \\
    O enunciado do Critério de Euler em \textit{Coq} é dado por:
        \begin{lstlisting}[language=coq,frame=single,tabsize=1]
Lemma eulerz_criterion {p : int} (pL2 : (2 < p)%R) 
(pP : primez.primez p) (a : int):
    (a ^ ((p - 1) %/ 2)%Z = 
        (legendre_symb pL2 pP a) %[mod p])%Z.
        \end{lstlisting}

\end{frame}

\begin{frame}[fragile]{Formalização do Símbolo de Legendre}
    
    e considerando um número primo $p > 2$ e $a, b \in \mathbb{Z}$ tem-se as seguintes propriedades:
        
        \begin{itemize}
            \item $a \equiv b \pmod{p} \rightarrow \left(\frac{a}{p}\right) = \left(\frac{b}{p}\right)$, cuja declaração dada em \textit{Coq} é:
            \newline
                \begin{lstlisting}[language=coq,frame=single,tabsize=1]
Lemma legendre_symbE (p a b : int) (pL2 : (2 < p)%R) 
(pP : primez.primez p):
   (a == b %[mod p])%Z -> 
   ((legendre_symb pL2 pP a) == (legendre_symb pL2 pP b)).
                \end{lstlisting}

        \end{itemize}
\end{frame}


\begin{frame}[fragile]{Formalização do Símbolo de Legendre}

    \begin{itemize}
        \item $p \nmid a \rightarrow \left(\frac{a^2}{p}\right) = 1$, cuja declaração dada em \textit{Coq} é:
        \newline
            \begin{lstlisting}[language=coq,frame=single,tabsize=1]
Lemma legendre_symb_Ndvd (p a b : int) (pL2 : (2 < p)%R) 
(pP : primez.primez p):
~~(p %| a)%Z -> (legendre_symb pL2 pP (a^2)) == 1.
            \end{lstlisting}

        \item $\left(\frac{-1}{p}\right) = (-1)^{\frac{p - 1}{2}} = 1 \leftrightarrow p \equiv 1 \pmod{4}$, cuja declaração dada em \textit{Coq} é:
        \newline
            \begin{lstlisting}[language=coq,frame=single,tabsize=1]
Lemma legendre_symb_Neg1 (p : int) (pL2 : (2 < p)%R) 
(pP : primez.primez p):
((legendre_symb pL2 pP (-1)) == 1) = (p == 1 %[mod 4])%Z.
            \end{lstlisting}

    \end{itemize}

\end{frame}

\begin{frame}[fragile]{Formalização do Símbolo de Legendre}

    \begin{itemize}
        \item $\left(\frac{a \cdot b}{p}\right) = \left(\frac{a}{p}\right) \cdot \left(\frac{b}{p}\right)$, cuja declaração dada em \textit{Coq} é:
        \newline
            \begin{lstlisting}[language=coq,frame=single,tabsize=1]
Lemma legendre_symb_mul (p a b : int) (pL2 : (2 < p)%R) (pP : primez.primez p):
    (legendre_symb pL2 pP (a * b)%R) = 
    ((legendre_symb pL2 pP a) * 
        (legendre_symb pL2 pP b))%R.
            \end{lstlisting}
    \end{itemize}

\end{frame}

    \section[]{Conclusões}

        % \chapter{Conclusão}
\label{cap:conclusao}

O \textit{símbolo de Legendre} junto aos conteúdos relacionados e necessários para sua formalização são elementos que possuem aplicações relevantes, pois são o caminho para o desenvolvimente de trabalhos sobre temas como o algoritmo \hyperref[algo:ressol]{\textsc{Ressol}} e a \textit{Lei de Reciprocidade Quadrática}. A exemplo estes dois conteúdos estão relacionados a aplicações como curvas elípticas que são amplamente utilizadas em criptografia. Além disso, esses não são conteúdos isolados, no sentido de que são parte do caminho para outros.

    No presente trabalho foram implementados diversos teoremas 

Apesar desses fatos, esta área especifica de Teoria dos Números não chegou (até o momento) a ser explorada na biblioteca Mathematical Components, o que é de se esperar que aconteça com diversos temas, dado que abranger toda a matemática já desenvolvida é algo difícil, se não impossível. Tal vácuo de conteúdo com relação à área citada é um problema que, após esse trabalho, se torna muito mais viável de ser resolvido em trabalhos futuros, tendo o conhecimento disponibilizado neste trabalho como base. Esta viabilidade muito se dá à possibilidade de que iniciantes no uso da biblioteca Mathematical Components tenham uma curva de aprendizado mais suave, dado que muitos detalhes muitas vezes não discutidos na documentação da biblioteca são, neste trabalho, discutidos.

Com relação especificimente ao tema sobre a \textit{Lei de Reciprocidade Quadrática}, uma implentação deste tema voltada para a bibloteca Mathematical Components ganha com este trabalho uma maior possibilidade de ser realizada. Isto ocorre pois neste trabalho é dada a implentação do \textit{símbolo de Legendre} em \textit{Coq} e a explicação sobre uma prova (na Seção \ref{sec:implementacoes}) que envolve (principalmente no que consta a Teoria de Conjuntos) elementos úteis (da biblioteca Mathematical Components) para a prova da \textit{Lei de Reciprocidade Quadrática} (haja vista o que se tem na prova manual apresentada em \ref{cap:reciprocidadequadratica}).


   
    \section[]{Referências}
    \begin{frame}[allowframebreaks]{Referências}
        \bibliography{referencias}
    \end{frame}

\end{document}