\setlength{\absparsep}{18pt}
\begin{resumo}

        \noindent
        O ramo da matemática conhecido como Teoria dos Números tem grande influência nos campos de estudo da Ciência da Computação, apresentando diversos algoritmos e teoremas relacionados principalmente à criptografia. Não isoladamente, como em todos os ramos da matemática, as formalizações e provas de conceitos desta área são essenciais para o seu desenvolvimento. Para isso, o presente trabalho busca contribuir com esses itens por meio de métodos formais utilizando o assistente de provas \textit{Coq} e estabelecendo, como objeto de implementação, a função conhecida pelo nome de \textit{símbolo de Legendre} e parte de suas propriedades. Além disso, se pretende utilizar nesta implementação, a biblioteca Mathematical Components, a fim de que o resultado deste trabalho possa servir como contribuição para a mesma.\\

        % O ramo da matemática conhecido como Teoria dos Números tem grande influência nos campos de estudo da Ciência da Computação, apresentando diversos algoritmos e teoremas relacionados principalmente à criptografia. Não isoladamente, como em todos os ramos da matemática, as formalizações e provas de conceitos desta área são essenciais para o seu desenvolvimento. Para isso, o presente trabalho busca contribuir com esses itens por meio de métodos formais utilizando o assistente de provas Coq e estabelecendo, como objeto de implementação, os seguintes conteúdos: o algoritmo \hyperref[algo:ressol]{\textsc{Ressol}} e a \textit{Lei de Reciprocidade Quadrática}. Além disso, se pretende utilizar nesta implementação, a biblioteca Mathematical Components, a fim de que o resultado deste trabalho possa servir como contribuição para a mesma.\\

        % O ramo da matemática conhecido como Teoria dos Números tem grande influência nos campos de estudo da Ciência da Computação, apresentando diversos algoritmos e teoremas relacionados à criptografia e à Teoria da Computação. Dentre os conteúdos relacionados estão o embasamento por trás do sistema RSA e o problema de fatoração em primos. Este último, além de fazer parte da justificativa de segurança sobre o RSA, é um tema que está muito presente nas discussões de Teoria da Computação na atualidade, visto que parece possível a não existência de um algoritmo determinístico de tempo polinomial (em outras palavras, de uma máquina de Turing determinística que resolva o problema em tempo polinomial) para este problema e esta não existência obviamente implica que $P \neq NP$ \cite{book:2399854}.  O presente trabalho busca apresentar formalizações e provas sobre algoritmos e teoremas em Teoria dos Números utilizando o assistente de provas Coq. \\
        
        % \noindent
        \textbf{Palavras-chave:} \textit{símbolo de Legendre}, criptografia, Teoria dos Números, \textit{Coq}, \textit{Lei de Reciprocidade Quadrática}.
        % \textbf{Palavras-chave:} criptografia, Teoria dos Números, \textit{símbolo de Legendre}, Algoritmo de Tonelli-Shanks, Algoritmo \hyperref[algo:ressol]{\textsc{Ressol}}, Coq, \textit{Lei de Reciprocidade Quadrática}.
        % Reciprocidade Quadrática
\end{resumo}