\chapter{Conclusão}
\label{cap:conclusao}

O \textit{símbolo de Legendre} junto aos conteúdos relacionados e necessários para sua formalização são elementos que possuem aplicações relevantes, pois são o caminho para o desenvolvimente de trabalhos sobre temas como o algoritmo \hyperref[algo:ressol]{\textsc{Ressol}} e a \textit{Lei de Reciprocidade Quadrática}. A exemplo estes dois conteúdos estão relacionados a aplicações como curvas elípticas que são amplamente utilizadas em criptografia. Além disso, esses não são conteúdos isolados, no sentido de que são parte do caminho para outros.

    No presente trabalho foram implementados diversos teoremas 

Apesar desses fatos, esta área especifica de Teoria dos Números não chegou (até o momento) a ser explorada na biblioteca Mathematical Components, o que é de se esperar que aconteça com diversos temas, dado que abranger toda a matemática já desenvolvida é algo difícil, se não impossível. Tal vácuo de conteúdo com relação à área citada é um problema que, após esse trabalho, se torna muito mais viável de ser resolvido em trabalhos futuros, tendo o conhecimento disponibilizado neste trabalho como base. Esta viabilidade muito se dá à possibilidade de que iniciantes no uso da biblioteca Mathematical Components tenham uma curva de aprendizado mais suave, dado que muitos detalhes muitas vezes não discutidos na documentação da biblioteca são, neste trabalho, discutidos.

Com relação especificimente ao tema sobre a \textit{Lei de Reciprocidade Quadrática}, uma implentação deste tema voltada para a bibloteca Mathematical Components ganha com este trabalho uma maior possibilidade de ser realizada. Isto ocorre pois neste trabalho é dada a implentação do \textit{símbolo de Legendre} em \textit{Coq} e a explicação sobre uma prova (na Seção \ref{sec:implementacoes}) que envolve (principalmente no que consta a Teoria de Conjuntos) elementos úteis (da biblioteca Mathematical Components) para a prova da \textit{Lei de Reciprocidade Quadrática} (haja vista o que se tem na prova manual apresentada em \ref{cap:reciprocidadequadratica}).

