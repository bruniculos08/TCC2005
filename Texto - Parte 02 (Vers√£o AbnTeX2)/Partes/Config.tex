% ---- Arquivo com as configurações do PDF

% Alteração para fonte de capítulos:
\renewcommand{\ABNTEXchapterfont}{\normalfont}
\renewcommand{\ABNTEXchapterfont}{\bfseries}

% Definindo a cor azul em RGB:
\definecolor{blue}{RGB}{41,5,195}

% Informações do PDF:
\makeatletter
\hypersetup{
		pdftitle={\@title}, 
		pdfauthor={\@author},
    	pdfsubject={\imprimirpreambulo},
	    pdfcreator={Bruno Rafael dos Santos},
		pdfkeywords={Algoritmo \textit{RESSOL}}{Algoritmo de Tonelli-Shanks}{Lei de Reciprocidade Quadrática}{abntex2}{trabalho acadêmico}, 
		colorlinks=true,
    	linkcolor=blue,
    	citecolor=blue,
    	filecolor=magenta,
		urlcolor=blue,
		bookmarksdepth=4
}
\makeatother

% Posiciona figuras e tabelas no topo da página quando adicionadas sozinhas em uma página em branco (ver https://github.com/abntex/abntex2/issues/170):
\makeatletter
\setlength{\@fptop}{5pt} 
\makeatother

% Possibilita criação de Quadros e Lista de quadros (ver https://github.com/abntex/abntex2/issues/176):
\newcommand{\quadroname}{Quadro}
\newcommand{\listofquadrosname}{Lista de quadros}
\newfloat[chapter]{quadro}{loq}{\quadroname}
\newlistof{listofquadros}{loq}{\listofquadrosname}
\newlistentry{quadro}{loq}{0}

% Configurações para atender às regras da ABNT
\setfloatadjustment{quadro}{\centering}
\counterwithout{quadro}{chapter}
\renewcommand{\cftquadroname}{\quadroname\space} 
\renewcommand*{\cftquadroaftersnum}{\hfill--\hfill}

\setfloatlocations{quadro}{hbtp} % Ver https://github.com/abntex/abntex2/issues/176

% O tamanho do parágrafo é dado por:
\setlength{\parindent}{1.3cm}

% Controle do espaçamento entre um parágrafo e outro:
\setlength{\parskip}{0.2cm}  % tente também \onelineskip

% Configurações adicionadas por Bruno:
	% % Do .tex da UDESC:
	% 	% Comando para inverter sobrenome e nome
	% 	\newcommand{\invertname}[1]{%
	% 	\StrBehind{#1}{{}}, \StrBefore{#1}{{}}%
	% 	}%  
	% Teoremas e etc:
		\newtheorem{definição}{Definição}
		\newtheorem{teorema}{Teorema}
		\newtheorem{lema}{Lema}
	% Operadores e etc:
		\usepackage{amsmath, amsfonts, amssymb}
		\usepackage{mathtools}
		\DeclareMathOperator{\mdc}{mdc}
		\DeclareMathOperator{\mmc}{mmc}
		\DeclarePairedDelimiter\abs{\lvert}{\rvert}
		\usepackage{proof}
		\usepackage{mathpartir}
		\newcommand{\qed}{\hfill $\blacksquare$}
	% Highlight de código em coq:
		\definecolor{violet}{RGB}{80,5,100}
		\definecolor{teal}{RGB}{0,128,128}
		\definecolor{orange}{RGB}{255,128,13}
		\definecolor{darkgreen}{RGB}{0,100,0}
		\definecolor{darkred}{RGB}{100,0,0}
		\definecolor{dkpink}{RGB}{231,84,128}
		\usepackage{listings, Estilos/coq, Estilos/coq-error}
	% Algoritmos:
		\usepackage[portuguese,linesnumbered,boxruled,noend]{algorithm2e}
		\usepackage{hyperref}
		\SetArgSty{textnormal}
		\SetNlSty{textbf}{}{:}
		\setlength{\algomargin}{2.5em}
		\SetKwInput{Entrada}{Entrada}
		\SetKwInput{Saida}{Sa\'{i}da}
		\SetKw{Retorna}{retorna}
		\SetKwFor{Enqto}{enquanto}{faça}{endw}
		\SetKwIF{Se}{eSe}{SeN}{se}{então}{e se}{senão}{fimse}
		% Comandos para suprimir e reativar numeração em lstlisting:
		\let\origthelstnumber\thelstnumber
		\makeatletter
		\newcommand*\Suppressnumber{%
		\lst@AddToHook{OnNewLine}{%
			\let\thelstnumber\relax%
			\advance\c@lstnumber-\@ne\relax%
			}%
		}

		\newcommand*\Reactivatenumber{%
		\lst@AddToHook{OnNewLine}{%
		\let\thelstnumber\origthelstnumber%
		\advance\c@lstnumber\@ne\relax}%
		}
		\makeatother
		% Comando para \vdots customizado:
		\makeatletter
		\DeclareRobustCommand{\rvdots}{%
		\vbox{
			\baselineskip4\p@\lineskiplimit\z@
			\kern-\p@
			\hbox{.}\hbox{.}\hbox{.}
		}}
		\makeatother
