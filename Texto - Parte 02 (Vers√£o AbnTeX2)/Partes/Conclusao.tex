\chapter{Conclusão}
\label{cap:conclusao}

Tanto \textit{símbolo de Legendre} quanto os conteúdos relacionados e necessários para sua formalização são elementos que possuem aplicações relevantes, pois são o caminho para o desenvolvimento de trabalhos como o algoritmo \hyperref[algo:ressol]{\textsc{Ressol}} e a \textit{Lei de Reciprocidade Quadrática}. Tais conteúdos estão relacionados a aplicações como curvas elípticas  e \textit{zero-knowledge-proofs} e portanto não são temas isolados, no sentido de que são parte do caminho para outros.

Esta área específica de Teoria dos Números (\textit{símbolo de Legendre}) já foi formalizada porém fora da biblioteca Mathematical Components. Portanto com este trabalho tanto o conteúdo tratado aqui quanto o tratado nos apêndices pode começar a ser implementado na biblioteca. Isto em razão das formalizações e das análises realizadas para e sobre a biblioteca, neste trabalho.  

% Apesar desses fatos, esta área específica de Teoria dos Números não chegou (até o momento) a ser explorada na biblioteca Mathematical Components, o que é de se esperar que aconteça com diversos temas, dado que abranger toda a matemática já desenvolvida é algo difícil, se não impossível. Tal vácuo de conteúdo com relação à área citada é um problema que, após esse trabalho, se torna muito mais viável de ser resolvido em trabalhos futuros, tendo o conhecimento disponibilizado neste trabalho como base. Esta viabilidade muito se dá à possibilidade de que iniciantes no uso da biblioteca Mathematical Components tenham uma curva de aprendizado mais suave, dado que muitos detalhes muitas vezes não discutidos na documentação da biblioteca são, neste trabalho, aqui discutidos.

% No presente trabalho foram implementados diversos lemas e teoremas, dos quais apesar de uma grande parte não ter sido diretamente utilizada para a implementação final entregue neste trabalho, estes que acabaram não sendo utilizados ainda podem servir de contribuição para a biblioteca

Os conteúdos implementados neste trabalho podem ser dividos em duas partes: os que foram utilizados na implementação final deste trabalho e os que, apesar de não terem servido para tal podem também constituir uma contribuição para a biblioteca Mathematical Components. 
% Parte deste segundo grupo se encontra em \url{https://github.com/bruniculos08/mathcomp-contributions/blob/main/primez.v}. 
Quanto a implementação final (e o primeiro grupo), esta se trata do \textit{símbolo de Legendre} e algumas de suas propriedades. O \textit{símbolo de Legendre} foi implementado através de uma função computável e uma das partes mais relevantes para tal implementação, a prova do lema \lstinline[language=coq]|fact_sqr_exp|, implementada por Laurent Théry, foi altamente explorada neste trabalho, o que pode contribuir, por meio de conhecimento das táticas da biblioteca, para trabalhos futuros relacionados aos temas tratados nos apêndices. Tal implementação (do \textit{símbolo de Legendre}) se encontra em \url{https://github.com/bruniculos08/mathcomp-tcc/blob/main/legendre.v}. Quanto ao segundo grupo mencionado no inicio do parágrafo, este abrange todas as implementações fora do arquivo \textit{legendre.v} exceto por parte do conteúdo do arquivo \textit{primez.v} (este foi importado em \textit{legendre.v}).

Com relação especificimente a \textit{Lei de Reciprocidade Quadrática}, tema do Apêndice \ref{cap:reciprocidadequadratica}, uma implentação deste conteúdo voltada para a bibloteca Mathematical Components ganha com este trabalho uma maior possibilidade de ser realizada. Isto ocorre pois neste trabalho é dada a implentação do \textit{símbolo de Legendre} em \textit{Coq} e a explicação sobre uma prova (na Seção \ref{sec:implementacoes}) que envolve (principalmente no que consta a manipulação de somatórios e produtórios, e Teoria de Conjuntos) elementos úteis (da biblioteca Mathematical Components) para a prova da \textit{Lei de Reciprocidade Quadrática} (haja vista o que se tem na prova manual apresentada no Apêndice \ref{cap:reciprocidadequadratica}).

Resumindo o conteúdo entregue no presente trabalho se elenca então as partes relecionadas a cada um dos objetivos específicos apresentados em \ref{sec:obj-esp}: 
\begin{itemize}
    \item Os objetivos específicos \ref{item:obj-esp-1} e \ref{item:obj-esp-2} foram realizados nos Capítulos \ref{cap:mathcomp}, \ref{cap:implementacao} e parcialmente no Capítulo \ref{cap:base} com a apresentação de conteúdos implementados na biblioteca diretamente relacionados aos conceitos teóricos apresentados.  
    
    \item O objetivo específico \ref{item:obj-esp-3} foi tratado no Capítulo \ref{cap:implementacao} com a dicussão sobre a implementações mencionadas em \ref{sec:implementacoes} e com a produção de código para a formalização do \textit{símbolo de Legendre} em \ref{sec:form-legendre}.
    
    \item O objetivo \ref{item:obj-esp-4} foi efetivado para apresentação dos conteúdos do Capítulo \ref{cap:base} e dos Apêndices \ref{cap:tonelli-shanks} e \ref{cap:reciprocidadequadratica}.
    
    \item Os objetivos \ref{obj:func} e \ref{obj:proofs} foram tratados no Capítulo \ref{cap:implementacao}.
\end{itemize}

