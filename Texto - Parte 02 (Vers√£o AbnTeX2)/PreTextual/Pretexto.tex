\begin{folhadeaprovacao}
	\begin{center}
	  {\ABNTEXchapterfont\large\imprimirautor}
	  \vspace*{\fill}\vspace*{\fill}
	  \begin{center}
		\ABNTEXchapterfont\bfseries\Large\imprimirtitulo
	  \end{center}
	  \vspace*{\fill}
	  \hspace{.45\textwidth}
	  \begin{minipage}{.5\textwidth}
		  \imprimirpreambulo
	  \end{minipage}%
	  \vspace*{\fill}
	 \end{center}
		  
	 Trabalho aprovado. \imprimirlocal, 24 de novembro de 2012:
  
	 \assinatura{\textbf{\imprimirorientador} \\ Orientadora (Doutora)} 
	 \assinatura{\textbf{Cristiano Damiani Vasconcelos} \\ Doutor}
	 \assinatura{\textbf{Rafael Castro Gonçalves} \\ Mestre}
	 \assinatura{\textbf{Paulo Henrique Torrens} \\ Co-orientador}
		
	 \begin{center}
	  \vspace*{0.5cm}
	  {\large\imprimirlocal}
	  \par
	  {\large\imprimirdata}
	  \vspace*{1cm}
	\end{center}
	
\end{folhadeaprovacao}

% \begin{dedicatoria}
% 	\vspace*{\fill}
% 	\centering
% 	\noindent
% 	\textit{ Este trabalho é dedicado às crianças adultas que,\\
% 	quando pequenas, sonharam em se tornar cientistas.} \vspace*{\fill}
% \end{dedicatoria}

\begin{agradecimentos}
	Incialmente agradeço a Laurent Théry pelas implementações disponibilizadas em seu repositório, \hyperlink{https://github.com/thery/mathcomp-extra/}{\textit{mathcomp-extra}}, no \textit{GitHub} e pela atenção em responder aos e-mails de um garoto aleatório (eu) de outro canto deste mundo.

	Agradeço à professora Karina por ter me aceitado como orientando, pela atenção durante todos esses meses e pela ajuda na escrita do texto. Agradeço também ao professor Torrens, que se disponibilizou para responder minhas inúmeras mensagens e marcar diversas reuniões.

	Agradeço a todos os colegas da UDESC, mas em especial, aos colegas de laboratório, por tornar esta longa jornada mais leve e pelos momentos e conhecimentos compartilhados durante o caminho.

	Também agradeço a banca pela energia empenhada para ler o meu texto.

	Por fim faço um agradecimento a minha família, pai, mãe e irmão, por estarem junto comigo nos altos e baixos desses últimos anos e pela confiaça que mantiveram em mim. A eles dedico a epígrafe deste trabalho, não por que o texto mencionado tenha algum significado relevante por si só, mas por que lhes traz lembranças alegres das quais compartilhamos.
\end{agradecimentos}

% \begin{epigrafe}
% 	\vspace*{\fill}
% 	\begin{flushright}
%         \textit{"Em cima desse rato\\
%         tinha uma pulga...\\
%         Será possível?\\
%         Uma pulga acordada,\\
%         em cima de um rato dormitando,\\
%         em cima de um gato ressonando,\\
%         em cima de um cachorro cochilando,\\
%         em cima de um menino sonhando,\\
%         em cima de uma avó roncando,\\
%         numa cama aconchegante,\\
%         numa casa sonolenta,\\
%         onde todos viviam dormindo."\\}
%         \cite{casasonolenta}.
% 	\end{flushright}
% \end{epigrafe}

\begin{epigrafe}
	\vspace*{\fill}
	\begin{flushright}
        \textit{``Uma pulga acordada,\\
		que picou o rato,\\
		que assustou o gato,\\
		que arranhou o cachorro,\\
		que caiu sobre o menino,\\
		que deu um susto na avó,\\
		que quebrou a cama,\\
		numa casa sonolenta,\\
		onde ninguém mais estava dormindo.''\\}
        \cite{casasonolenta}.
	\end{flushright}
\end{epigrafe}