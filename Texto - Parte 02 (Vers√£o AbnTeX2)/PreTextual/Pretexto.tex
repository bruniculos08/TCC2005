\begin{folhadeaprovacao}
	\begin{center}
	  {\ABNTEXchapterfont\large\imprimirautor}
	  \vspace*{\fill}\vspace*{\fill}
	  \begin{center}
		\ABNTEXchapterfont\bfseries\Large\imprimirtitulo
	  \end{center}
	  \vspace*{\fill}
	  \hspace{.45\textwidth}
	  \begin{minipage}{.5\textwidth}
		  \imprimirpreambulo
	  \end{minipage}%
	  \vspace*{\fill}
	 \end{center}
		  
	 Trabalho aprovado. \imprimirlocal, 24 de novembro de 2012:
  
	 \assinatura{\textbf{\imprimirorientador} \\ Orientadora (Doutora)} 
	 \assinatura{\textbf{Cristiano Damiani Vasconcelos} \\ Doutor}
	 \assinatura{\textbf{Rafael Castro Gonçalves} \\ Mestre}
		
	 \begin{center}
	  \vspace*{0.5cm}
	  {\large\imprimirlocal}
	  \par
	  {\large\imprimirdata}
	  \vspace*{1cm}
	\end{center}
	
\end{folhadeaprovacao}

\begin{dedicatoria}
	\vspace*{\fill}
	\centering
	\noindent
	\textit{ Este trabalho é dedicado às crianças adultas que,\\
	quando pequenas, sonharam em se tornar cientistas.} \vspace*{\fill}
\end{dedicatoria}

\begin{agradecimentos}
	Incialmente agradeço a Laurent Théry pela provas disponibilizadas em \hyperlink{https://github.com/thery/mathcomp-extra/blob/master/euler.v}{seu GitHub} (principalmente as mencionadas no Capítulo \ref{cap:implementacao}), além de todas as suas contribuições feitas tanto para a biblioteca Mathematical Components quanto para outros temas de grande relevância acadêmica. Agradeço a Laurent também pela atenção em responder aos e-mails de um garoto (eu) aleatório de outro canto deste mundo. 
\end{agradecimentos}

\begin{epigrafe}
	\vspace*{\fill}
	\begin{flushright}
        \textit{"Em cima desse rato\\
        tinha uma pulga...\\
        Será possível?\\
        Uma pulga acordada,\\
        em cima de um rato dormitando,\\
        em cima de um gato ressonando,\\
        em cima de um cachorro cochilando,\\
        em cima de um menino sonhando,\\
        em cima de uma avó roncando,\\
        numa cama aconchegante,\\
        numa casa sonolenta,\\
        onde todos viviam dormindo."\\}
        \cite{casasonolenta}.
	\end{flushright}
\end{epigrafe}