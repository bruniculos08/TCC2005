% \setlength{\absparsep}{18pt}
\begin{resumo}[Abstract]
        \begin{otherlanguage*}{english}
                % \noindent
                The math field known as Number Theory has a great influence in the study fields from Computer Science, presenting a series of algorithms and theorems mainly related to cryptography. Not alone, as all the math fields, formalizations and proofs for concepts in this area are essencial for it's development. For that, the following work seeks to contribute for these items by means of formal methods, using the proof assistant Coq and establishing, as implementation objects, the following contents: the \hyperref[algo:ressol]{\textsc{Ressol}} algorithm and the \textit{Quadratic Reciprocity Law}. Furthermore, it's pretended to be used in these implementations, the library Mathematical Components, in order to make this work's result to serve as a contribution for the same.  
                
                
                % The math field known as Number Theory has great influence in the study fields from Computer Science, presenting a series of algorithms and theorems related to cryptography and Theory of Computation. Among these related subjects there are the basis behind the RSA system and the prime factoring problem. This last one, in addition to being part of the security justification for the RSA, is also a topic a very present topic in discussions about Theory of Computation currently, since it seems possible that does not exist a deterministic polynomial time algorithm (in other words, a deterministic Turing machine which solves such problem in polynomial time) to solve such problem and this fact obviously implies that $P \neq NP$ \cite{book:2399854}.
                % This assignment seeks to present formalization and proofs about algorithms and theorems in Number Theory using the Coq proof assistent. \\
                \vspace{\onelineskip}
                
                \noindent
                \textbf{Keywords:} cryptography, Number Theory, \textit{Legendre Symbol}, Tonelli-Shanks algorithm, \hyperref[algo:ressol]{\textsc{Ressol}} algorithm, Coq, \textit{Quadratic Reciprocity Law}.
                % Quadratic Rseciprocity
        \end{otherlanguage*}
\end{resumo}